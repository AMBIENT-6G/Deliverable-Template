
\chapter{Examples for a coherent document}\label{ch:deliverable-guidelines}
\contents{Revision done}{KUL}


\begin{itemize}
    \item Add yourself to the contributor list in \texttt{input/contributors.tex}
    \item Make use of the \textbf{glossaries} package to conveniently expand abbreviations in the correct manner. The abbreviations are located in \texttt{abbr.tex}.
    % \item Use \textbf{booktabs} (Gilles can do this in the first revision phase, if need be)
    \item Use distinct and descriptive labels to reference figures/tables/...
%     \item make use of short and long captions:
% \verb_\caption[this is shown in TOC]{this is shown at figure/table.}_
\item \textbf{British English} is preferred for EU projects.
\item Make use of \textit{tikzplotlib} (Python),  \textit{matlab2tikz} (Matlab) or use plain old CSV to transform eps/png/pdf figures to \textbf{tikz} figures which gives us more control over the aesthetics of the figures (coherent coloring scheme, font size, font family, size,...)
\item Try to make use of pre-defined commands to \textbf{format math notations}. These are listed in the \texttt{sty/math\_notations.tex} file. Example:
\begin{verbatim}
\usepackage{xifthen}
\newcommand{\prob}[1][]{%requires xifthen package%
\ifthenelse{\isempty{#1}}%
      {\ensuremath{P}}%
    {\ensuremath{P\left\(#1\right\)}}%
}
\newcommand{\vect}[1]{\boldsymbol{\mathrm{#1}}}
\newcommand{\mat}[1]{\boldsymbol{\mathrm{#1}}}
\newcommand{\MSE}{\mathrm{MSE}}
\newcommand{\tr}{\mathrm{tr}}
\newcommand{\moddef}{\mathrm{mod}}
\newcommand{\diag}{\mathrm{diag}}
\newcommand{\vecop}{\text{vec}}
\newcommand{\CP}{L}
\newcommand{\hddots}{\hdots}
\newcommand*{\inC}[1]{\in\mathbb{C}^{#1}}
\newcommand{\norm}[1]{\left\lVert#1\right\rVert}
\newcommand{\abs}[1]{\left\lvert#1\right\rvert}
\newcommand{\expt}[1]{\mathbb{E} \left\{#1\right\}}
\newcommand{\cn}[2]{\ensuremath{\sim\mathcal{C}\mathcal{N}\left(#1,#2\right)}}
\end{verbatim}
\item You can use \atkul to address a question to a certain partner. Please use your own comment command, so the referenced partner, knows who asked the question, made the remark, e.g., \nxp{has a question, \attug can you elaborate?} These definitions are located in the `sty/defs.tex' file.
\end{itemize}

\section{Using Colors}
\subsection{Color theme of the project}
\begin{tabular}{ll@{\hskip 4cm}l l}
      \cellcolor{pc1}\phantom{some} &  pc1&\cellcolor{pcolor}\phantom{some} &  pcolor\\
     \cellcolor{pc2}\phantom{some} &  pc2& \cellcolor{pcolordark}\phantom{some} &  pcolordark\\
     \cellcolor{pc3}\phantom{some} &  pc3 & \cellcolor{pcolorlight}\phantom{some} &  pcolorlight\\
     \cellcolor{pc4}\phantom{some} &  pc4\\ 
     \cellcolor{pc5}\phantom{some} &  pc5\\
     \cellcolor{pc6}\phantom{some} &  pc6\\     
\end{tabular}

\subsection{Color theme of the graphs}
\begin{tabular}{ll@{\hskip 4cm}l l}
      \cellcolor{color1}\phantom{some} &  color1&\cellcolor{color4}\phantom{some} &  color4\\
     \cellcolor{color2}\phantom{some} &  color2& \cellcolor{color5}\phantom{some} &  color5\\
     \cellcolor{color3}\phantom{some} & color3 & \cellcolor{color6}\phantom{some} &  color6\\
\end{tabular}


\section{Using Tikz}
Make use of tikz whenever possible. This allows us to have coherent coloring schemes and correct scaling and naming conventions. Examples:

\begin{figure}[h]
    \centering
    \setlength{\figurewidth}{0.8\linewidth}
    \setlength{\figureheight}{0.3\linewidth}
    % This file wcolor created with tikzplotlib v0.10.1.post9.
\begin{tikzpicture}

\begin{axis}[
legend style={nodes={scale=0.8, transform shape}}, 
width=\figurewidth,
height=\figureheight,
legend cell align={left},
legend style={
  fill opacity=0.8,
  draw opacity=1,
  text opacity=1,
  at={(0.03,0.97)},
  anchor=north west,
},
tick align=outside,
tick pos=left,
%xmin=-80.2, xmax=-38.8,
xmin=-80, xmax=-40,
xtick style={color=black},
%ymin=-0.05, ymax=1.05,
ymin=0, ymax=1.02,
ytick style={color=black},
tick align=outside,
tick pos=left,
xtick style={color=black},
ytick style={color=black},
grid style=dotted,
ymajorgrids,
xmajorgrids,
clip mode=individual,
ytick={0,0.2,0.4,0.6,0.8,1.0},
extra y ticks={0,0.1,0.2,0.3,0.4,0.5,0.6,0.7,0.8,0.9,1.0},
extra y tick label=\empty,
ylabel={\Acrshort{ecdf}},
xlabel={\Acrlong{rss} in dBm}
]
\addplot [thick, forget plot, color2, dashed]
table {%
-102.2 0
-86.71 0.0004762
-85.43 0.000697
-82.42 0.001208
-79.46 0.002436
-79.44 0.002478
-77.68 0.003747
-77.64 0.003782
-77.19 0.004272
-76.95 0.004589
-76.52 0.005121
-76.42 0.005238
-76.05 0.005756
-75.86 0.006101
-75.47 0.006736
-75.22 0.007198
-74.78 0.008137
-74.27 0.008992
-73.97 0.009634
-73.91 0.009745
-73.63 0.01035
-73.52 0.01056
-73.16 0.0115
-73.07 0.01171
-72.89 0.01218
-72.65 0.01277
-72.63 0.01284
-72.46 0.01331
-72.44 0.0134
-72.4 0.01357
-71.97 0.01492
-71.37 0.01714
-71.24 0.01771
-71.22 0.01778
-71.08 0.01838
-70.88 0.01929
-70.8 0.01972
-70.65 0.02055
-70.12 0.02301
-70.03 0.02352
-70 0.02366
-69.89 0.02411
-69.75 0.02488
-69.74 0.025
-69.55 0.02614
-69.52 0.02629
-69.44 0.02681
-69.42 0.02696
-69.32 0.02749
-69.27 0.02781
-69.21 0.02828
-69.15 0.0287
-69.1 0.0292
-69.03 0.02959
-68.93 0.03022
-68.91 0.03053
-68.77 0.03167
-68.36 0.03465
-68.33 0.03483
-68.26 0.03539
-68.19 0.03607
-67.51 0.04232
-66.86 0.04896
-66.85 0.04916
-66.82 0.04959
-66.77 0.05024
-66.72 0.05097
-66.63 0.05183
-66.57 0.05246
-66.55 0.05275
-66.4 0.05458
-65.38 0.06924
-65.36 0.06972
-65.31 0.07045
-64.99 0.07557
-64.95 0.07614
-64.88 0.07754
-64.83 0.07846
-64.77 0.07979
-64.2 0.09066
-64.11 0.09242
-64.09 0.09284
-64.07 0.09338
-64.03 0.0942
-63.94 0.09621
-63.93 0.09672
-63.88 0.09794
-63.85 0.09843
-63.84 0.09872
-63.82 0.09917
-63.7 0.1017
-63.66 0.1027
-63.64 0.1031
-63.59 0.1041
-63.56 0.1049
-63.53 0.1055
-63.51 0.1059
-63.48 0.1069
-63.45 0.1074
-63.34 0.1102
-63.32 0.1106
-62.91 0.1212
-62.89 0.1217
-62.84 0.123
-62.8 0.124
-62.63 0.1286
-62.62 0.1288
-62.59 0.1294
-62.55 0.1305
-62.46 0.1329
-62.45 0.1332
-62.43 0.1338
-62.41 0.1343
-62.38 0.1352
-62.27 0.1383
-62.24 0.1389
-62.13 0.1422
-62.1 0.143
-62.08 0.1435
-62.08 0.1439
-62.07 0.1443
-62.06 0.1451
-62.06 0.1457
-62.05 0.1469
-62.05 0.1476
-62.04 0.1498
-62.02 0.154
-62.02 0.155
-62.01 0.1566
-62.01 0.1574
-62 0.1603
-61.98 0.1689
-61.97 0.1704
-61.95 0.178
-61.94 0.1799
-61.94 0.1809
-61.93 0.184
-61.92 0.1859
-61.9 0.1933
-61.89 0.1961
-61.89 0.198
-61.86 0.2079
-61.84 0.212
-61.84 0.2127
-61.81 0.2232
-61.81 0.2238
-61.81 0.2245
-61.78 0.2326
-61.78 0.2336
-61.73 0.2452
-61.73 0.2463
-61.72 0.2489
-61.72 0.2495
-61.72 0.2505
-61.7 0.2563
-61.69 0.2576
-61.67 0.2633
-61.66 0.2641
-61.66 0.266
-61.65 0.2667
-61.65 0.2682
-61.61 0.2774
-61.6 0.279
-61.59 0.2799
-61.58 0.2826
-61.58 0.2835
-61.57 0.2849
-61.56 0.286
-61.56 0.2873
-61.55 0.2882
-61.55 0.289
-61.54 0.2906
-61.53 0.2921
-61.47 0.3033
-61.46 0.3041
-61.46 0.305
-61.45 0.3066
-61.45 0.3072
-61.44 0.3084
-61.34 0.3215
-61.34 0.3222
-61.33 0.3228
-61.32 0.3238
-61.31 0.3257
-61.25 0.3321
-61.22 0.3371
-61.21 0.3377
-61.2 0.3384
-61.18 0.339
-61.16 0.3397
-61.16 0.34
-61.14 0.3406
-61.1 0.3421
-61.09 0.3425
-61.07 0.343
-61 0.3453
-60.96 0.3466
-60.95 0.3472
-60.92 0.3481
-60.9 0.3488
-60.89 0.3491
-60.87 0.3495
-60.81 0.3517
-60.8 0.3521
-60.77 0.353
-60.72 0.3545
-60.71 0.355
-60.69 0.3556
-60.66 0.3567
-60.62 0.3582
-60.57 0.3601
-60.56 0.3606
-60.54 0.3611
-60.52 0.3619
-60.48 0.3633
-60.45 0.3646
-60.43 0.3652
-60.33 0.3686
-60.29 0.3704
-60.24 0.3719
-60.22 0.3729
-60.12 0.3764
-60.09 0.3779
-60 0.3811
-59.97 0.3822
-59.96 0.3828
-59.95 0.3833
-59.9 0.3852
-59.89 0.3856
-59.88 0.3862
-59.86 0.3868
-59.78 0.3902
-59.77 0.3907
-59.76 0.3911
-59.75 0.3916
-59.72 0.3928
-59.7 0.3936
-59.68 0.3946
-59.66 0.3952
-59.61 0.3974
-59.6 0.3978
-59.28 0.411
-59.23 0.4131
-59.16 0.4162
-59.14 0.4168
-59.13 0.4173
-59.12 0.4179
-59.07 0.42
-59.06 0.4206
-58.95 0.425
-58.94 0.4255
-58.93 0.4261
-58.84 0.4299
-58.83 0.4305
-58.81 0.4314
-58.71 0.4361
-58.7 0.4365
-58.66 0.4381
-58.65 0.4387
-58.63 0.4402
-58.61 0.4408
-58.34 0.4541
-58.32 0.4551
-58.31 0.4556
-58.27 0.4577
-58.26 0.4583
-58.26 0.4587
-58.25 0.4592
-58.23 0.4601
-58.22 0.4606
-58.15 0.4642
-58.13 0.4652
-58.12 0.4656
-58.08 0.4681
-58.06 0.4688
-57.9 0.4772
-57.89 0.4778
-57.81 0.4822
-57.79 0.4837
-57.78 0.4842
-57.76 0.485
-57.74 0.4861
-57.73 0.4866
-57.7 0.4886
-57.69 0.4891
-57.68 0.4893
-57.67 0.4897
-57.66 0.4903
-57.65 0.4909
-57.64 0.4914
-57.63 0.4919
-56.73 0.5443
-56.72 0.5448
-56.67 0.5475
-56.66 0.5481
-56.65 0.5487
-56.65 0.5492
-56.64 0.5497
-56.58 0.5536
-56.54 0.5557
-56.54 0.5562
-56.53 0.5567
-56.5 0.5585
-56.48 0.5592
-56.47 0.5601
-56.46 0.5604
-56.45 0.5609
-56.42 0.5627
-56.41 0.5631
-56.4 0.5638
-56.34 0.5672
-56.29 0.5697
-56.28 0.5702
-56.27 0.5709
-56.26 0.5717
-56.18 0.5759
-56.14 0.5787
-56.13 0.5792
-56.11 0.58
-56.11 0.5804
-55.94 0.5892
-55.92 0.5904
-55.91 0.591
-55.89 0.5924
-55.88 0.5929
-55.85 0.5945
-55.83 0.5955
-55.7 0.6029
-55.63 0.6071
-55.59 0.6092
-55.58 0.6096
-55.57 0.61
-55.56 0.6105
-55.53 0.6119
-55.26 0.626
-55.22 0.6288
-55.21 0.6293
-55.19 0.6301
-55.18 0.631
-54.55 0.6644
-54.54 0.6649
-54.45 0.6691
-54.43 0.67
-54.4 0.6718
-54.39 0.6722
-54.38 0.6728
-54.37 0.6735
-54.35 0.6746
-54.34 0.6751
-54.33 0.6756
-54.28 0.6783
-54.18 0.6828
-54.17 0.6832
-54.16 0.6836
-54.12 0.6856
-54.11 0.6861
-54.1 0.6869
-54.08 0.6875
-54.07 0.6884
-54.06 0.6889
-54.02 0.6905
-54.01 0.691
-53.85 0.6988
-53.85 0.6993
-53.82 0.7007
-53.79 0.7023
-53.78 0.7028
-53.76 0.7035
-53.63 0.7102
-53.62 0.7106
-53.6 0.7117
-53.59 0.7123
-53.57 0.7131
-53.56 0.7137
-53.55 0.7143
-53.54 0.7145
-53.53 0.7154
-53.51 0.7163
-53.32 0.7257
-53.31 0.7263
-53.2 0.7315
-53.19 0.732
-53.18 0.7327
-53.17 0.7332
-53.16 0.7335
-53.15 0.7341
-53.13 0.7349
-53.11 0.7359
-53.06 0.738
-53.05 0.7388
-53.03 0.7398
-53.02 0.7418
-53.01 0.7425
-53.01 0.7437
-53 0.7452
-52.98 0.7489
-52.97 0.7503
-52.95 0.7546
-52.95 0.7561
-52.93 0.7588
-52.88 0.7687
-52.87 0.7705
-52.86 0.7715
-52.85 0.7745
-52.84 0.776
-52.83 0.7779
-52.83 0.7785
-52.82 0.7803
-52.81 0.782
-52.78 0.7884
-52.75 0.7941
-52.75 0.7948
-52.73 0.7969
-52.72 0.7987
-52.72 0.8002
-52.71 0.8011
-52.7 0.8021
-52.64 0.8125
-52.64 0.8133
-52.64 0.8137
-52.63 0.8153
-52.62 0.8166
-52.61 0.8177
-52.6 0.8199
-52.59 0.8208
-52.58 0.8223
-52.57 0.8243
-52.57 0.8249
-52.49 0.8354
-52.48 0.8365
-52.47 0.8378
-52.46 0.839
-52.46 0.8396
-52.38 0.8479
-52.37 0.849
-52.37 0.8499
-52.36 0.8505
-52.36 0.851
-52.3 0.8564
-52.28 0.8581
-52.27 0.8587
-52.27 0.8594
-52.26 0.8598
-52.25 0.8606
-52.24 0.862
-52.19 0.8678
-52.06 0.8735
-52.05 0.874
-52.03 0.8747
-52 0.8761
-51.99 0.8766
-51.96 0.8778
-51.85 0.8828
-51.84 0.883
-51.43 0.9
-51.41 0.9007
-51.39 0.9014
-51.38 0.9019
-51.35 0.903
-51.33 0.9035
-51.25 0.9066
-51.24 0.9073
-51.22 0.9077
-51.2 0.9087
-51.17 0.9096
-51.12 0.9116
-51.06 0.9134
-51.05 0.9138
-50.83 0.9212
-50.81 0.9217
-50.71 0.9243
-50.7 0.9249
-50.66 0.9257
-50.64 0.9262
-50.61 0.9273
-50.58 0.928
-50.54 0.9292
-50.5 0.9301
-50.44 0.9315
-50.42 0.9321
-50.38 0.9331
-50.35 0.9338
-50.34 0.9343
-50.31 0.9349
-50.3 0.9353
-50.28 0.9359
-50.26 0.9363
-50.24 0.9369
-50.23 0.9371
-50.21 0.9377
-50.16 0.9392
-50.14 0.9397
-49.83 0.9473
-49.8 0.9479
-49.76 0.9487
-49.72 0.9496
-49.69 0.9502
-49.64 0.9516
-49.61 0.9521
-49.48 0.9547
-49.46 0.9552
-49.4 0.9562
-49.34 0.9571
-49.19 0.9597
-49.18 0.9599
-49.12 0.9608
-49.09 0.9613
-49.01 0.9628
-48.95 0.9638
-48.92 0.9643
-48.73 0.9672
-48.7 0.9676
-48.65 0.9682
-48.59 0.9692
-48.54 0.9699
-48.48 0.9707
-48.47 0.9708
-48.44 0.9713
-48.29 0.9732
-48.25 0.9737
-48.16 0.9746
-48.01 0.9764
-47.97 0.9768
-47.88 0.9778
-47.82 0.9784
-47.78 0.9789
-47.36 0.9829
-47.31 0.9833
-47.24 0.9839
-47.21 0.9842
-47.12 0.9848
-47.06 0.9853
-46.96 0.9859
-46.92 0.9863
-46.85 0.9868
-46.74 0.9875
-46.64 0.9881
-46.52 0.9889
-46.32 0.99
-46.19 0.9905
-45.51 0.9937
-45.39 0.9943
-44.55 0.9975
-44.4 0.998
-44.37 0.9981
-44.14 0.9987
-43.86 0.9995
-43.04 1
-42.86 1
};
%\addlegendentry{outside color2 - color2}
\addplot [thick, forget plot, color2]
table {%
-49.66 0
-48.59 0.0005822
-48.55 0.0009316
-47.98 0.001747
-47.97 0.00198
-47.89 0.002679
-47.7 0.00361
-47.64 0.004076
-47.59 0.004542
-47.58 0.004775
-47.46 0.005357
-47.4 0.006056
-47.38 0.006405
-47.25 0.007686
-47.25 0.007802
-47.14 0.008385
-47.12 0.008734
-47.06 0.0092
-47.02 0.009666
-47.02 0.009782
-46.99 0.01036
-46.9 0.01165
-46.9 0.01176
-46.87 0.01234
-46.85 0.01269
-46.81 0.01351
-46.8 0.01374
-46.76 0.01444
-46.75 0.01467
-46.7 0.01572
-46.69 0.01665
-46.66 0.01712
-46.6 0.01782
-46.52 0.0198
-46.52 0.02003
-46.44 0.02166
-46.39 0.02236
-46.39 0.02259
-46.33 0.02352
-46.33 0.02376
-46.31 0.02446
-46.29 0.0248
-46.26 0.02539
-46.2 0.02713
-46.2 0.02725
-46.19 0.02772
-46.18 0.02807
-46.17 0.02888
-46.17 0.02911
-46.14 0.02993
-46.14 0.03005
-46.06 0.03191
-46.03 0.03237
-46.02 0.03307
-45.99 0.03366
-45.96 0.0347
-45.92 0.03529
-45.92 0.03552
-45.9 0.03622
-45.88 0.03692
-45.87 0.03738
-45.82 0.03901
-45.78 0.03983
-45.78 0.03994
-45.76 0.04053
-45.76 0.04099
-45.76 0.04146
-45.72 0.04216
-45.72 0.04239
-45.69 0.04332
-45.68 0.04344
-45.67 0.0439
-45.61 0.04647
-45.61 0.04693
-45.6 0.04751
-45.5 0.05077
-45.5 0.05101
-45.46 0.05287
-45.44 0.05438
-45.43 0.05462
-45.4 0.05532
-45.4 0.05543
-45.38 0.05613
-45.36 0.05776
-45.34 0.05834
-45.33 0.05904
-45.31 0.06044
-45.3 0.06102
-45.29 0.06137
-45.28 0.06207
-45.27 0.06289
-45.27 0.06335
-45.24 0.06405
-45.24 0.06417
-45.23 0.06475
-45.22 0.06568
-45.21 0.06661
-45.18 0.06778
-45.18 0.06801
-45.16 0.06859
-45.15 0.06941
-45.14 0.06999
-45.11 0.07209
-45.11 0.0722
-45.09 0.07325
-45.09 0.07348
-45.06 0.0743
-45.05 0.075
-45.03 0.07639
-45.02 0.07698
-45.02 0.07709
-45.01 0.07756
-45.01 0.07791
-44.99 0.07884
-44.98 0.07966
-44.97 0.08035
-44.96 0.0807
-44.92 0.08327
-44.91 0.08385
-44.9 0.08466
-44.9 0.0849
-44.89 0.08559
-44.88 0.08664
-44.87 0.08711
-44.87 0.08792
-44.87 0.08839
-44.86 0.0892
-44.85 0.09025
-44.84 0.09118
-44.83 0.09188
-44.83 0.0927
-44.83 0.09375
-44.83 0.09479
-44.82 0.09794
-44.82 0.09992
-44.82 0.1006
-44.82 0.1013
-44.81 0.1019
-44.81 0.1028
-44.81 0.1052
-44.81 0.1064
-44.81 0.1074
-44.81 0.1082
-44.81 0.1087
-44.81 0.1091
-44.8 0.1133
-44.8 0.1155
-44.8 0.1166
-44.8 0.1174
-44.8 0.1187
-44.8 0.1194
-44.8 0.1201
-44.8 0.1208
-44.8 0.1217
-44.79 0.1223
-44.79 0.1234
-44.79 0.1259
-44.79 0.1266
-44.79 0.1279
-44.79 0.1315
-44.79 0.1321
-44.78 0.1333
-44.78 0.1338
-44.78 0.1367
-44.78 0.1404
-44.78 0.141
-44.77 0.1431
-44.77 0.1453
-44.77 0.1464
-44.77 0.148
-44.77 0.1493
-44.77 0.15
-44.76 0.1535
-44.76 0.1598
-44.76 0.1607
-44.76 0.1622
-44.76 0.1627
-44.76 0.1633
-44.74 0.1757
-44.74 0.1768
-44.74 0.1782
-44.74 0.1798
-44.74 0.1812
-44.74 0.1831
-44.73 0.1871
-44.73 0.1919
-44.73 0.1953
-44.72 0.1984
-44.72 0.2009
-44.72 0.205
-44.72 0.2055
-44.72 0.2061
-44.71 0.21
-44.71 0.2111
-44.71 0.2122
-44.71 0.214
-44.71 0.2166
-44.71 0.2186
-44.7 0.2193
-44.7 0.2201
-44.7 0.2243
-44.7 0.2249
-44.7 0.2253
-44.7 0.2263
-44.69 0.2287
-44.69 0.2296
-44.69 0.2349
-44.69 0.2355
-44.69 0.2376
-44.68 0.2383
-44.68 0.2414
-44.68 0.2437
-44.68 0.246
-44.68 0.2465
-44.68 0.2474
-44.68 0.2486
-44.67 0.2506
-44.67 0.2514
-44.67 0.2519
-44.67 0.2546
-44.67 0.2556
-44.67 0.2566
-44.66 0.2655
-44.66 0.2674
-44.66 0.2702
-44.65 0.271
-44.65 0.2723
-44.65 0.2739
-44.65 0.2746
-44.65 0.2756
-44.65 0.2783
-44.65 0.2797
-44.64 0.2888
-44.64 0.2938
-44.63 0.3009
-44.62 0.3059
-44.62 0.3083
-44.62 0.3087
-44.62 0.3104
-44.62 0.3116
-44.62 0.3121
-44.62 0.3137
-44.62 0.3155
-44.61 0.3176
-44.61 0.3192
-44.61 0.3204
-44.61 0.3222
-44.61 0.3236
-44.61 0.326
-44.61 0.3276
-44.6 0.3282
-44.6 0.3293
-44.6 0.3305
-44.6 0.3313
-44.6 0.3321
-44.6 0.3343
-44.6 0.335
-44.59 0.3389
-44.59 0.3438
-44.59 0.3444
-44.59 0.3453
-44.59 0.348
-44.58 0.349
-44.58 0.3516
-44.58 0.3529
-44.57 0.3603
-44.57 0.3611
-44.57 0.3657
-44.57 0.3666
-44.57 0.3673
-44.57 0.3679
-44.57 0.3715
-44.57 0.3729
-44.56 0.3737
-44.56 0.3749
-44.56 0.3791
-44.56 0.38
-44.56 0.3856
-44.55 0.3873
-44.55 0.3879
-44.55 0.3892
-44.55 0.3898
-44.55 0.3919
-44.55 0.3928
-44.55 0.3993
-44.54 0.4003
-44.54 0.4035
-44.54 0.404
-44.54 0.4053
-44.53 0.4154
-44.53 0.4188
-44.53 0.4199
-44.53 0.4223
-44.53 0.4232
-44.52 0.4238
-44.52 0.4274
-44.52 0.4301
-44.52 0.4308
-44.51 0.4402
-44.51 0.4418
-44.5 0.4522
-44.5 0.4527
-44.5 0.4546
-44.5 0.456
-44.49 0.4584
-44.49 0.4588
-44.49 0.4599
-44.49 0.4603
-44.49 0.462
-44.49 0.4658
-44.49 0.4663
-44.49 0.4672
-44.49 0.4678
-44.49 0.469
-44.48 0.4709
-44.48 0.4719
-44.48 0.4726
-44.48 0.473
-44.48 0.4755
-44.47 0.4868
-44.47 0.4872
-44.47 0.4898
-44.47 0.4914
-44.46 0.4939
-44.46 0.4977
-44.46 0.4995
-44.46 0.5003
-44.46 0.5016
-44.46 0.5066
-44.45 0.5075
-44.45 0.5082
-44.45 0.5087
-44.45 0.5123
-44.45 0.5133
-44.45 0.5139
-44.44 0.5194
-44.44 0.52
-44.44 0.5207
-44.44 0.5215
-44.44 0.5224
-44.44 0.5239
-44.44 0.5246
-44.44 0.5259
-44.43 0.5306
-44.43 0.5312
-44.43 0.5319
-44.43 0.5356
-44.43 0.5384
-44.42 0.5406
-44.42 0.5416
-44.42 0.5434
-44.42 0.5491
-44.42 0.5498
-44.42 0.5528
-44.41 0.5542
-44.41 0.5553
-44.41 0.5567
-44.41 0.5574
-44.41 0.5585
-44.41 0.5606
-44.41 0.5618
-44.41 0.5633
-44.4 0.565
-44.4 0.5661
-44.4 0.5678
-44.4 0.5689
-44.4 0.5719
-44.4 0.573
-44.39 0.5767
-44.39 0.5784
-44.39 0.5795
-44.39 0.5804
-44.39 0.581
-44.39 0.5867
-44.39 0.5877
-44.38 0.5883
-44.38 0.5947
-44.38 0.5974
-44.38 0.5981
-44.38 0.5993
-44.37 0.6003
-44.37 0.6009
-44.37 0.6052
-44.37 0.606
-44.37 0.607
-44.37 0.6084
-44.36 0.6117
-44.36 0.614
-44.36 0.615
-44.36 0.6156
-44.36 0.617
-44.36 0.6177
-44.36 0.6194
-44.35 0.6228
-44.35 0.6244
-44.35 0.6258
-44.35 0.6265
-44.35 0.6304
-44.35 0.6308
-44.34 0.632
-44.34 0.6333
-44.34 0.636
-44.34 0.6368
-44.34 0.6397
-44.34 0.6403
-44.34 0.6412
-44.33 0.6436
-44.33 0.6445
-44.33 0.6452
-44.33 0.646
-44.33 0.6466
-44.33 0.647
-44.33 0.6484
-44.33 0.6492
-44.33 0.6506
-44.33 0.6531
-44.32 0.6542
-44.32 0.6556
-44.32 0.657
-44.32 0.6579
-44.32 0.6606
-44.32 0.6619
-44.32 0.6637
-44.31 0.666
-44.31 0.6672
-44.31 0.6705
-44.31 0.6737
-44.3 0.6764
-44.3 0.677
-44.3 0.6779
-44.3 0.6809
-44.3 0.6814
-44.3 0.6821
-44.29 0.6853
-44.29 0.687
-44.29 0.6874
-44.28 0.6941
-44.28 0.6951
-44.28 0.6962
-44.28 0.6968
-44.28 0.6983
-44.28 0.699
-44.27 0.7022
-44.27 0.7033
-44.27 0.7046
-44.27 0.7056
-44.27 0.7063
-44.27 0.7076
-44.26 0.708
-44.26 0.7113
-44.26 0.7145
-44.25 0.7171
-44.25 0.7177
-44.25 0.7227
-44.25 0.7235
-44.25 0.7242
-44.25 0.7249
-44.25 0.7254
-44.24 0.7264
-44.24 0.7277
-44.24 0.7283
-44.24 0.7301
-44.23 0.7334
-44.23 0.7339
-44.23 0.7348
-44.23 0.736
-44.23 0.7374
-44.23 0.7388
-44.23 0.7398
-44.22 0.741
-44.22 0.7436
-44.22 0.744
-44.22 0.7455
-44.22 0.7474
-44.21 0.7481
-44.21 0.7495
-44.21 0.7521
-44.21 0.7547
-44.2 0.7578
-44.2 0.7582
-44.2 0.7593
-44.2 0.7598
-44.2 0.7603
-44.2 0.7612
-44.19 0.7642
-44.19 0.7648
-44.19 0.7656
-44.19 0.7665
-44.19 0.7671
-44.19 0.7679
-44.19 0.7691
-44.18 0.7712
-44.18 0.7716
-44.18 0.7734
-44.18 0.774
-44.18 0.7744
-44.17 0.7757
-44.17 0.7762
-44.17 0.7768
-44.17 0.7798
-44.17 0.7804
-44.16 0.7827
-44.16 0.7833
-44.16 0.7846
-44.16 0.7851
-44.15 0.7864
-44.15 0.7872
-44.15 0.7884
-44.15 0.789
-44.15 0.7899
-44.15 0.7905
-44.14 0.7925
-44.14 0.7943
-44.14 0.795
-44.14 0.7956
-44.13 0.7995
-44.13 0.8018
-44.13 0.8034
-44.13 0.8039
-44.12 0.8051
-44.12 0.8065
-44.12 0.807
-44.12 0.8084
-44.12 0.8115
-44.11 0.8132
-44.11 0.8145
-44.11 0.8152
-44.11 0.8161
-44.1 0.817
-44.1 0.8179
-44.1 0.8183
-44.1 0.8189
-44.1 0.8208
-44.09 0.8217
-44.09 0.8222
-44.09 0.823
-44.09 0.8235
-44.09 0.826
-44.08 0.8267
-44.07 0.8323
-44.07 0.8329
-44.07 0.8335
-44.07 0.8344
-44.07 0.8351
-44.07 0.8356
-44.07 0.8364
-44.06 0.8373
-44.06 0.838
-44.06 0.8388
-44.06 0.8401
-44.05 0.8416
-44.05 0.8436
-44.05 0.8444
-44.05 0.8451
-44.04 0.8462
-44.04 0.847
-44.04 0.8478
-44.04 0.8485
-44.04 0.8494
-44.04 0.8499
-44.04 0.8505
-44.03 0.8514
-44.03 0.852
-44.03 0.8527
-44.03 0.8533
-44.03 0.8538
-44.03 0.8548
-44.03 0.8564
-44.02 0.8596
-44.02 0.8608
-44.01 0.8623
-44.01 0.8637
-44.01 0.8644
-44.01 0.8651
-44.01 0.8655
-44.01 0.866
-44 0.8685
-44 0.8692
-44 0.8702
-44 0.8711
-44 0.8722
-43.99 0.8728
-43.99 0.8735
-43.98 0.8759
-43.97 0.8763
-43.97 0.8767
-43.96 0.8771
-43.96 0.8776
-43.94 0.8783
-43.94 0.8785
-43.92 0.8796
-43.92 0.8799
-43.91 0.8805
-43.86 0.8848
-43.85 0.8858
-43.84 0.8862
-43.79 0.8908
-43.79 0.8915
-43.79 0.8919
-43.78 0.8926
-43.78 0.8929
-43.77 0.8936
-43.76 0.895
-43.76 0.8954
-43.75 0.8959
-43.75 0.8964
-43.74 0.8967
-43.73 0.8975
-43.73 0.8979
-43.72 0.8983
-43.71 0.8988
-43.71 0.8998
-43.7 0.9004
-43.65 0.9047
-43.65 0.9051
-43.64 0.906
-43.64 0.907
-43.64 0.9074
-43.63 0.9081
-43.63 0.9087
-43.62 0.9102
-43.61 0.9109
-43.59 0.9123
-43.59 0.9128
-43.59 0.913
-43.58 0.9135
-43.57 0.915
-43.56 0.9153
-43.55 0.9163
-43.55 0.9166
-43.55 0.9171
-43.54 0.9187
-43.53 0.9194
-43.52 0.9205
-43.51 0.9213
-43.49 0.9224
-43.48 0.9237
-43.48 0.9242
-43.45 0.9263
-43.44 0.9267
-43.44 0.9269
-43.42 0.928
-43.38 0.9308
-43.38 0.9314
-43.37 0.9319
-43.37 0.9327
-43.36 0.9333
-43.36 0.9336
-43.34 0.9351
-43.32 0.937
-43.32 0.9375
-43.31 0.9384
-43.3 0.9389
-43.29 0.9396
-43.29 0.9401
-43.28 0.9408
-43.27 0.9415
-43.27 0.9421
-43.26 0.9427
-43.26 0.9431
-43.24 0.9438
-43.24 0.9443
-43.22 0.946
-43.22 0.9463
-43.21 0.947
-43.21 0.9479
-43.13 0.9519
-43.12 0.9527
-43.11 0.9535
-43.1 0.9537
-43.09 0.9545
-43.09 0.9549
-43.08 0.9556
-43.08 0.956
-43.04 0.958
-43.03 0.9584
-43.02 0.9589
-43.02 0.9594
-43.02 0.9598
-43 0.9609
-42.98 0.962
-42.96 0.963
-42.96 0.9633
-42.95 0.9639
-42.92 0.9653
-42.91 0.9659
-42.88 0.9682
-42.87 0.9687
-42.86 0.9693
-42.86 0.9696
-42.83 0.9702
-42.83 0.9707
-42.82 0.9711
-42.82 0.9715
-42.81 0.9721
-42.8 0.973
-42.8 0.9733
-42.79 0.9739
-42.77 0.9748
-42.75 0.9755
-42.72 0.9769
-42.71 0.9776
-42.69 0.9782
-42.68 0.9785
-42.67 0.9793
-42.66 0.9799
-42.65 0.9804
-42.65 0.9807
-42.63 0.9813
-42.61 0.9819
-42.6 0.9829
-42.6 0.9833
-42.59 0.9838
-42.59 0.9839
-42.57 0.9845
-42.56 0.9846
-42.53 0.9857
-42.52 0.9865
-42.5 0.9871
-42.45 0.9882
-42.44 0.9886
-42.41 0.9896
-42.37 0.9908
-42.34 0.9913
-42.34 0.9914
-42.31 0.9918
-42.19 0.9946
-42.19 0.9948
-42.14 0.9955
-42 0.9959
-41.99 0.996
-41.93 0.9965
-41.92 0.9966
-41.37 0.9999
};
%\addlegendentry{inside color2 - color2}
\addplot [thick, forget plot, color3, dashed]
table {%
-68.43 0
-68.38 8.571e-05
-67.17 0.0006855
-66.22 0.002656
-66.11 0.003256
-65.81 0.003941
-65.44 0.006255
-65.38 0.007026
-65.37 0.007112
-65.02 0.01028
-64.93 0.0108
-64.9 0.01105
-64.79 0.01242
-64.79 0.0126
-64.66 0.01345
-64.57 0.01499
-64.56 0.01525
-64.5 0.01585
-64.5 0.01594
-64.45 0.01688
-64.37 0.01808
-64.34 0.01911
-64.29 0.01979
-64.28 0.02005
-64.18 0.02176
-64.18 0.02202
-64.15 0.02279
-64.12 0.02348
-64.07 0.02399
-64.02 0.02545
-64 0.02579
-63.96 0.0269
-63.93 0.02768
-63.87 0.02836
-63.87 0.02862
-63.65 0.03384
-63.65 0.03402
-63.52 0.03736
-63.52 0.03761
-63.45 0.03984
-63.41 0.04036
-63.41 0.04053
-63.38 0.04147
-63.37 0.0419
-63.35 0.04276
-63.34 0.04318
-63.29 0.04396
-63.28 0.04473
-63.28 0.04524
-63.27 0.04558
-63.25 0.04627
-63.14 0.05124
-63.12 0.05175
-63.11 0.05184
-63.1 0.05235
-63.07 0.05312
-63.05 0.05381
-63.05 0.05389
-63.01 0.05484
-63 0.05509
-62.99 0.05569
-62.98 0.05655
-62.96 0.05749
-62.9 0.05998
-62.87 0.06092
-62.81 0.06332
-62.8 0.06452
-62.77 0.06606
-62.71 0.06992
-62.69 0.07052
-62.68 0.07154
-62.62 0.07454
-62.61 0.07506
-62.6 0.07583
-62.6 0.07626
-62.6 0.07651
-62.56 0.07908
-62.54 0.07968
-62.53 0.0808
-62.52 0.08114
-62.49 0.0826
-62.49 0.08285
-62.48 0.08405
-62.44 0.08534
-62.44 0.0856
-62.43 0.0862
-62.41 0.08842
-62.38 0.09057
-62.37 0.09117
-62.36 0.09202
-62.36 0.09219
-62.35 0.09279
-62.34 0.09382
-62.32 0.09468
-62.32 0.09511
-62.3 0.09622
-62.3 0.09674
-62.27 0.09896
-62.26 0.09956
-62.25 0.09982
-62.24 0.1003
-62.23 0.1013
-62.22 0.102
-62.21 0.1026
-62.2 0.1032
-62.17 0.1049
-62.16 0.1055
-62.13 0.1073
-62.12 0.1076
-62.11 0.1082
-62.1 0.1088
-62.1 0.1094
-62.08 0.1103
-62.07 0.111
-62.06 0.1116
-62.06 0.1118
-62.05 0.1124
-62.05 0.1125
-62.03 0.1132
-62.02 0.1141
-62.01 0.1147
-61.95 0.1189
-61.95 0.1193
-61.93 0.1201
-61.92 0.1214
-61.88 0.1236
-61.88 0.1242
-61.86 0.1259
-61.86 0.1261
-61.85 0.1266
-61.85 0.1268
-61.85 0.1273
-61.84 0.1275
-61.83 0.1286
-61.83 0.129
-61.78 0.1329
-61.78 0.1338
-61.77 0.1347
-61.75 0.1358
-61.73 0.1379
-61.73 0.1383
-61.71 0.139
-61.71 0.1395
-61.7 0.1399
-61.7 0.1402
-61.69 0.1407
-61.68 0.1415
-61.67 0.1421
-61.67 0.1427
-61.65 0.1433
-61.55 0.1511
-61.53 0.1535
-61.53 0.1539
-61.51 0.156
-61.51 0.1562
-61.5 0.1572
-61.49 0.1583
-61.45 0.1615
-61.37 0.1683
-61.32 0.1728
-61.32 0.1733
-61.31 0.174
-61.3 0.1756
-61.29 0.1762
-61.28 0.1768
-61.28 0.1774
-61.26 0.1787
-61.25 0.1797
-61.25 0.1804
-61.22 0.1829
-61.2 0.1862
-61.17 0.1888
-61.16 0.1893
-61.15 0.1898
-61.14 0.1906
-61.13 0.1926
-61.11 0.1936
-61.1 0.1959
-61.08 0.197
-61.08 0.1973
-61.06 0.1986
-61.06 0.199
-61.05 0.1996
-60.99 0.2061
-60.97 0.2067
-60.97 0.2075
-60.94 0.2104
-60.93 0.2111
-60.93 0.2117
-60.92 0.2124
-60.92 0.2129
-60.91 0.2138
-60.9 0.2146
-60.9 0.2151
-60.9 0.2157
-60.9 0.2158
-60.89 0.2169
-60.87 0.2182
-60.87 0.2187
-60.86 0.2198
-60.85 0.2205
-60.84 0.2226
-60.83 0.2231
-60.83 0.2238
-60.82 0.2247
-60.82 0.2253
-60.81 0.226
-60.8 0.2271
-60.79 0.2279
-60.78 0.2291
-60.78 0.2294
-60.77 0.23
-60.77 0.2302
-60.76 0.231
-60.76 0.2319
-60.75 0.2325
-60.74 0.2335
-60.74 0.2341
-60.73 0.2351
-60.72 0.2356
-60.72 0.2367
-60.71 0.2371
-60.69 0.2391
-60.69 0.2397
-60.69 0.2403
-60.68 0.2413
-60.67 0.2418
-60.67 0.242
-60.66 0.2427
-60.64 0.2465
-60.63 0.2475
-60.62 0.2492
-60.6 0.251
-60.6 0.2514
-60.59 0.2522
-60.59 0.2525
-60.59 0.253
-60.58 0.254
-60.58 0.255
-60.57 0.2556
-60.57 0.2564
-60.56 0.2576
-60.55 0.2581
-60.55 0.2583
-60.54 0.26
-60.54 0.2602
-60.53 0.2616
-60.52 0.2618
-60.52 0.2624
-60.52 0.2628
-60.51 0.2634
-60.51 0.2636
-60.51 0.2642
-60.5 0.2649
-60.5 0.2656
-60.49 0.2666
-60.47 0.2682
-60.47 0.2687
-60.44 0.2724
-60.43 0.273
-60.43 0.2738
-60.43 0.2744
-60.43 0.275
-60.42 0.2758
-60.41 0.2763
-60.4 0.2778
-60.4 0.2786
-60.39 0.2793
-60.39 0.28
-60.38 0.2806
-60.37 0.2825
-60.37 0.283
-60.34 0.2852
-60.33 0.2864
-60.33 0.2876
-60.32 0.2887
-60.32 0.2893
-60.28 0.2932
-60.27 0.2948
-60.25 0.2982
-60.24 0.2988
-60.24 0.2989
-60.24 0.2994
-60.23 0.2998
-60.23 0.3001
-60.23 0.3007
-60.23 0.3009
-60.21 0.3015
-60.21 0.3018
-60.2 0.3028
-60.2 0.3038
-60.2 0.3043
-60.19 0.3055
-60.18 0.3062
-60.18 0.3068
-60.18 0.3074
-60.17 0.308
-60.17 0.3088
-60.16 0.3096
-60.14 0.3119
-60.14 0.3126
-60.13 0.3139
-60.13 0.3145
-60.12 0.3152
-60.12 0.3168
-60.11 0.3173
-60.09 0.319
-60.09 0.3195
-60.09 0.3198
-60.08 0.3211
-60.08 0.3213
-60.07 0.3218
-60.07 0.3223
-60.06 0.323
-60.03 0.327
-60.03 0.3277
-60.01 0.3294
-60 0.3315
-60 0.3319
-59.99 0.3326
-59.99 0.3328
-59.98 0.3333
-59.97 0.3357
-59.97 0.3363
-59.97 0.3368
-59.96 0.3375
-59.96 0.339
-59.94 0.3403
-59.94 0.3411
-59.94 0.3417
-59.93 0.3422
-59.93 0.3429
-59.93 0.3435
-59.92 0.3439
-59.92 0.3445
-59.91 0.3457
-59.91 0.3461
-59.91 0.3465
-59.91 0.3473
-59.9 0.3479
-59.9 0.348
-59.89 0.3486
-59.88 0.3509
-59.88 0.3512
-59.88 0.3519
-59.87 0.353
-59.87 0.3532
-59.86 0.3537
-59.83 0.3586
-59.83 0.3588
-59.83 0.3594
-59.82 0.3606
-59.82 0.3609
-59.81 0.3614
-59.81 0.3621
-59.81 0.3624
-59.8 0.363
-59.79 0.366
-59.78 0.3665
-59.76 0.3708
-59.76 0.3711
-59.75 0.3719
-59.75 0.373
-59.74 0.3735
-59.74 0.3741
-59.73 0.3746
-59.73 0.3749
-59.73 0.3756
-59.73 0.3765
-59.72 0.377
-59.71 0.3784
-59.71 0.379
-59.69 0.3821
-59.68 0.3828
-59.68 0.3839
-59.67 0.3845
-59.66 0.3854
-59.65 0.3878
-59.64 0.3887
-59.64 0.3897
-59.62 0.3935
-59.61 0.3943
-59.61 0.395
-59.59 0.398
-59.59 0.3984
-59.57 0.4006
-59.57 0.4008
-59.57 0.4013
-59.56 0.4028
-59.55 0.4036
-59.55 0.4042
-59.55 0.4048
-59.54 0.405
-59.53 0.4062
-59.53 0.4068
-59.52 0.4076
-59.52 0.4085
-59.51 0.4096
-59.5 0.4103
-59.48 0.4138
-59.47 0.4147
-59.47 0.4152
-59.46 0.4171
-59.45 0.4177
-59.44 0.4192
-59.44 0.4204
-59.43 0.4208
-59.43 0.4217
-59.43 0.4222
-59.42 0.4228
-59.42 0.4238
-59.4 0.4271
-59.4 0.4275
-59.39 0.4284
-59.39 0.4288
-59.37 0.4305
-59.37 0.4312
-59.37 0.4317
-59.36 0.4324
-59.36 0.4332
-59.35 0.4338
-59.35 0.4349
-59.34 0.4357
-59.32 0.4388
-59.32 0.4392
-59.32 0.4396
-59.31 0.4401
-59.31 0.4413
-59.28 0.4456
-59.27 0.4468
-59.27 0.4473
-59.27 0.4482
-59.27 0.4486
-59.26 0.4492
-59.26 0.4501
-59.26 0.4505
-59.24 0.453
-59.23 0.454
-59.23 0.4548
-59.23 0.4554
-59.22 0.4566
-59.22 0.4571
-59.22 0.4577
-59.2 0.46
-59.2 0.4609
-59.2 0.4613
-59.19 0.4622
-59.16 0.4677
-59.16 0.4681
-59.15 0.4685
-59.15 0.4691
-59.15 0.4695
-59.14 0.4702
-59.14 0.4706
-59.13 0.4713
-59.13 0.472
-59.13 0.4728
-59.13 0.473
-59.12 0.4734
-59.11 0.4752
-59.1 0.4775
-59.09 0.478
-59.09 0.4785
-59.08 0.4798
-59.08 0.4803
-59.07 0.4809
-59.07 0.4819
-59.07 0.4826
-59.06 0.4837
-59.03 0.4879
-59.03 0.4882
-59.03 0.4888
-59.01 0.4917
-59 0.4923
-58.99 0.4933
-58.99 0.4934
-58.99 0.494
-58.98 0.4948
-58.98 0.4954
-58.97 0.4963
-58.96 0.4987
-58.96 0.4994
-58.95 0.5002
-58.94 0.5008
-58.94 0.5017
-58.93 0.5023
-58.93 0.503
-58.93 0.5035
-58.92 0.5046
-58.92 0.505
-58.91 0.5064
-58.91 0.5067
-58.9 0.5089
-58.89 0.5102
-58.88 0.5114
-58.87 0.5122
-58.87 0.5128
-58.86 0.5146
-58.86 0.5149
-58.85 0.5154
-58.85 0.5161
-58.85 0.5162
-58.84 0.5168
-58.84 0.5174
-58.84 0.5179
-58.83 0.5184
-58.81 0.5208
-58.8 0.5216
-58.8 0.5224
-58.79 0.5229
-58.79 0.5242
-58.78 0.5259
-58.77 0.5267
-58.77 0.5274
-58.76 0.5293
-58.75 0.5314
-58.74 0.5321
-58.73 0.5341
-58.73 0.5347
-58.72 0.5355
-58.71 0.5369
-58.71 0.5382
-58.71 0.539
-58.7 0.5399
-58.7 0.5405
-58.69 0.541
-58.69 0.5417
-58.69 0.542
-58.68 0.5426
-58.64 0.5494
-58.64 0.5497
-58.62 0.5522
-58.62 0.5527
-58.61 0.5533
-58.59 0.5557
-58.58 0.559
-58.57 0.5597
-58.57 0.5604
-58.57 0.5614
-58.56 0.5618
-58.55 0.5634
-58.55 0.5637
-58.55 0.5643
-58.55 0.5645
-58.55 0.5649
-58.54 0.5654
-58.54 0.566
-58.52 0.5693
-58.51 0.5698
-58.51 0.5704
-58.51 0.5707
-58.5 0.5713
-58.46 0.5793
-58.45 0.5798
-58.45 0.5802
-58.45 0.5812
-58.44 0.5818
-58.43 0.5835
-58.43 0.5838
-58.42 0.5844
-58.4 0.5869
-58.4 0.588
-58.39 0.5885
-58.38 0.5904
-58.38 0.5904
-58.38 0.5909
-58.37 0.5912
-58.37 0.5922
-58.37 0.5927
-58.36 0.5933
-58.34 0.5974
-58.34 0.5979
-58.33 0.5994
-58.33 0.5998
-58.33 0.6003
-58.33 0.6007
-58.32 0.6013
-58.32 0.6017
-58.32 0.6023
-58.3 0.6057
-58.28 0.6085
-58.28 0.6091
-58.28 0.6099
-58.27 0.6107
-58.26 0.6133
-58.25 0.6138
-58.24 0.616
-58.23 0.6164
-58.23 0.6169
-58.23 0.6175
-58.22 0.6181
-58.22 0.6186
-58.21 0.6195
-58.21 0.6201
-58.21 0.6206
-58.19 0.6227
-58.19 0.6233
-58.19 0.6234
-58.18 0.6239
-58.17 0.6271
-58.16 0.6277
-58.16 0.6286
-58.16 0.629
-58.15 0.6297
-58.14 0.6316
-58.11 0.6368
-58.11 0.6373
-58.1 0.6381
-58.09 0.6393
-58.08 0.6406
-58.08 0.6418
-58.07 0.643
-58.06 0.6449
-58.06 0.6455
-58.05 0.6463
-58.04 0.6471
-58.04 0.6476
-58.03 0.6482
-58.03 0.6488
-58.03 0.6491
-58.03 0.6496
-58.02 0.6503
-58.02 0.6509
-58.02 0.6514
-58 0.6539
-57.99 0.6551
-57.98 0.6558
-57.96 0.6599
-57.96 0.6603
-57.96 0.6609
-57.95 0.6617
-57.95 0.6622
-57.94 0.6636
-57.93 0.6646
-57.93 0.6655
-57.92 0.6662
-57.91 0.6685
-57.9 0.6694
-57.9 0.6699
-57.89 0.6717
-57.89 0.6719
-57.89 0.6724
-57.88 0.673
-57.88 0.6744
-57.87 0.6746
-57.87 0.6751
-57.87 0.6756
-57.86 0.6765
-57.84 0.6799
-57.84 0.6803
-57.83 0.6812
-57.83 0.6817
-57.82 0.6841
-57.82 0.6843
-57.81 0.6849
-57.81 0.685
-57.8 0.6855
-57.79 0.6879
-57.79 0.6886
-57.78 0.6891
-57.78 0.6896
-57.77 0.6903
-57.74 0.6945
-57.73 0.6951
-57.73 0.6957
-57.73 0.6966
-57.72 0.6978
-57.71 0.6986
-57.71 0.6996
-57.7 0.7014
-57.69 0.7026
-57.68 0.7031
-57.68 0.7037
-57.68 0.7043
-57.66 0.705
-57.66 0.7058
-57.66 0.7061
-57.65 0.7067
-57.64 0.7087
-57.63 0.7095
-57.62 0.7106
-57.62 0.7113
-57.61 0.7122
-57.6 0.7139
-57.6 0.7147
-57.59 0.7152
-57.58 0.7165
-57.58 0.7171
-57.57 0.7184
-57.56 0.7191
-57.56 0.7202
-57.55 0.7206
-57.55 0.7212
-57.54 0.7229
-57.54 0.7238
-57.53 0.7243
-57.53 0.7252
-57.51 0.7278
-57.51 0.728
-57.51 0.7288
-57.5 0.73
-57.5 0.7305
-57.49 0.7316
-57.46 0.7359
-57.45 0.7372
-57.45 0.7379
-57.44 0.7385
-57.44 0.7392
-57.44 0.7396
-57.43 0.7407
-57.4 0.745
-57.39 0.7456
-57.38 0.7468
-57.38 0.7474
-57.38 0.7479
-57.37 0.749
-57.33 0.7534
-57.32 0.7542
-57.32 0.7544
-57.31 0.7552
-57.31 0.7561
-57.3 0.7565
-57.3 0.7568
-57.29 0.7576
-57.29 0.7581
-57.28 0.7593
-57.27 0.7602
-57.27 0.7605
-57.26 0.7613
-57.25 0.7631
-57.25 0.7639
-57.23 0.7661
-57.23 0.767
-57.22 0.7675
-57.22 0.7682
-57.22 0.7687
-57.21 0.7694
-57.21 0.7698
-57.19 0.7711
-57.18 0.7729
-57.17 0.7739
-57.16 0.7745
-57.16 0.7747
-57.16 0.7753
-57.15 0.7764
-57.14 0.7771
-57.13 0.7783
-57.13 0.7788
-57.12 0.7799
-57.11 0.7805
-57.1 0.7811
-57.1 0.7818
-57.09 0.7824
-57.08 0.784
-57.07 0.7847
-57.07 0.7851
-57.07 0.7857
-57.06 0.7861
-57.06 0.7867
-57.05 0.7878
-57.05 0.7885
-57.04 0.7893
-57.02 0.7908
-57.01 0.7914
-57 0.7929
-57 0.7933
-57 0.7937
-56.99 0.7947
-56.98 0.7958
-56.97 0.7965
-56.97 0.7969
-56.96 0.7976
-56.96 0.798
-56.95 0.7986
-56.93 0.8007
-56.92 0.8013
-56.92 0.8018
-56.92 0.8021
-56.9 0.8037
-56.89 0.8045
-56.88 0.805
-56.87 0.8059
-56.86 0.8077
-56.85 0.8082
-56.84 0.809
-56.84 0.8094
-56.84 0.81
-56.83 0.8115
-56.82 0.812
-56.81 0.8139
-56.8 0.8147
-56.79 0.8153
-56.79 0.8159
-56.78 0.8164
-56.78 0.8166
-56.76 0.8185
-56.76 0.8189
-56.75 0.8198
-56.74 0.8208
-56.74 0.8216
-56.73 0.823
-56.72 0.8239
-56.71 0.8259
-56.7 0.8264
-56.68 0.8284
-56.68 0.8287
-56.67 0.8298
-56.65 0.8322
-56.65 0.833
-56.64 0.8336
-56.63 0.8347
-56.61 0.8358
-56.61 0.8365
-56.59 0.8381
-56.58 0.8393
-56.57 0.8398
-56.57 0.8405
-56.55 0.8413
-56.55 0.8415
-56.54 0.842
-56.54 0.8422
-56.53 0.843
-56.52 0.8444
-56.5 0.8457
-56.5 0.8459
-56.49 0.8471
-56.48 0.8492
-56.47 0.8497
-56.46 0.8505
-56.43 0.8544
-56.42 0.855
-56.41 0.8567
-56.4 0.8573
-56.39 0.8592
-56.37 0.8601
-56.37 0.8604
-56.36 0.861
-56.36 0.8616
-56.35 0.8621
-56.35 0.8628
-56.34 0.8639
-56.32 0.8645
-56.31 0.8656
-56.3 0.8667
-56.29 0.8673
-56.29 0.8675
-56.27 0.8692
-56.26 0.8701
-56.25 0.8708
-56.24 0.8712
-56.22 0.8726
-56.22 0.873
-56.22 0.8736
-56.21 0.8739
-56.21 0.8742
-56.2 0.8748
-56.16 0.8796
-56.15 0.8802
-56.14 0.8817
-56.14 0.8819
-56.12 0.8828
-56.12 0.8833
-56.11 0.8839
-56.11 0.8842
-56.09 0.8853
-56.06 0.8874
-56.03 0.8903
-56.02 0.8909
-56.01 0.8924
-55.98 0.8951
-55.97 0.8956
-55.97 0.896
-55.96 0.8967
-55.96 0.8972
-55.95 0.898
-55.95 0.8982
-55.92 0.8997
-55.9 0.9009
-55.9 0.9012
-55.89 0.9019
-55.88 0.9025
-55.87 0.903
-55.87 0.9032
-55.86 0.9037
-55.86 0.9041
-55.83 0.9055
-55.83 0.9059
-55.82 0.9067
-55.76 0.9122
-55.75 0.9128
-55.74 0.9141
-55.72 0.9148
-55.72 0.9152
-55.69 0.9165
-55.69 0.9169
-55.68 0.9176
-55.65 0.9191
-55.63 0.9213
-55.63 0.9215
-55.57 0.9239
-55.56 0.9245
-55.55 0.9253
-55.55 0.9257
-55.54 0.9262
-55.54 0.9265
-55.53 0.927
-55.5 0.9285
-55.49 0.929
-55.49 0.9291
-55.47 0.9299
-55.42 0.9329
-55.41 0.9333
-55.39 0.9343
-55.38 0.9348
-55.37 0.9352
-55.36 0.9358
-55.35 0.9363
-55.35 0.9368
-55.35 0.9369
-55.34 0.9375
-55.33 0.9383
-55.32 0.939
-55.3 0.9394
-55.29 0.9395
-55.28 0.9402
-55.27 0.9405
-55.26 0.9411
-55.26 0.9414
-55.26 0.942
-55.25 0.9425
-55.25 0.9426
-55.23 0.9431
-55.23 0.9434
-55.22 0.9439
-55.22 0.9441
-55.2 0.945
-55.2 0.9456
-55.09 0.9503
-55.08 0.9515
-55.07 0.9518
-55.06 0.9525
-54.99 0.9561
-54.98 0.9566
-54.97 0.9572
-54.97 0.9575
-54.94 0.9584
-54.94 0.9589
-54.93 0.9594
-54.93 0.9596
-54.91 0.9602
-54.91 0.9605
-54.88 0.9612
-54.87 0.9614
-54.81 0.9632
-54.78 0.9641
-54.74 0.9647
-54.74 0.965
-54.72 0.9655
-54.72 0.9656
-54.68 0.9663
-54.67 0.9668
-54.65 0.9677
-54.65 0.9678
-54.63 0.9683
-54.63 0.9686
-54.61 0.9691
-54.61 0.9696
-54.56 0.9708
-54.56 0.971
-54.54 0.9715
-54.54 0.9716
-54.49 0.9728
-54.48 0.9736
-54.41 0.9752
-54.41 0.9752
-54.38 0.9758
-54.38 0.9762
-54.36 0.9767
-54.33 0.9772
-54.12 0.9813
-54.11 0.9818
-54.07 0.9825
-54.05 0.9829
-54.02 0.9836
-53.98 0.9845
-53.93 0.9856
-53.93 0.9858
-53.9 0.9863
-53.69 0.9894
-53.69 0.9895
-53.69 0.9895
-53.62 0.9901
-53.58 0.9907
-53.58 0.991
-53.18 0.9948
-53.04 0.9954
-53.03 0.9955
-52.87 0.996
-52.85 0.9961
-52.69 0.9969
-52.67 0.9972
-52.33 0.9988
-52.01 0.9994
-51.98 0.9996
-51.73 0.9998
-51.56 0.9999
};
%\addlegendentry{outside color2 - color}
\addplot [thick, forget plot, color3]
table {%
-68.94 0
-65.73 0.003788
-65.31 0.007576
-64.52 0.01136
-64.46 0.01515
-63.32 0.01894
-63.13 0.02273
-63.06 0.02652
-63.05 0.0303
-62.81 0.03409
-62.78 0.03788
-62.76 0.04167
-62.73 0.04545
-62.51 0.04924
-62.51 0.05303
-62.43 0.05682
-62.38 0.06061
-62.37 0.06439
-62.37 0.06818
-62.23 0.07197
-62.2 0.07576
-62.18 0.07955
-62.1 0.08333
-62.09 0.08712
-61.91 0.09091
-61.77 0.0947
-61.76 0.09848
-61.76 0.1023
-61.76 0.1061
-61.75 0.1098
-61.7 0.1136
-61.56 0.1174
-61.4 0.125
-61.4 0.1288
-61.38 0.1326
-61.34 0.1364
-61.34 0.1402
-61.34 0.1439
-61.22 0.1477
-61.21 0.1515
-61.07 0.1553
-61.06 0.1591
-61.06 0.1629
-61.05 0.1667
-60.97 0.1705
-60.95 0.1742
-60.84 0.1818
-60.76 0.1856
-60.67 0.1932
-60.66 0.197
-60.62 0.2008
-60.6 0.2045
-60.57 0.2083
-60.5 0.2121
-60.44 0.2159
-60.43 0.2197
-60.36 0.2273
-60.36 0.2311
-60.27 0.2348
-60.21 0.2386
-60.14 0.2424
-60.14 0.2462
-60.1 0.25
-60.09 0.2538
-60.06 0.2576
-60.04 0.2614
-60 0.2652
-59.96 0.2689
-59.95 0.2727
-59.92 0.2765
-59.9 0.2803
-59.88 0.2841
-59.87 0.2879
-59.83 0.2917
-59.82 0.2955
-59.77 0.2992
-59.74 0.303
-59.71 0.3068
-59.65 0.3106
-59.65 0.3144
-59.6 0.3182
-59.57 0.322
-59.54 0.3258
-59.54 0.3295
-59.5 0.3333
-59.45 0.3371
-59.42 0.3447
-59.33 0.3485
-59.29 0.3523
-59.23 0.3598
-59.2 0.3636
-59.16 0.3674
-59.16 0.375
-59.15 0.3788
-59.13 0.3826
-59.11 0.3864
-59.03 0.3902
-59.01 0.3939
-59 0.3977
-58.93 0.4053
-58.93 0.4091
-58.9 0.4129
-58.89 0.4167
-58.86 0.4242
-58.8 0.428
-58.78 0.4356
-58.78 0.4394
-58.74 0.4432
-58.74 0.447
-58.73 0.4508
-58.7 0.4545
-58.66 0.4583
-58.56 0.4621
-58.53 0.4659
-58.52 0.4697
-58.5 0.4735
-58.45 0.4773
-58.45 0.4811
-58.45 0.4848
-58.37 0.4886
-58.37 0.4924
-58.36 0.4962
-58.36 0.5
-58.3 0.5038
-58.26 0.5076
-58.25 0.5114
-58.13 0.5152
-58.12 0.5189
-58.11 0.5227
-58.09 0.5265
-58.07 0.5303
-58.03 0.5341
-57.99 0.5379
-57.99 0.5417
-57.98 0.5455
-57.98 0.5492
-57.96 0.553
-57.96 0.5568
-57.94 0.5606
-57.9 0.5644
-57.86 0.5682
-57.85 0.572
-57.84 0.5758
-57.82 0.5795
-57.81 0.5833
-57.76 0.5871
-57.73 0.5909
-57.73 0.5947
-57.71 0.5985
-57.71 0.6023
-57.69 0.6061
-57.69 0.6098
-57.69 0.6136
-57.68 0.6174
-57.67 0.6212
-57.64 0.625
-57.63 0.6288
-57.52 0.6326
-57.43 0.6402
-57.38 0.6439
-57.37 0.6477
-57.3 0.6515
-57.24 0.6591
-57.18 0.6629
-57.17 0.6667
-57.16 0.6705
-57.14 0.6742
-57.14 0.678
-57.13 0.6818
-57.12 0.6856
-57.12 0.6894
-57.11 0.6932
-57.07 0.697
-57.02 0.7008
-57.02 0.7045
-57 0.7121
-56.99 0.7159
-56.92 0.7197
-56.9 0.7235
-56.83 0.7273
-56.83 0.7311
-56.8 0.7348
-56.79 0.7386
-56.79 0.7424
-56.77 0.7462
-56.77 0.75
-56.77 0.7538
-56.75 0.7576
-56.72 0.7614
-56.51 0.7652
-56.51 0.7689
-56.49 0.7727
-56.47 0.7765
-56.37 0.7803
-56.36 0.7841
-56.35 0.7879
-56.32 0.7917
-56.3 0.7955
-56.29 0.7992
-56.24 0.803
-56.24 0.8068
-56.23 0.8106
-56.2 0.8144
-56.17 0.822
-56.17 0.8258
-56.14 0.8295
-56.13 0.8371
-56.11 0.8409
-56.11 0.8447
-56.1 0.8485
-56.07 0.8523
-56.06 0.8561
-55.99 0.8598
-55.98 0.8636
-55.96 0.8674
-55.89 0.8712
-55.78 0.875
-55.75 0.8788
-55.74 0.8826
-55.68 0.8864
-55.56 0.8902
-55.5 0.8939
-55.48 0.8977
-55.34 0.9015
-55.32 0.9053
-55.29 0.9091
-55.28 0.9129
-55.17 0.9167
-55.15 0.9205
-55.06 0.9242
-55 0.928
-54.98 0.9318
-54.94 0.9356
-54.93 0.9394
-54.84 0.9432
-54.38 0.947
-54.35 0.9508
-54.34 0.9545
-54.13 0.9583
-54.1 0.9621
-53.99 0.9659
-53.8 0.9735
-53.62 0.9773
-53.51 0.9811
-53.5 0.9848
-53.01 0.9886
-52.61 0.9924
-51.94 0.9962

};
%\addlegendentry{inside color2 - color}
\addplot [thick, forget plot, color4, dashed]
table {%

-74.01 0
-74.01 0.0002328
-74 0.0004656
-72.88 0.00163
-72.87 0.001863
-72.83 0.002095
-72.71 0.002561
-72.6 0.003492
-72.27 0.004424
-72.26 0.004657
-71.85 0.006985
-71.83 0.007451
-71.51 0.00908
-71.43 0.01001
-71.34 0.01071
-71.34 0.01094
-71.3 0.01187
-71.3 0.01211
-71.23 0.01281
-71.21 0.0135
-71.07 0.0142
-70.96 0.0149
-70.95 0.01537
-70.81 0.01607
-70.77 0.01723
-70.68 0.01793
-70.68 0.01816
-70.58 0.01909
-70.58 0.01932
-70.46 0.02095
-70.39 0.02212
-70.35 0.02258
-70.35 0.02282
-70.34 0.02305
-70.29 0.02375
-70.28 0.02445
-70.2 0.02747
-70.11 0.02934
-70.03 0.03073
-70.01 0.03143
-70.01 0.03166
-69.96 0.03283
-69.9 0.03329
-69.9 0.03353
-69.86 0.03446
-69.85 0.03492
-69.82 0.03586
-69.8 0.03632
-69.74 0.03702
-69.73 0.03772
-69.72 0.03795
-69.65 0.03888
-69.64 0.03958
-69.59 0.04051
-69.58 0.04098
-69.57 0.04144
-69.54 0.04191
-69.53 0.04261
-69.45 0.04563
-69.45 0.04587
-69.43 0.04633
-69.43 0.04703
-69.42 0.04773
-69.42 0.04796
-69.41 0.04843
-69.38 0.04889
-69.36 0.04983
-69.34 0.05029
-69.32 0.05099
-69.32 0.05169
-69.32 0.05192
-69.24 0.05308
-69.23 0.05355
-69.23 0.05402
-69.2 0.05541
-69.2 0.05565
-69.19 0.05634
-69.19 0.05658
-69.14 0.05751
-69.03 0.06123
-69.01 0.0631
-68.99 0.0638
-68.97 0.06589
-68.97 0.06612
-68.93 0.06775
-68.91 0.06892
-68.88 0.07008
-68.88 0.07078
-68.87 0.07148
-68.86 0.07218
-68.84 0.07288
-68.84 0.07311
-68.81 0.07381
-68.81 0.07404
-68.79 0.07497
-68.79 0.0759
-68.75 0.0766
-68.75 0.07683
-68.74 0.0773
-68.72 0.078
-68.71 0.0787
-68.7 0.07939
-68.7 0.07986
-68.69 0.08102
-68.69 0.08196
-68.67 0.08359
-68.67 0.08382
-68.67 0.08405
-68.65 0.08475
-68.65 0.08591
-68.64 0.08638
-68.63 0.08708
-68.62 0.08731
-68.61 0.08801
-68.6 0.08941
-68.6 0.08987
-68.58 0.09104
-68.57 0.0922
-68.57 0.09313
-68.56 0.09383
-68.56 0.09406
-68.55 0.09476
-68.55 0.09546
-68.55 0.09616
-68.54 0.09802
-68.54 0.09849
-68.53 0.09965
-68.53 0.1001
-68.53 0.1006
-68.53 0.101
-68.53 0.1015
-68.52 0.1024
-68.52 0.1031
-68.52 0.1034
-68.52 0.1038
-68.52 0.1048
-68.52 0.1052
-68.52 0.1057
-68.51 0.1071
-68.51 0.1097
-68.5 0.1115
-68.5 0.1122
-68.5 0.1139
-68.49 0.1148
-68.49 0.1166
-68.48 0.1178
-68.48 0.118
-68.47 0.1183
-68.46 0.119
-68.45 0.1204
-68.45 0.1206
-68.44 0.1213
-68.43 0.1222
-68.42 0.1229
-68.42 0.1239
-68.38 0.1255
-68.36 0.1264
-68.36 0.1267
-68.35 0.1271
-68.33 0.1278
-68.32 0.1288
-68.32 0.129
-68.31 0.1295
-68.31 0.1299
-68.31 0.1311
-68.29 0.1329
-68.29 0.1334
-68.28 0.1341
-68.27 0.1353
-68.27 0.1357
-68.27 0.1362
-68.23 0.1383
-68.23 0.1397
-68.22 0.1404
-68.21 0.1413
-68.2 0.142
-68.19 0.1427
-68.19 0.143
-68.17 0.1448
-68.16 0.1455
-68.15 0.146
-68.15 0.1464
-68.15 0.1471
-68.14 0.1476
-68.14 0.1483
-68.14 0.1488
-68.13 0.1495
-68.13 0.1497
-68.12 0.1506
-68.11 0.1509
-68.1 0.1523
-68.1 0.1527
-68.1 0.153
-68.09 0.1539
-68.08 0.1544
-68.07 0.1551
-68.07 0.1555
-68.05 0.1569
-68.05 0.1574
-68.04 0.1581
-68.04 0.159
-68.03 0.1597
-68.03 0.1602
-68.01 0.1614
-68.01 0.162
-68 0.1625
-67.99 0.1632
-67.98 0.1651
-67.98 0.1655
-67.97 0.1662
-67.97 0.1672
-67.96 0.1679
-67.95 0.1683
-67.95 0.1686
-67.93 0.169
-67.92 0.1704
-67.92 0.1709
-67.91 0.1718
-67.91 0.1725
-67.9 0.173
-67.9 0.1732
-67.88 0.1751
-67.87 0.176
-67.86 0.1765
-67.86 0.1774
-67.86 0.1776
-67.85 0.1783
-67.84 0.1793
-67.84 0.1795
-67.83 0.1802
-67.83 0.1804
-67.82 0.1809
-67.81 0.1818
-67.81 0.1825
-67.79 0.1837
-67.78 0.1853
-67.78 0.186
-67.78 0.1863
-67.74 0.1902
-67.73 0.1912
-67.72 0.1923
-67.72 0.193
-67.72 0.1935
-67.72 0.1939
-67.71 0.1942
-67.71 0.1949
-67.7 0.1956
-67.7 0.1965
-67.69 0.1972
-67.69 0.1974
-67.68 0.1988
-67.66 0.1995
-67.65 0.2014
-67.64 0.2019
-67.63 0.2026
-67.63 0.2035
-67.61 0.2054
-67.58 0.2084
-67.58 0.2088
-67.57 0.2098
-67.56 0.2105
-67.56 0.2109
-67.56 0.2116
-67.55 0.2121
-67.55 0.213
-67.55 0.2135
-67.55 0.214
-67.54 0.2147
-67.54 0.2151
-67.5 0.2207
-67.5 0.2212
-67.49 0.2219
-67.49 0.2226
-67.48 0.2235
-67.48 0.224
-67.47 0.2249
-67.46 0.2272
-67.45 0.2277
-67.43 0.231
-67.43 0.2314
-67.42 0.2331
-67.41 0.2338
-67.41 0.234
-67.4 0.2347
-67.4 0.2354
-67.4 0.2359
-67.39 0.2375
-67.38 0.238
-67.37 0.2405
-67.36 0.2412
-67.35 0.2421
-67.35 0.2428
-67.34 0.2435
-67.34 0.2449
-67.34 0.2456
-67.34 0.2463
-67.33 0.247
-67.33 0.248
-67.33 0.2482
-67.33 0.2487
-67.33 0.2494
-67.33 0.2496
-67.32 0.2503
-67.31 0.2512
-67.31 0.2522
-67.29 0.254
-67.28 0.2554
-67.28 0.2561
-67.27 0.2582
-67.25 0.2589
-67.25 0.2598
-67.25 0.2603
-67.23 0.2622
-67.22 0.264
-67.21 0.2654
-67.21 0.2661
-67.21 0.2664
-67.2 0.2673
-67.19 0.2682
-67.19 0.2685
-67.19 0.2689
-67.18 0.2694
-67.18 0.2696
-67.17 0.2701
-67.17 0.2703
-67.16 0.271
-67.15 0.2722
-67.14 0.2743
-67.13 0.275
-67.13 0.2761
-67.12 0.2771
-67.12 0.278
-67.12 0.2785
-67.12 0.2787
-67.11 0.2799
-67.11 0.2806
-67.09 0.2817
-67.09 0.2824
-67.09 0.2831
-67.08 0.2841
-67.07 0.2857
-67.07 0.2866
-67.06 0.2882
-67.06 0.2887
-67.06 0.2892
-67.05 0.2896
-67.05 0.2901
-67.05 0.2908
-67.04 0.2913
-67.04 0.2922
-67.03 0.2934
-67.03 0.2945
-67.01 0.2985
-67 0.2992
-67 0.3003
-67 0.301
-66.99 0.3027
-66.98 0.3034
-66.98 0.3038
-66.98 0.3043
-66.97 0.3055
-66.97 0.3066
-66.97 0.3071
-66.96 0.3085
-66.95 0.3094
-66.94 0.3101
-66.94 0.3104
-66.93 0.3111
-66.93 0.312
-66.93 0.3129
-66.92 0.3143
-66.91 0.3155
-66.91 0.3159
-66.91 0.3164
-66.91 0.3173
-66.89 0.3194
-66.89 0.3211
-66.88 0.3218
-66.88 0.3227
-66.88 0.3234
-66.86 0.3253
-66.86 0.3255
-66.85 0.3264
-66.85 0.3269
-66.84 0.3283
-66.82 0.3295
-66.82 0.3302
-66.82 0.3306
-66.82 0.3313
-66.81 0.3318
-66.8 0.3336
-66.8 0.3341
-66.79 0.3353
-66.79 0.3362
-66.78 0.3367
-66.78 0.3371
-66.77 0.3392
-66.76 0.3395
-66.74 0.3444
-66.74 0.3446
-66.73 0.3455
-66.73 0.346
-66.72 0.3469
-66.72 0.3471
-66.71 0.3481
-66.7 0.3485
-66.7 0.3492
-66.69 0.3497
-66.69 0.3504
-66.68 0.3511
-66.68 0.352
-66.67 0.3525
-66.67 0.3527
-66.67 0.3532
-66.66 0.3534
-66.66 0.3548
-66.65 0.3551
-66.65 0.3558
-66.65 0.3565
-66.63 0.3593
-66.63 0.36
-66.62 0.3607
-66.61 0.3627
-66.61 0.3639
-66.61 0.3646
-66.59 0.3662
-66.59 0.3667
-66.59 0.3672
-66.58 0.3676
-66.58 0.3679
-66.57 0.369
-66.57 0.37
-66.57 0.3707
-66.54 0.373
-66.54 0.3737
-66.53 0.3746
-66.53 0.3756
-66.53 0.3767
-66.52 0.3776
-66.52 0.379
-66.5 0.3823
-66.49 0.3835
-66.49 0.3839
-66.49 0.3844
-66.48 0.3877
-66.47 0.3881
-66.47 0.3886
-66.47 0.3893
-66.46 0.3898
-66.46 0.3902
-66.46 0.3912
-66.46 0.3914
-66.43 0.3942
-66.43 0.3949
-66.42 0.3956
-66.42 0.3958
-66.41 0.3967
-66.41 0.3972
-66.4 0.3986
-66.4 0.3993
-66.39 0.4
-66.39 0.4005
-66.39 0.4009
-66.38 0.4014
-66.38 0.4023
-66.37 0.4033
-66.37 0.4042
-66.37 0.4044
-66.36 0.4049
-66.36 0.4051
-66.35 0.4061
-66.35 0.4065
-66.34 0.4072
-66.34 0.4077
-66.34 0.4081
-66.33 0.4086
-66.33 0.4091
-66.33 0.4098
-66.32 0.4109
-66.32 0.4114
-66.31 0.4119
-66.31 0.4123
-66.31 0.4128
-66.3 0.4135
-66.3 0.4144
-66.3 0.4149
-66.29 0.4154
-66.29 0.4163
-66.28 0.4172
-66.28 0.4179
-66.28 0.4186
-66.27 0.4198
-66.27 0.4203
-66.27 0.421
-66.26 0.4219
-66.26 0.4228
-66.26 0.4237
-66.25 0.4244
-66.25 0.4249
-66.25 0.4258
-66.25 0.4272
-66.24 0.4279
-66.24 0.4282
-66.23 0.4291
-66.23 0.4296
-66.21 0.4319
-66.21 0.4324
-66.21 0.4333
-66.2 0.4338
-66.19 0.4349
-66.18 0.4354
-66.18 0.4361
-66.18 0.4366
-66.17 0.4373
-66.16 0.4386
-66.15 0.4396
-66.15 0.44
-66.15 0.4405
-66.15 0.441
-66.14 0.4421
-66.14 0.4435
-66.13 0.444
-66.13 0.4445
-66.13 0.4447
-66.12 0.4454
-66.12 0.4461
-66.12 0.4468
-66.11 0.4477
-66.1 0.4484
-66.1 0.4489
-66.09 0.4517
-66.07 0.4531
-66.05 0.4559
-66.05 0.4566
-66.05 0.4575
-66.05 0.458
-66.03 0.4594
-66.03 0.4603
-66.03 0.4608
-66.03 0.461
-66.03 0.4612
-66.02 0.4617
-66.02 0.4626
-66.01 0.4633
-66.01 0.4643
-66 0.4652
-65.98 0.4685
-65.98 0.4687
-65.97 0.4694
-65.97 0.4701
-65.96 0.4708
-65.96 0.4712
-65.96 0.4722
-65.96 0.4729
-65.96 0.4736
-65.95 0.474
-65.95 0.475
-65.95 0.4761
-65.95 0.4764
-65.94 0.4773
-65.94 0.4778
-65.93 0.4789
-65.92 0.482
-65.92 0.4827
-65.91 0.4841
-65.91 0.4843
-65.9 0.4852
-65.9 0.4864
-65.9 0.4866
-65.89 0.4873
-65.89 0.4885
-65.88 0.4899
-65.88 0.4903
-65.88 0.4908
-65.88 0.491
-65.87 0.492
-65.87 0.4927
-65.86 0.4934
-65.85 0.4964
-65.85 0.4969
-65.84 0.4973
-65.84 0.4983
-65.83 0.499
-65.83 0.4994
-65.82 0.5003
-65.82 0.5022
-65.8 0.5029
-65.79 0.505
-65.79 0.5059
-65.78 0.5071
-65.78 0.5076
-65.78 0.508
-65.78 0.5085
-65.77 0.5092
-65.76 0.5106
-65.76 0.5111
-65.76 0.5125
-65.75 0.5134
-65.74 0.5155
-65.74 0.5169
-65.73 0.5187
-65.72 0.5199
-65.72 0.5208
-65.71 0.5222
-65.71 0.5229
-65.71 0.5234
-65.7 0.5243
-65.69 0.5274
-65.68 0.5278
-65.68 0.5281
-65.68 0.5285
-65.67 0.5299
-65.67 0.5304
-65.67 0.5311
-65.66 0.5334
-65.66 0.5336
-65.66 0.5341
-65.65 0.536
-65.64 0.5367
-65.63 0.5385
-65.63 0.5395
-65.62 0.542
-65.62 0.5425
-65.61 0.5434
-65.61 0.5446
-65.6 0.5453
-65.6 0.546
-65.6 0.5464
-65.59 0.5471
-65.59 0.5476
-65.58 0.5481
-65.58 0.5485
-65.57 0.5504
-65.57 0.5516
-65.56 0.5525
-65.55 0.5532
-65.55 0.5534
-65.54 0.5546
-65.54 0.5548
-65.54 0.5553
-65.54 0.5555
-65.54 0.5562
-65.54 0.5567
-65.53 0.5579
-65.52 0.5586
-65.52 0.5593
-65.52 0.5602
-65.52 0.5607
-65.52 0.5611
-65.51 0.5616
-65.51 0.5623
-65.51 0.5627
-65.51 0.563
-65.5 0.5634
-65.5 0.5646
-65.5 0.5655
-65.49 0.5667
-65.48 0.5674
-65.48 0.569
-65.47 0.5695
-65.47 0.5702
-65.46 0.5707
-65.46 0.5711
-65.45 0.5716
-65.45 0.5725
-65.44 0.5751
-65.43 0.576
-65.43 0.5767
-65.42 0.5774
-65.42 0.5776
-65.42 0.5783
-65.41 0.5793
-65.41 0.5797
-65.41 0.5809
-65.4 0.5816
-65.4 0.5823
-65.39 0.583
-65.39 0.5837
-65.38 0.5849
-65.38 0.586
-65.36 0.5881
-65.36 0.5886
-65.35 0.5891
-65.35 0.59
-65.35 0.5907
-65.34 0.5912
-65.34 0.5916
-65.34 0.5919
-65.32 0.5935
-65.31 0.596
-65.3 0.597
-65.3 0.5974
-65.29 0.5981
-65.29 0.5984
-65.28 0.5995
-65.27 0.6014
-65.27 0.6021
-65.26 0.603
-65.25 0.6037
-65.25 0.6044
-65.24 0.6054
-65.23 0.6061
-65.22 0.6065
-65.21 0.6079
-65.2 0.6084
-65.19 0.6095
-65.19 0.6102
-65.18 0.6107
-65.18 0.6109
-65.18 0.6114
-65.18 0.6116
-65.17 0.6123
-65.17 0.6128
-65.16 0.6137
-65.15 0.617
-65.15 0.6172
-65.14 0.6182
-65.14 0.6189
-65.13 0.6198
-65.12 0.6205
-65.12 0.621
-65.12 0.6217
-65.12 0.6224
-65.1 0.6249
-65.1 0.6256
-65.1 0.6261
-65.09 0.627
-65.09 0.6286
-65.08 0.6291
-65.08 0.6298
-65.08 0.6305
-65.07 0.631
-65.07 0.6314
-65.06 0.6321
-65.05 0.634
-65.04 0.6345
-65.04 0.6349
-65.03 0.6356
-65.02 0.638
-65.01 0.6386
-65.01 0.6393
-65.01 0.6396
-65 0.6403
-65 0.6405
-64.99 0.6424
-64.99 0.6428
-64.97 0.6447
-64.96 0.6461
-64.96 0.6466
-64.96 0.648
-64.95 0.6484
-64.95 0.6489
-64.95 0.6494
-64.94 0.6503
-64.93 0.6522
-64.93 0.6529
-64.93 0.6533
-64.93 0.6545
-64.92 0.6549
-64.91 0.6568
-64.9 0.6575
-64.9 0.6582
-64.89 0.6587
-64.89 0.6594
-64.88 0.6596
-64.87 0.6605
-64.86 0.6619
-64.86 0.6624
-64.86 0.6633
-64.84 0.6657
-64.82 0.6678
-64.82 0.6687
-64.82 0.6698
-64.81 0.6703
-64.81 0.6708
-64.81 0.6717
-64.81 0.6719
-64.81 0.6726
-64.8 0.6733
-64.8 0.674
-64.8 0.6745
-64.8 0.6752
-64.79 0.6759
-64.78 0.6761
-64.78 0.6768
-64.77 0.6782
-64.77 0.6792
-64.77 0.6799
-64.77 0.6806
-64.73 0.6829
-64.72 0.6838
-64.72 0.6845
-64.72 0.6847
-64.72 0.6854
-64.71 0.6873
-64.7 0.6882
-64.7 0.6889
-64.69 0.692
-64.68 0.6927
-64.68 0.6934
-64.68 0.6943
-64.66 0.6957
-64.66 0.6962
-64.66 0.6966
-64.65 0.6976
-64.65 0.6983
-64.65 0.699
-64.64 0.7001
-64.64 0.7008
-64.63 0.7015
-64.62 0.7027
-64.62 0.7029
-64.61 0.7034
-64.61 0.7045
-64.6 0.7052
-64.6 0.7064
-64.6 0.7069
-64.6 0.7076
-64.59 0.7083
-64.59 0.709
-64.58 0.7094
-64.58 0.7097
-64.56 0.712
-64.56 0.7127
-64.55 0.7139
-64.55 0.7146
-64.55 0.715
-64.55 0.7155
-64.54 0.7164
-64.53 0.7197
-64.52 0.7201
-64.51 0.7215
-64.51 0.7225
-64.51 0.7227
-64.5 0.7234
-64.5 0.7236
-64.5 0.7241
-64.5 0.7246
-64.49 0.7253
-64.49 0.7267
-64.48 0.7274
-64.47 0.7281
-64.47 0.7288
-64.47 0.7295
-64.47 0.7297
-64.47 0.7304
-64.45 0.7327
-64.45 0.7332
-64.45 0.7341
-64.44 0.7355
-64.44 0.736
-64.43 0.7369
-64.43 0.7385
-64.42 0.7397
-64.41 0.742
-64.41 0.7427
-64.41 0.743
-64.39 0.7437
-64.39 0.7441
-64.37 0.7453
-64.35 0.749
-64.35 0.7499
-64.33 0.7511
-64.32 0.752
-64.32 0.7523
-64.31 0.7527
-64.31 0.7537
-64.29 0.7548
-64.29 0.7555
-64.29 0.756
-64.29 0.7567
-64.28 0.7574
-64.28 0.7576
-64.27 0.7583
-64.27 0.7588
-64.26 0.7597
-64.26 0.76
-64.25 0.762
-64.24 0.7625
-64.24 0.763
-64.23 0.7648
-64.23 0.7653
-64.22 0.7658
-64.22 0.7662
-64.21 0.7669
-64.2 0.7686
-64.19 0.769
-64.19 0.7697
-64.19 0.7704
-64.18 0.7709
-64.18 0.7716
-64.18 0.7723
-64.17 0.7739
-64.16 0.7746
-64.15 0.7753
-64.15 0.7756
-64.15 0.7763
-64.15 0.7765
-64.12 0.7809
-64.12 0.7818
-64.11 0.7823
-64.11 0.783
-64.11 0.7835
-64.11 0.7837
-64.1 0.7846
-64.07 0.7879
-64.07 0.7886
-64.06 0.7895
-64.04 0.7914
-64.03 0.7919
-64.03 0.7925
-64.02 0.7932
-64.02 0.7935
-64 0.7942
-64 0.7951
-64 0.7953
-63.98 0.7963
-63.96 0.7979
-63.94 0.7991
-63.94 0.7998
-63.94 0.8007
-63.94 0.8009
-63.92 0.8019
-63.91 0.8037
-63.91 0.8042
-63.9 0.8049
-63.9 0.8054
-63.9 0.8058
-63.89 0.8061
-63.88 0.8068
-63.87 0.8077
-63.87 0.8081
-63.87 0.8084
-63.86 0.8093
-63.86 0.8095
-63.85 0.8107
-63.84 0.8116
-63.84 0.8123
-63.82 0.8128
-63.82 0.813
-63.82 0.8135
-63.82 0.8137
-63.81 0.8142
-63.8 0.8172
-63.8 0.8179
-63.8 0.8184
-63.78 0.8198
-63.77 0.8217
-63.76 0.8226
-63.76 0.8237
-63.76 0.8242
-63.75 0.8247
-63.72 0.8275
-63.72 0.8282
-63.71 0.8291
-63.7 0.8296
-63.7 0.8303
-63.69 0.8312
-63.69 0.8319
-63.69 0.8324
-63.68 0.8338
-63.68 0.834
-63.64 0.8363
-63.64 0.8366
-63.63 0.8375
-63.63 0.838
-63.61 0.8398
-63.61 0.8403
-63.6 0.841
-63.6 0.8414
-63.6 0.8419
-63.59 0.8426
-63.59 0.8428
-63.59 0.8435
-63.59 0.8438
-63.57 0.8442
-63.56 0.8454
-63.54 0.8487
-63.52 0.8494
-63.52 0.8503
-63.51 0.8512
-63.51 0.8517
-63.5 0.8526
-63.5 0.8529
-63.5 0.8531
-63.49 0.8538
-63.49 0.854
-63.48 0.8547
-63.48 0.8549
-63.47 0.8563
-63.47 0.8568
-63.46 0.8575
-63.46 0.8584
-63.43 0.8617
-63.42 0.8622
-63.42 0.8624
-63.4 0.8643
-63.4 0.8647
-63.4 0.8652
-63.39 0.8654
-63.39 0.8659
-63.39 0.8661
-63.37 0.8668
-63.37 0.8678
-63.36 0.8694
-63.35 0.8698
-63.35 0.8703
-63.32 0.8722
-63.32 0.8726
-63.31 0.8733
-63.31 0.8738
-63.3 0.8745
-63.3 0.8747
-63.29 0.8754
-63.29 0.8759
-63.28 0.8766
-63.26 0.8785
-63.26 0.8796
-63.25 0.8801
-63.25 0.8803
-63.2 0.8831
-63.19 0.8838
-63.16 0.8861
-63.15 0.8866
-63.14 0.8875
-63.13 0.888
-63.13 0.8882
-63.12 0.8896
-63.12 0.8901
-63.1 0.8924
-63.08 0.8943
-63.08 0.895
-63.07 0.8957
-63.07 0.8962
-63.07 0.8969
-63.05 0.898
-63.05 0.8987
-63.04 0.8994
-63.04 0.9006
-63.03 0.9013
-63.03 0.902
-63.03 0.9024
-63.03 0.9029
-63.02 0.9036
-63.02 0.9041
-63 0.9048
-62.99 0.9052
-62.98 0.9066
-62.98 0.9071
-62.97 0.9078
-62.96 0.9083
-62.96 0.9087
-62.95 0.9094
-62.95 0.9099
-62.94 0.9106
-62.94 0.9108
-62.92 0.912
-62.91 0.9127
-62.87 0.915
-62.87 0.9155
-62.84 0.9192
-62.84 0.9194
-62.83 0.9201
-62.82 0.9213
-62.82 0.9218
-62.82 0.9229
-62.81 0.9232
-62.79 0.9239
-62.79 0.9241
-62.77 0.926
-62.77 0.9262
-62.77 0.9267
-62.76 0.9278
-62.76 0.9283
-62.75 0.9297
-62.74 0.9306
-62.73 0.9313
-62.72 0.932
-62.72 0.9325
-62.69 0.9341
-62.68 0.9348
-62.68 0.9355
-62.63 0.9376
-62.61 0.9383
-62.59 0.9404
-62.59 0.9409
-62.58 0.9411
-62.58 0.9416
-62.57 0.942
-62.56 0.9427
-62.54 0.9441
-62.54 0.9448
-62.52 0.9462
-62.51 0.9467
-62.5 0.9474
-62.47 0.9488
-62.47 0.9492
-62.45 0.9502
-62.45 0.9511
-62.42 0.953
-62.41 0.9548
-62.4 0.9555
-62.4 0.9558
-62.39 0.9562
-62.38 0.9569
-62.37 0.9576
-62.37 0.9588
-62.37 0.9593
-62.32 0.9609
-62.28 0.9627
-62.28 0.963
-62.25 0.9641
-62.24 0.9646
-62.23 0.9651
-62.23 0.9655
-62.22 0.9662
-62.22 0.9665
-62.21 0.9669
-62.2 0.9683
-62.19 0.969
-62.18 0.9695
-62.17 0.97
-62.17 0.9704
-62.13 0.9723
-62.13 0.9725
-62.12 0.9732
-62.1 0.9739
-62.1 0.9742
-62.08 0.9746
-62.08 0.9751
-62 0.9795
-61.91 0.9825
-61.85 0.9844
-61.83 0.9851
-61.82 0.9856
-61.82 0.9858
-61.82 0.986
-61.79 0.9867
-61.79 0.987
-61.77 0.9879
-61.66 0.9912
-61.64 0.9916
-61.64 0.9919
-61.58 0.9925
-61.56 0.993
-61.51 0.9953
-61.51 0.9956
-61.48 0.9963
-61.39 0.9977
-61.36 0.9988
-61.14 0.9995
-61.14 0.9998

};
%\addlegendentry{outside color2 - color4}
\addplot [thick, forget plot, color4]
table {%

-66.96 0
-66.83 0.0119
-66.59 0.02381
-66.49 0.03571
-66.49 0.04762
-66.47 0.05952
-66.39 0.07143
-66.36 0.08333
-66.35 0.09524
-66.35 0.1071
-66.32 0.119
-66.26 0.131
-66.21 0.1429
-66.09 0.1548
-66.09 0.1667
-66.09 0.1786
-66.08 0.1905
-66.07 0.2024
-66.07 0.2143
-66.06 0.2262
-66.05 0.2381
-65.93 0.25
-65.91 0.2619
-65.88 0.2857
-65.86 0.2976
-65.85 0.3095
-65.75 0.3214
-65.69 0.3333
-65.64 0.3452
-65.5 0.3571
-65.48 0.369
-65.48 0.381
-65.44 0.3929
-65.43 0.4048
-65.3 0.4167
-65.28 0.4286
-65.21 0.4405
-65.12 0.4524
-64.95 0.4643
-64.92 0.4762
-64.8 0.4881
-64.79 0.5
-64.76 0.5119
-64.74 0.5238
-64.71 0.5357
-64.58 0.5476
-64.53 0.5595
-64.46 0.5714
-64.43 0.5833
-64.42 0.5952
-64.39 0.6071
-64.39 0.619
-64.37 0.631
-64.36 0.6429
-64.36 0.6548
-64.35 0.6667
-64.35 0.6786
-64.34 0.6905
-64.34 0.7024
-64.33 0.7143
-64.3 0.7262
-64.25 0.7381
-64.24 0.75
-64.24 0.7619
-64.23 0.7738
-64.21 0.7857
-64.19 0.7976
-64.06 0.8095
-64.01 0.8214
-64 0.8333
-63.86 0.8452
-63.71 0.8571
-63.69 0.869
-63.59 0.881
-63.33 0.8929
-63.31 0.9048
-63.28 0.9167
-63.27 0.9286
-63.26 0.9405
-63.24 0.9524
-63.24 0.9643
-63.23 0.9762
-63.23 0.9881
};
%\addlegendentry{inside color2 - color4}
\addplot [thick, forget plot, color1, dashed]
table {%

-86.97 0
-86.24 0.0002272
-82.66 0.0009089
-82.04 0.001136
-79.59 0.001818
-79.15 0.002272
-77.95 0.002954
-75.63 0.005226
-75.62 0.005453
-75.57 0.005908
-75.56 0.006135
-75.48 0.006362
-74.79 0.007044
-74.55 0.007953
-74.51 0.00818
-74.19 0.009089
-73.7 0.01022
-73.3 0.01068
-72.26 0.01386
-72.01 0.01454
-72 0.01477
-71.44 0.01613
-71.43 0.01636
-71.41 0.01659
-71.26 0.01727
-71.24 0.0175
-71.08 0.0184
-70.98 0.01909
-70.78 0.01977
-70.69 0.02045
-70.5 0.02113
-70.42 0.02181
-70.37 0.02249
-70.03 0.02409
-69.86 0.02477
-69.85 0.02499
-69.59 0.02613
-69.29 0.02795
-69.17 0.02954
-69.1 0.03045
-68.98 0.03181
-68.44 0.03272
-68.44 0.03295
-67.92 0.03476
-67.76 0.03567
-67.75 0.0359
-67.57 0.03636
-67.57 0.03658
-67.5 0.03726
-67.48 0.03772
-67.45 0.0384
-67.36 0.03908
-67.36 0.03931
-67.24 0.04135
-67.16 0.04204
-66.66 0.04703
-66.59 0.0484
-66.43 0.04931
-66.42 0.05022
-66.18 0.05135
-66.17 0.05181
-66.06 0.05226
-66.03 0.05294
-66.02 0.0534
-65.89 0.05385
-65.68 0.05749
-65.68 0.05771
-65.66 0.0584
-65.64 0.05885
-65.61 0.0593
-65.6 0.05976
-65.47 0.06135
-65.46 0.06226
-65.44 0.06294
-65.44 0.06362
-65.42 0.06408
-65.42 0.0643
-65.39 0.06521
-65.27 0.06748
-65.27 0.06771
-65.16 0.06908
-65.15 0.06953
-64.97 0.07294
-64.97 0.07317
-64.95 0.07385
-64.88 0.07521
-64.85 0.07612
-64.83 0.07726
-64.83 0.07748
-64.78 0.07839
-64.78 0.07862
-64.77 0.07907
-64.73 0.07975
-64.72 0.08021
-64.66 0.08112
-64.64 0.0818
-64.58 0.08294
-64.53 0.08362
-64.53 0.08384
-64.41 0.08589
-64.38 0.08634
-64.35 0.08725
-64.32 0.08793
-64.31 0.08816
-64.26 0.08884
-64.22 0.0893
-64.22 0.08953
-64.1 0.09043
-64.04 0.09112
-64.03 0.09157
-64 0.09225
-63.98 0.09271
-63.95 0.09339
-63.95 0.09407
-63.81 0.09589
-63.77 0.09657
-63.61 0.09861
-63.53 0.1007
-63.52 0.1013
-63.49 0.1022
-63.46 0.1032
-63.44 0.1036
-63.44 0.1038
-63.42 0.1043
-63.42 0.1045
-63.26 0.1063
-63.22 0.1077
-63.22 0.1079
-63.16 0.1088
-63.16 0.1091
-63.15 0.1093
-63.1 0.11
-63.1 0.1102
-63.04 0.1113
-63.03 0.1122
-62.99 0.1129
-62.99 0.1132
-62.97 0.1136
-62.97 0.1138
-62.96 0.1141
-62.88 0.1157
-62.87 0.1161
-62.83 0.1168
-62.8 0.1175
-62.76 0.1182
-62.75 0.1184
-62.72 0.1191
-62.72 0.1193
-62.7 0.1197
-62.7 0.1202
-62.67 0.1211
-62.63 0.1218
-62.58 0.1232
-62.57 0.1236
-62.56 0.1243
-62.55 0.1247
-62.42 0.1266
-62.42 0.1268
-62.28 0.1291
-62.21 0.1297
-62.19 0.1307
-62.18 0.1313
-62.18 0.1316
-62.12 0.1334
-62.09 0.1341
-62.09 0.1343
-62.04 0.1347
-62.03 0.1357
-62.03 0.1359
-62 0.1366
-62 0.1368
-61.94 0.1377
-61.94 0.1379
-61.92 0.1386
-61.9 0.1391
-61.9 0.1397
-61.85 0.1413
-61.83 0.142
-61.78 0.1431
-61.76 0.1436
-61.76 0.1438
-61.7 0.1459
-61.67 0.1475
-61.6 0.1481
-61.6 0.1484
-61.6 0.1491
-61.56 0.1495
-61.56 0.1497
-61.52 0.1513
-61.52 0.1516
-61.48 0.1522
-61.47 0.1527
-61.45 0.1534
-61.45 0.1536
-61.4 0.155
-61.32 0.1575
-61.28 0.1581
-61.27 0.1591
-61.18 0.1609
-61.18 0.1611
-61.15 0.1618
-61.11 0.1629
-61.08 0.1638
-61.07 0.1645
-61.07 0.165
-61.06 0.1654
-61.04 0.1659
-61.03 0.1663
-61 0.1672
-60.96 0.1691
-60.93 0.1697
-60.82 0.1736
-60.79 0.1745
-60.79 0.1747
-60.76 0.1759
-60.73 0.177
-60.68 0.1779
-60.68 0.1781
-60.61 0.1802
-60.6 0.1811
-60.59 0.1818
-60.57 0.1827
-60.56 0.184
-60.56 0.1843
-60.53 0.1852
-60.53 0.1854
-60.52 0.1859
-60.52 0.1861
-60.5 0.1865
-60.49 0.1872
-60.47 0.1879
-60.46 0.1884
-60.45 0.189
-60.45 0.1893
-60.43 0.1906
-60.39 0.1913
-60.39 0.1918
-60.37 0.1925
-60.37 0.1927
-60.35 0.1936
-60.28 0.1968
-60.19 0.1981
-60.14 0.2
-60.11 0.2009
-60.11 0.2011
-60.04 0.2025
-60.04 0.2027
-60.02 0.2034
-60.01 0.204
-59.95 0.2056
-59.94 0.2061
-59.92 0.2068
-59.92 0.207
-59.88 0.2075
-59.88 0.2077
-59.86 0.2084
-59.85 0.209
-59.81 0.2097
-59.81 0.21
-59.79 0.2109
-59.79 0.2125
-59.77 0.2131
-59.76 0.2134
-59.76 0.214
-59.76 0.2145
-59.74 0.2156
-59.72 0.2165
-59.71 0.217
-59.69 0.2177
-59.69 0.2179
-59.68 0.2186
-59.65 0.2199
-59.65 0.2202
-59.59 0.2211
-59.53 0.2245
-59.52 0.2252
-59.51 0.2256
-59.48 0.2272
-59.48 0.2274
-59.46 0.2281
-59.44 0.2297
-59.43 0.2304
-59.43 0.2306
-59.42 0.2311
-59.42 0.2318
-59.41 0.2322
-59.41 0.2327
-59.39 0.2336
-59.38 0.2343
-59.31 0.237
-59.31 0.2372
-59.28 0.2381
-59.25 0.2399
-59.24 0.2409
-59.24 0.2415
-59.24 0.2418
-59.22 0.2427
-59.21 0.2434
-59.17 0.247
-59.13 0.2486
-59.1 0.2502
-59.1 0.2504
-59.04 0.2524
-59.04 0.2529
-59.04 0.2531
-59.02 0.2538
-59.01 0.254
-59 0.2545
-59 0.2549
-59 0.2554
-58.99 0.2561
-58.97 0.2565
-58.97 0.2568
-58.95 0.2574
-58.95 0.2577
-58.94 0.259
-58.94 0.2595
-58.9 0.2613
-58.9 0.262
-58.86 0.2638
-58.83 0.2652
-58.8 0.2663
-58.79 0.2668
-58.78 0.2674
-58.78 0.2679
-58.76 0.2693
-58.73 0.2711
-58.71 0.2718
-58.71 0.2724
-58.69 0.2729
-58.69 0.2733
-58.64 0.2763
-58.64 0.2774
-58.61 0.2781
-58.61 0.2783
-58.58 0.279
-58.58 0.2793
-58.56 0.2804
-58.56 0.2811
-58.54 0.2822
-58.54 0.2824
-58.52 0.2829
-58.52 0.2836
-58.51 0.284
-58.51 0.2843
-58.5 0.2847
-58.49 0.2852
-58.48 0.2858
-58.48 0.2861
-58.48 0.2865
-58.47 0.2872
-58.44 0.2883
-58.44 0.2893
-58.42 0.2899
-58.42 0.2908
-58.4 0.2913
-58.4 0.292
-58.4 0.2924
-58.39 0.2931
-58.38 0.2936
-58.34 0.2947
-58.34 0.2949
-58.33 0.2954
-58.3 0.2965
-58.29 0.2972
-58.26 0.299
-58.24 0.3004
-58.22 0.3024
-58.21 0.3031
-58.21 0.3038
-58.21 0.3042
-58.18 0.3056
-58.18 0.3061
-58.16 0.3072
-58.16 0.3074
-58.14 0.3081
-58.11 0.3099
-58.05 0.3127
-58.04 0.3138
-58 0.3161
-58 0.3163
-58 0.3167
-57.97 0.3174
-57.96 0.3186
-57.95 0.3195
-57.95 0.3197
-57.95 0.3199
-57.92 0.3208
-57.92 0.3211
-57.92 0.3213
-57.91 0.322
-57.91 0.3222
-57.9 0.3229
-57.9 0.3231
-57.88 0.3249
-57.84 0.3265
-57.84 0.3267
-57.83 0.3274
-57.83 0.3279
-57.8 0.3297
-57.78 0.3311
-57.77 0.3317
-57.76 0.3327
-57.71 0.3336
-57.71 0.334
-57.71 0.3342
-57.66 0.3363
-57.65 0.3367
-57.64 0.3372
-57.63 0.3379
-57.63 0.3383
-57.62 0.3397
-57.56 0.3422
-57.55 0.3431
-57.54 0.3438
-57.51 0.3456
-57.49 0.3461
-57.49 0.3467
-57.48 0.3474
-57.46 0.3481
-57.42 0.3504
-57.42 0.3515
-57.38 0.3538
-57.37 0.3542
-57.34 0.3567
-57.33 0.3574
-57.31 0.3586
-57.31 0.3588
-57.28 0.3611
-57.28 0.3615
-57.27 0.3622
-57.25 0.3638
-57.22 0.3676
-57.19 0.3695
-57.18 0.3701
-57.14 0.3736
-57.12 0.3742
-57.09 0.3761
-57.06 0.3767
-57.05 0.3776
-57.04 0.3788
-57 0.3822
-57 0.3824
-57 0.3826
-56.99 0.3833
-56.98 0.3842
-56.97 0.3851
-56.96 0.3858
-56.95 0.3867
-56.95 0.3876
-56.94 0.3883
-56.94 0.3885
-56.93 0.3892
-56.92 0.3897
-56.9 0.3915
-56.89 0.392
-56.89 0.3926
-56.87 0.3938
-56.86 0.3949
-56.84 0.3956
-56.84 0.3958
-56.83 0.3963
-56.82 0.3967
-56.8 0.3983
-56.8 0.3985
-56.79 0.3995
-56.79 0.3997
-56.78 0.401
-56.76 0.4017
-56.72 0.4033
-56.7 0.4038
-56.7 0.4042
-56.67 0.4067
-56.67 0.4072
-56.67 0.4074
-56.65 0.4097
-56.63 0.4104
-56.63 0.4108
-56.63 0.4113
-56.58 0.4129
-56.58 0.4133
-56.56 0.4149
-56.55 0.4158
-56.54 0.4167
-56.53 0.4174
-56.52 0.4179
-56.52 0.4181
-56.51 0.4185
-56.51 0.419
-56.51 0.4195
-56.48 0.4204
-56.48 0.4215
-56.48 0.4217
-56.46 0.4222
-56.45 0.4235
-56.45 0.4238
-56.43 0.4247
-56.43 0.4256
-56.42 0.4272
-56.42 0.4276
-56.41 0.4281
-56.41 0.4285
-56.41 0.4292
-56.39 0.4315
-56.38 0.4322
-56.38 0.4324
-56.37 0.4326
-56.36 0.4331
-56.35 0.4344
-56.34 0.4354
-56.34 0.436
-56.33 0.4367
-56.32 0.4372
-56.31 0.4381
-56.29 0.4399
-56.29 0.4406
-56.27 0.4419
-56.26 0.4429
-56.24 0.4438
-56.23 0.4444
-56.22 0.446
-56.21 0.4469
-56.21 0.4472
-56.2 0.4479
-56.17 0.4501
-56.17 0.4504
-56.15 0.451
-56.15 0.4524
-56.14 0.4531
-56.14 0.4533
-56.14 0.4538
-56.13 0.4544
-56.13 0.4549
-56.12 0.4558
-56.12 0.4565
-56.11 0.4572
-56.11 0.4574
-56.1 0.4579
-56.08 0.4597
-56.06 0.4608
-56.06 0.4615
-56.04 0.4622
-56.04 0.4628
-56.03 0.4635
-56.03 0.4642
-56.02 0.4647
-56.02 0.4651
-56.01 0.466
-55.99 0.4669
-55.99 0.4683
-55.97 0.469
-55.97 0.4699
-55.95 0.4706
-55.95 0.471
-55.94 0.4717
-55.92 0.4731
-55.92 0.4735
-55.91 0.474
-55.9 0.4753
-55.9 0.4756
-55.88 0.4767
-55.88 0.4769
-55.86 0.4774
-55.85 0.4781
-55.84 0.479
-55.84 0.4794
-55.83 0.4799
-55.83 0.4803
-55.81 0.4817
-55.78 0.4831
-55.75 0.4842
-55.75 0.4847
-55.74 0.4853
-55.74 0.4856
-55.73 0.4863
-55.72 0.4867
-55.72 0.4872
-55.72 0.4874
-55.71 0.4885
-55.68 0.4894
-55.68 0.4908
-55.67 0.4915
-55.66 0.4926
-55.65 0.4931
-55.65 0.4933
-55.64 0.4942
-55.64 0.4944
-55.63 0.4949
-55.63 0.4953
-55.61 0.4963
-55.61 0.4967
-55.6 0.4969
-55.59 0.4976
-55.59 0.4978
-55.58 0.4985
-55.58 0.499
-55.57 0.4997
-55.56 0.5001
-55.56 0.5008
-55.55 0.5015
-55.54 0.5022
-55.54 0.5024
-55.54 0.5028
-55.54 0.5033
-55.53 0.5035
-55.53 0.5042
-55.52 0.5051
-55.51 0.5056
-55.48 0.5074
-55.48 0.5078
-55.46 0.5101
-55.45 0.5108
-55.44 0.5122
-55.43 0.5133
-55.42 0.5147
-55.36 0.5174
-55.36 0.5176
-55.35 0.5183
-55.32 0.5215
-55.31 0.5219
-55.31 0.5224
-55.29 0.5231
-55.29 0.5237
-55.28 0.5242
-55.28 0.5244
-55.28 0.5249
-55.28 0.5253
-55.27 0.5258
-55.26 0.5262
-55.25 0.5276
-55.22 0.5283
-55.22 0.529
-55.21 0.5294
-55.19 0.5324
-55.18 0.5331
-55.16 0.5344
-55.16 0.5347
-55.16 0.5353
-55.16 0.5356
-55.14 0.5362
-55.13 0.5378
-55.12 0.539
-55.12 0.5394
-55.11 0.5403
-55.09 0.5412
-55.06 0.5426
-55.04 0.5435
-55.04 0.544
-55.03 0.5446
-55.03 0.5451
-55.02 0.546
-55.01 0.5465
-55.01 0.5471
-55 0.5478
-54.97 0.5496
-54.96 0.5506
-54.96 0.551
-54.95 0.5517
-54.93 0.5531
-54.93 0.5533
-54.91 0.554
-54.91 0.5544
-54.88 0.5551
-54.87 0.5562
-54.87 0.5569
-54.86 0.5574
-54.86 0.5578
-54.84 0.5603
-54.82 0.5615
-54.8 0.5626
-54.79 0.5633
-54.77 0.5642
-54.77 0.5649
-54.76 0.5656
-54.76 0.5658
-54.73 0.5676
-54.73 0.5678
-54.71 0.569
-54.71 0.5694
-54.71 0.5703
-54.7 0.571
-54.69 0.5719
-54.68 0.5726
-54.67 0.5731
-54.66 0.5746
-54.66 0.5756
-54.66 0.5758
-54.64 0.5771
-54.64 0.5776
-54.63 0.5781
-54.63 0.5785
-54.63 0.5796
-54.61 0.5803
-54.61 0.5805
-54.59 0.5812
-54.58 0.5828
-54.58 0.5837
-54.57 0.5844
-54.57 0.5846
-54.56 0.5855
-54.55 0.5862
-54.55 0.5867
-54.54 0.5871
-54.53 0.5883
-54.51 0.5887
-54.51 0.589
-54.49 0.5894
-54.49 0.5905
-54.49 0.591
-54.47 0.5919
-54.47 0.5924
-54.46 0.593
-54.46 0.5935
-54.45 0.594
-54.45 0.5942
-54.35 0.5996
-54.33 0.6003
-54.32 0.6005
-54.32 0.601
-54.32 0.6017
-54.3 0.6035
-54.29 0.6044
-54.28 0.6049
-54.28 0.6053
-54.27 0.6058
-54.27 0.6062
-54.26 0.6069
-54.26 0.6074
-54.25 0.6083
-54.25 0.609
-54.23 0.6103
-54.2 0.6119
-54.18 0.613
-54.17 0.6144
-54.17 0.6149
-54.16 0.6155
-54.16 0.6162
-54.14 0.6171
-54.14 0.6176
-54.14 0.6183
-54.11 0.6199
-54.11 0.6201
-54.11 0.6205
-54.07 0.6224
-54.05 0.6239
-54.05 0.6244
-54.03 0.6253
-54.03 0.626
-54.01 0.6267
-54.01 0.6271
-53.98 0.6299
-53.97 0.6308
-53.96 0.6321
-53.95 0.6326
-53.95 0.633
-53.93 0.6337
-53.92 0.6342
-53.92 0.6349
-53.91 0.636
-53.91 0.6371
-53.91 0.6374
-53.88 0.6394
-53.86 0.6403
-53.86 0.6408
-53.86 0.641
-53.85 0.6417
-53.85 0.6421
-53.84 0.6428
-53.84 0.6437
-53.83 0.646
-53.82 0.6464
-53.82 0.6471
-53.82 0.6483
-53.81 0.6489
-53.8 0.6503
-53.8 0.6508
-53.79 0.6514
-53.78 0.6526
-53.78 0.653
-53.77 0.6539
-53.76 0.6544
-53.76 0.6549
-53.75 0.6555
-53.74 0.6564
-53.74 0.6571
-53.74 0.6574
-53.73 0.6583
-53.73 0.6585
-53.68 0.6617
-53.65 0.6626
-53.65 0.6628
-53.64 0.6639
-53.62 0.6646
-53.62 0.6648
-53.6 0.666
-53.58 0.6667
-53.57 0.6673
-53.56 0.6683
-53.55 0.6696
-53.55 0.6701
-53.54 0.6705
-53.54 0.6708
-53.51 0.6719
-53.5 0.673
-53.49 0.6739
-53.49 0.6748
-53.47 0.6767
-53.45 0.6773
-53.45 0.6776
-53.44 0.678
-53.42 0.6787
-53.41 0.6798
-53.41 0.6808
-53.4 0.6817
-53.39 0.6828
-53.39 0.6835
-53.39 0.6839
-53.39 0.6844
-53.38 0.6848
-53.38 0.6853
-53.37 0.686
-53.37 0.6864
-53.37 0.6869
-53.36 0.6876
-53.36 0.6878
-53.34 0.6898
-53.34 0.6905
-53.34 0.6912
-53.33 0.6921
-53.3 0.6942
-53.28 0.6962
-53.28 0.6964
-53.28 0.6967
-53.27 0.6971
-53.27 0.6978
-53.27 0.6983
-53.26 0.6989
-53.26 0.6992
-53.25 0.7001
-53.23 0.7019
-53.22 0.7026
-53.21 0.7032
-53.21 0.7039
-53.21 0.7046
-53.21 0.7048
-53.15 0.7085
-53.15 0.7087
-53.15 0.7094
-53.14 0.7105
-53.13 0.7114
-53.1 0.7144
-53.09 0.716
-53.08 0.7167
-53.08 0.7173
-53.07 0.7182
-53.07 0.7187
-53.06 0.7194
-53.06 0.7203
-53.06 0.7205
-53.05 0.7212
-53.05 0.7214
-53.04 0.7228
-53.04 0.723
-53.02 0.7239
-53.02 0.7248
-53.01 0.7255
-53.01 0.726
-53 0.7267
-53 0.7269
-52.99 0.728
-52.98 0.7285
-52.98 0.7289
-52.97 0.7301
-52.97 0.7307
-52.96 0.7312
-52.96 0.7319
-52.95 0.733
-52.95 0.7332
-52.95 0.7337
-52.95 0.7339
-52.94 0.7342
-52.92 0.7353
-52.91 0.7364
-52.91 0.7367
-52.9 0.7373
-52.89 0.738
-52.87 0.7387
-52.87 0.7392
-52.86 0.7401
-52.86 0.7405
-52.86 0.741
-52.86 0.7412
-52.84 0.7426
-52.83 0.7437
-52.82 0.7446
-52.81 0.7457
-52.8 0.7464
-52.8 0.7469
-52.8 0.7471
-52.79 0.7485
-52.76 0.7514
-52.76 0.7521
-52.75 0.7526
-52.75 0.7528
-52.72 0.7548
-52.72 0.7557
-52.71 0.7569
-52.69 0.7598
-52.67 0.7605
-52.67 0.7614
-52.66 0.7619
-52.66 0.7626
-52.65 0.7628
-52.64 0.7639
-52.64 0.7641
-52.63 0.7646
-52.63 0.7653
-52.63 0.7662
-52.62 0.7669
-52.62 0.7687
-52.61 0.7701
-52.61 0.7705
-52.61 0.7721
-52.6 0.7737
-52.6 0.7773
-52.59 0.7789
-52.59 0.7794
-52.59 0.7798
-52.59 0.7807
-52.58 0.7812
-52.58 0.7819
-52.58 0.7825
-52.58 0.7837
-52.57 0.7844
-52.56 0.7875
-52.55 0.7882
-52.55 0.7889
-52.52 0.7912
-52.51 0.7919
-52.5 0.7925
-52.49 0.7937
-52.48 0.7944
-52.48 0.7946
-52.47 0.7953
-52.45 0.7973
-52.43 0.7985
-52.42 0.7991
-52.42 0.7996
-52.41 0.8
-52.4 0.8007
-52.4 0.8012
-52.39 0.8019
-52.38 0.803
-52.37 0.8032
-52.37 0.8041
-52.34 0.8048
-52.34 0.805
-52.34 0.8055
-52.34 0.8057
-52.31 0.8071
-52.28 0.8082
-52.26 0.8091
-52.25 0.8098
-52.25 0.81
-52.22 0.811
-52.22 0.8116
-52.21 0.8121
-52.21 0.813
-52.2 0.8137
-52.2 0.8141
-52.19 0.815
-52.18 0.816
-52.18 0.8162
-52.16 0.8166
-52.15 0.8171
-52.13 0.8182
-52.13 0.8185
-52.11 0.8194
-52.11 0.8196
-52.09 0.8203
-52.05 0.8221
-52.04 0.8225
-52.04 0.8234
-52.03 0.8239
-52.02 0.8244
-52.01 0.8246
-52 0.8253
-51.98 0.8273
-51.96 0.8284
-51.96 0.8287
-51.96 0.8291
-51.96 0.8294
-51.93 0.8314
-51.9 0.8325
-51.9 0.8328
-51.83 0.8373
-51.83 0.8378
-51.8 0.8396
-51.8 0.8405
-51.78 0.8409
-51.78 0.8412
-51.75 0.8425
-51.69 0.8459
-51.69 0.8464
-51.68 0.8475
-51.62 0.8505
-51.62 0.8512
-51.62 0.8514
-51.61 0.8521
-51.6 0.8525
-51.59 0.8532
-51.59 0.8539
-51.58 0.8544
-51.58 0.8546
-51.55 0.8553
-51.55 0.8559
-51.54 0.8564
-51.5 0.858
-51.47 0.86
-51.46 0.8605
-51.45 0.8614
-51.45 0.8621
-51.44 0.863
-51.43 0.8637
-51.41 0.8648
-51.39 0.8666
-51.38 0.8673
-51.35 0.8693
-51.35 0.8696
-51.31 0.8714
-51.3 0.8718
-51.3 0.8721
-51.28 0.8725
-51.27 0.8741
-51.25 0.875
-51.24 0.8759
-51.22 0.8766
-51.22 0.8771
-51.19 0.8782
-51.19 0.8784
-51.18 0.8793
-51.15 0.8805
-51.14 0.8807
-51.13 0.8816
-51.1 0.883
-51.08 0.8843
-51.08 0.8846
-51.06 0.8857
-51.06 0.8859
-51.03 0.8868
-51.02 0.8875
-51.02 0.888
-51 0.8887
-50.96 0.8898
-50.96 0.8905
-50.94 0.8914
-50.93 0.8928
-50.88 0.8946
-50.88 0.8948
-50.87 0.8955
-50.87 0.8962
-50.85 0.8971
-50.82 0.8984
-50.8 0.8991
-50.78 0.9002
-50.78 0.9005
-50.71 0.9025
-50.63 0.9052
-50.63 0.9055
-50.62 0.9059
-50.6 0.9068
-50.6 0.9071
-50.58 0.908
-50.57 0.9089
-50.55 0.9098
-50.53 0.9107
-50.47 0.9125
-50.42 0.9132
-50.42 0.9137
-50.4 0.9143
-50.36 0.9148
-50.31 0.9171
-50.31 0.9177
-50.3 0.918
-50.25 0.9187
-50.21 0.9202
-50.2 0.9207
-50.19 0.9216
-50.18 0.9221
-50.14 0.923
-50.09 0.9259
-50.09 0.9262
-50.05 0.9275
-50 0.9282
-49.82 0.9355
-49.82 0.9359
-49.8 0.9368
-49.79 0.9371
-49.77 0.9382
-49.77 0.9384
-49.74 0.9393
-49.73 0.94
-49.73 0.9402
-49.69 0.9416
-49.69 0.9418
-49.61 0.9446
-49.61 0.9448
-49.53 0.9471
-49.52 0.948
-49.49 0.9493
-49.48 0.9498
-49.45 0.9507
-49.45 0.9509
-49.38 0.9523
-49.38 0.9525
-49.31 0.9539
-49.24 0.955
-49.23 0.9555
-49.1 0.958
-49.06 0.9584
-49.06 0.9589
-49 0.96
-49 0.9602
-48.95 0.9609
-48.53 0.9664
-48.52 0.9668
-48.49 0.9675
-48.48 0.9677
-48.43 0.9693
-48.43 0.9696
-48.43 0.9698
-48.25 0.9714
-48.16 0.9721
-48.16 0.9723
-48.14 0.973
-48.13 0.9734
-48.05 0.9746
-48.04 0.9748
-47.99 0.9755
-47.99 0.9757
-47.95 0.9764
-47.94 0.9768
-47.85 0.9775
-47.84 0.9777
-47.78 0.9789
-47.78 0.9793
-47.77 0.9798
-47.74 0.9809
-47.71 0.9818
-47.7 0.9825
-47.62 0.9841
-47.6 0.9848
-47.59 0.9852
-47.56 0.9857
-47.5 0.9864
-47.47 0.9868
-47.47 0.987
-47.32 0.9895
-47.3 0.9902
-47.12 0.9918
-47.11 0.992
-47.07 0.9927
-47.06 0.9934
-47.04 0.9943
-46.96 0.995
-46.82 0.9957
-46.62 0.9968
-46.45 0.9973
-46.45 0.9975
-46.44 0.998
-46.17 0.9998

};
%\addlegendentry{outside color2 - color1 (pi/2)}
\addplot [thick, forget plot, color1]
table {%

-55.97 0
-55.96 0.007752
-55.62 0.0155
-55.62 0.02326
-55.39 0.03101
-55.36 0.03876
-55.36 0.04651
-55.33 0.05426
-55.29 0.06202
-55.23 0.06977
-55.22 0.07752
-55.2 0.08527
-55.18 0.09302
-55.17 0.1008
-55.14 0.1085
-55.13 0.1163
-55.13 0.124
-55.12 0.1318
-55.12 0.1473
-55.11 0.155
-55.11 0.1628
-55.11 0.1705
-55.09 0.1783
-55.08 0.186
-55.08 0.1938
-55.08 0.2016
-55.07 0.2093
-55.07 0.2171
-55.06 0.2248
-55.05 0.2326
-55.04 0.2558
-55.03 0.2713
-55.03 0.2868
-55.02 0.2946
-55.02 0.3023
-55.02 0.3101
-55.01 0.3178
-55.01 0.3256
-55.01 0.3333
-55.01 0.3411
-55.01 0.3488
-55.01 0.3566
-55 0.3643
-55 0.3721
-55 0.3798
-55 0.3876
-54.99 0.3953
-54.99 0.4031
-54.99 0.4109
-54.98 0.4186
-54.97 0.4264
-54.95 0.4341
-54.95 0.4419
-54.89 0.4496
-54.78 0.4574
-54.68 0.4729
-54.58 0.4806
-54.5 0.4961
-54.47 0.5039
-54.45 0.5116
-54.38 0.5194
-54.37 0.5271
-54.35 0.5349
-54.34 0.5426
-54.25 0.5504
-54.24 0.5581
-54.2 0.5659
-54.13 0.5736
-54.12 0.5814
-54.12 0.5891
-54.02 0.5969
-54.01 0.6047
-53.99 0.6124
-53.98 0.6202
-53.98 0.6279
-53.92 0.6357
-53.85 0.6434
-53.85 0.6512
-53.81 0.6589
-53.79 0.6667
-53.68 0.6744
-53.65 0.6822
-53.62 0.6899
-53.61 0.6977
-53.61 0.7054
-53.56 0.7132
-53.54 0.7209
-53.53 0.7287
-53.5 0.7364
-53.49 0.7442
-53.46 0.7519
-53.4 0.7597
-53.39 0.7674
-53.39 0.7752
-53.38 0.7829
-53.32 0.7907
-53.3 0.7984
-53.28 0.8062
-53.28 0.814
-53.22 0.8217
-53.22 0.8295
-53.22 0.8372
-53.21 0.845
-53.2 0.8527
-53.17 0.8605
-53.17 0.8682
-53.03 0.876
-53.03 0.8837
-53.02 0.8915
-53.01 0.8992
-52.91 0.907
-52.89 0.9147
-52.79 0.9225
-52.78 0.9302
-52.76 0.938
-52.66 0.9457
-52.62 0.9535
-52.53 0.9612
-52.51 0.969
-52.36 0.9767
-52.34 0.9845
-52.26 0.9922

};
%\addlegendentry{inside color2 - color1 (pi/2)}


\addlegendimage{no markers, black, line width=1pt}
\addlegendimage{no markers, black, dashed, line width=1pt}

\addlegendimage{empty legend}

\addlegendimage{draw opacity=0, area legend, no markers, fill=color2}
\addlegendimage{draw opacity=0, area legend, no markers, fill=color3}
\addlegendimage{draw opacity=0, area legend, no markers, fill=color4}
\addlegendimage{draw opacity=0, area legend, no markers, fill=color1}


\legend{Inside focal region, Outside focal region, \phantom{placehold}, strategy 1, strategy 2,strategy 3, strategy 4}
\end{axis}

\end{tikzpicture}

\end{figure}

\begin{figure}[h]
    \centering
    \setlength{\figurewidth}{0.6\linewidth}
    \setlength{\figureheight}{0.3\linewidth}
    % This file wcolor created with tikzplotlib v0.10.1.post9.
\begin{tikzpicture}

\begin{axis}[
legend style={nodes={scale=0.8, transform shape}}, 
width=\figurewidth,
height=\figureheight,
legend cell align={left},
legend style={
  fill opacity=0.8,
  draw opacity=1,
  text opacity=1,
  at={(0.03,0.97)},
  anchor=north west,
},
tick align=outside,
tick pos=left,
%xmin=-80.2, xmax=-38.8,
xmin=-80, xmax=-40,
xtick style={color=black},
%ymin=-0.05, ymax=1.05,
ymin=0, ymax=1.02,
ytick style={color=black},
tick align=outside,
tick pos=left,
xtick style={color=black},
ytick style={color=black},
grid style=dotted,
ymajorgrids,
xmajorgrids,
clip mode=individual,
ytick={0,0.2,0.4,0.6,0.8,1.0},
extra y ticks={0,0.1,0.2,0.3,0.4,0.5,0.6,0.7,0.8,0.9,1.0},
extra y tick label=\empty,
ylabel={\Acrshort{ecdf}},
xlabel={\Acrlong{rss} in dBm}
]
\addplot [thick, forget plot, color2, dashed]
table {%
-102.2 0
-86.71 0.0004762
-85.43 0.000697
-82.42 0.001208
-79.46 0.002436
-79.44 0.002478
-77.68 0.003747
-77.64 0.003782
-77.19 0.004272
-76.95 0.004589
-76.52 0.005121
-76.42 0.005238
-76.05 0.005756
-75.86 0.006101
-75.47 0.006736
-75.22 0.007198
-74.78 0.008137
-74.27 0.008992
-73.97 0.009634
-73.91 0.009745
-73.63 0.01035
-73.52 0.01056
-73.16 0.0115
-73.07 0.01171
-72.89 0.01218
-72.65 0.01277
-72.63 0.01284
-72.46 0.01331
-72.44 0.0134
-72.4 0.01357
-71.97 0.01492
-71.37 0.01714
-71.24 0.01771
-71.22 0.01778
-71.08 0.01838
-70.88 0.01929
-70.8 0.01972
-70.65 0.02055
-70.12 0.02301
-70.03 0.02352
-70 0.02366
-69.89 0.02411
-69.75 0.02488
-69.74 0.025
-69.55 0.02614
-69.52 0.02629
-69.44 0.02681
-69.42 0.02696
-69.32 0.02749
-69.27 0.02781
-69.21 0.02828
-69.15 0.0287
-69.1 0.0292
-69.03 0.02959
-68.93 0.03022
-68.91 0.03053
-68.77 0.03167
-68.36 0.03465
-68.33 0.03483
-68.26 0.03539
-68.19 0.03607
-67.51 0.04232
-66.86 0.04896
-66.85 0.04916
-66.82 0.04959
-66.77 0.05024
-66.72 0.05097
-66.63 0.05183
-66.57 0.05246
-66.55 0.05275
-66.4 0.05458
-65.38 0.06924
-65.36 0.06972
-65.31 0.07045
-64.99 0.07557
-64.95 0.07614
-64.88 0.07754
-64.83 0.07846
-64.77 0.07979
-64.2 0.09066
-64.11 0.09242
-64.09 0.09284
-64.07 0.09338
-64.03 0.0942
-63.94 0.09621
-63.93 0.09672
-63.88 0.09794
-63.85 0.09843
-63.84 0.09872
-63.82 0.09917
-63.7 0.1017
-63.66 0.1027
-63.64 0.1031
-63.59 0.1041
-63.56 0.1049
-63.53 0.1055
-63.51 0.1059
-63.48 0.1069
-63.45 0.1074
-63.34 0.1102
-63.32 0.1106
-62.91 0.1212
-62.89 0.1217
-62.84 0.123
-62.8 0.124
-62.63 0.1286
-62.62 0.1288
-62.59 0.1294
-62.55 0.1305
-62.46 0.1329
-62.45 0.1332
-62.43 0.1338
-62.41 0.1343
-62.38 0.1352
-62.27 0.1383
-62.24 0.1389
-62.13 0.1422
-62.1 0.143
-62.08 0.1435
-62.08 0.1439
-62.07 0.1443
-62.06 0.1451
-62.06 0.1457
-62.05 0.1469
-62.05 0.1476
-62.04 0.1498
-62.02 0.154
-62.02 0.155
-62.01 0.1566
-62.01 0.1574
-62 0.1603
-61.98 0.1689
-61.97 0.1704
-61.95 0.178
-61.94 0.1799
-61.94 0.1809
-61.93 0.184
-61.92 0.1859
-61.9 0.1933
-61.89 0.1961
-61.89 0.198
-61.86 0.2079
-61.84 0.212
-61.84 0.2127
-61.81 0.2232
-61.81 0.2238
-61.81 0.2245
-61.78 0.2326
-61.78 0.2336
-61.73 0.2452
-61.73 0.2463
-61.72 0.2489
-61.72 0.2495
-61.72 0.2505
-61.7 0.2563
-61.69 0.2576
-61.67 0.2633
-61.66 0.2641
-61.66 0.266
-61.65 0.2667
-61.65 0.2682
-61.61 0.2774
-61.6 0.279
-61.59 0.2799
-61.58 0.2826
-61.58 0.2835
-61.57 0.2849
-61.56 0.286
-61.56 0.2873
-61.55 0.2882
-61.55 0.289
-61.54 0.2906
-61.53 0.2921
-61.47 0.3033
-61.46 0.3041
-61.46 0.305
-61.45 0.3066
-61.45 0.3072
-61.44 0.3084
-61.34 0.3215
-61.34 0.3222
-61.33 0.3228
-61.32 0.3238
-61.31 0.3257
-61.25 0.3321
-61.22 0.3371
-61.21 0.3377
-61.2 0.3384
-61.18 0.339
-61.16 0.3397
-61.16 0.34
-61.14 0.3406
-61.1 0.3421
-61.09 0.3425
-61.07 0.343
-61 0.3453
-60.96 0.3466
-60.95 0.3472
-60.92 0.3481
-60.9 0.3488
-60.89 0.3491
-60.87 0.3495
-60.81 0.3517
-60.8 0.3521
-60.77 0.353
-60.72 0.3545
-60.71 0.355
-60.69 0.3556
-60.66 0.3567
-60.62 0.3582
-60.57 0.3601
-60.56 0.3606
-60.54 0.3611
-60.52 0.3619
-60.48 0.3633
-60.45 0.3646
-60.43 0.3652
-60.33 0.3686
-60.29 0.3704
-60.24 0.3719
-60.22 0.3729
-60.12 0.3764
-60.09 0.3779
-60 0.3811
-59.97 0.3822
-59.96 0.3828
-59.95 0.3833
-59.9 0.3852
-59.89 0.3856
-59.88 0.3862
-59.86 0.3868
-59.78 0.3902
-59.77 0.3907
-59.76 0.3911
-59.75 0.3916
-59.72 0.3928
-59.7 0.3936
-59.68 0.3946
-59.66 0.3952
-59.61 0.3974
-59.6 0.3978
-59.28 0.411
-59.23 0.4131
-59.16 0.4162
-59.14 0.4168
-59.13 0.4173
-59.12 0.4179
-59.07 0.42
-59.06 0.4206
-58.95 0.425
-58.94 0.4255
-58.93 0.4261
-58.84 0.4299
-58.83 0.4305
-58.81 0.4314
-58.71 0.4361
-58.7 0.4365
-58.66 0.4381
-58.65 0.4387
-58.63 0.4402
-58.61 0.4408
-58.34 0.4541
-58.32 0.4551
-58.31 0.4556
-58.27 0.4577
-58.26 0.4583
-58.26 0.4587
-58.25 0.4592
-58.23 0.4601
-58.22 0.4606
-58.15 0.4642
-58.13 0.4652
-58.12 0.4656
-58.08 0.4681
-58.06 0.4688
-57.9 0.4772
-57.89 0.4778
-57.81 0.4822
-57.79 0.4837
-57.78 0.4842
-57.76 0.485
-57.74 0.4861
-57.73 0.4866
-57.7 0.4886
-57.69 0.4891
-57.68 0.4893
-57.67 0.4897
-57.66 0.4903
-57.65 0.4909
-57.64 0.4914
-57.63 0.4919
-56.73 0.5443
-56.72 0.5448
-56.67 0.5475
-56.66 0.5481
-56.65 0.5487
-56.65 0.5492
-56.64 0.5497
-56.58 0.5536
-56.54 0.5557
-56.54 0.5562
-56.53 0.5567
-56.5 0.5585
-56.48 0.5592
-56.47 0.5601
-56.46 0.5604
-56.45 0.5609
-56.42 0.5627
-56.41 0.5631
-56.4 0.5638
-56.34 0.5672
-56.29 0.5697
-56.28 0.5702
-56.27 0.5709
-56.26 0.5717
-56.18 0.5759
-56.14 0.5787
-56.13 0.5792
-56.11 0.58
-56.11 0.5804
-55.94 0.5892
-55.92 0.5904
-55.91 0.591
-55.89 0.5924
-55.88 0.5929
-55.85 0.5945
-55.83 0.5955
-55.7 0.6029
-55.63 0.6071
-55.59 0.6092
-55.58 0.6096
-55.57 0.61
-55.56 0.6105
-55.53 0.6119
-55.26 0.626
-55.22 0.6288
-55.21 0.6293
-55.19 0.6301
-55.18 0.631
-54.55 0.6644
-54.54 0.6649
-54.45 0.6691
-54.43 0.67
-54.4 0.6718
-54.39 0.6722
-54.38 0.6728
-54.37 0.6735
-54.35 0.6746
-54.34 0.6751
-54.33 0.6756
-54.28 0.6783
-54.18 0.6828
-54.17 0.6832
-54.16 0.6836
-54.12 0.6856
-54.11 0.6861
-54.1 0.6869
-54.08 0.6875
-54.07 0.6884
-54.06 0.6889
-54.02 0.6905
-54.01 0.691
-53.85 0.6988
-53.85 0.6993
-53.82 0.7007
-53.79 0.7023
-53.78 0.7028
-53.76 0.7035
-53.63 0.7102
-53.62 0.7106
-53.6 0.7117
-53.59 0.7123
-53.57 0.7131
-53.56 0.7137
-53.55 0.7143
-53.54 0.7145
-53.53 0.7154
-53.51 0.7163
-53.32 0.7257
-53.31 0.7263
-53.2 0.7315
-53.19 0.732
-53.18 0.7327
-53.17 0.7332
-53.16 0.7335
-53.15 0.7341
-53.13 0.7349
-53.11 0.7359
-53.06 0.738
-53.05 0.7388
-53.03 0.7398
-53.02 0.7418
-53.01 0.7425
-53.01 0.7437
-53 0.7452
-52.98 0.7489
-52.97 0.7503
-52.95 0.7546
-52.95 0.7561
-52.93 0.7588
-52.88 0.7687
-52.87 0.7705
-52.86 0.7715
-52.85 0.7745
-52.84 0.776
-52.83 0.7779
-52.83 0.7785
-52.82 0.7803
-52.81 0.782
-52.78 0.7884
-52.75 0.7941
-52.75 0.7948
-52.73 0.7969
-52.72 0.7987
-52.72 0.8002
-52.71 0.8011
-52.7 0.8021
-52.64 0.8125
-52.64 0.8133
-52.64 0.8137
-52.63 0.8153
-52.62 0.8166
-52.61 0.8177
-52.6 0.8199
-52.59 0.8208
-52.58 0.8223
-52.57 0.8243
-52.57 0.8249
-52.49 0.8354
-52.48 0.8365
-52.47 0.8378
-52.46 0.839
-52.46 0.8396
-52.38 0.8479
-52.37 0.849
-52.37 0.8499
-52.36 0.8505
-52.36 0.851
-52.3 0.8564
-52.28 0.8581
-52.27 0.8587
-52.27 0.8594
-52.26 0.8598
-52.25 0.8606
-52.24 0.862
-52.19 0.8678
-52.06 0.8735
-52.05 0.874
-52.03 0.8747
-52 0.8761
-51.99 0.8766
-51.96 0.8778
-51.85 0.8828
-51.84 0.883
-51.43 0.9
-51.41 0.9007
-51.39 0.9014
-51.38 0.9019
-51.35 0.903
-51.33 0.9035
-51.25 0.9066
-51.24 0.9073
-51.22 0.9077
-51.2 0.9087
-51.17 0.9096
-51.12 0.9116
-51.06 0.9134
-51.05 0.9138
-50.83 0.9212
-50.81 0.9217
-50.71 0.9243
-50.7 0.9249
-50.66 0.9257
-50.64 0.9262
-50.61 0.9273
-50.58 0.928
-50.54 0.9292
-50.5 0.9301
-50.44 0.9315
-50.42 0.9321
-50.38 0.9331
-50.35 0.9338
-50.34 0.9343
-50.31 0.9349
-50.3 0.9353
-50.28 0.9359
-50.26 0.9363
-50.24 0.9369
-50.23 0.9371
-50.21 0.9377
-50.16 0.9392
-50.14 0.9397
-49.83 0.9473
-49.8 0.9479
-49.76 0.9487
-49.72 0.9496
-49.69 0.9502
-49.64 0.9516
-49.61 0.9521
-49.48 0.9547
-49.46 0.9552
-49.4 0.9562
-49.34 0.9571
-49.19 0.9597
-49.18 0.9599
-49.12 0.9608
-49.09 0.9613
-49.01 0.9628
-48.95 0.9638
-48.92 0.9643
-48.73 0.9672
-48.7 0.9676
-48.65 0.9682
-48.59 0.9692
-48.54 0.9699
-48.48 0.9707
-48.47 0.9708
-48.44 0.9713
-48.29 0.9732
-48.25 0.9737
-48.16 0.9746
-48.01 0.9764
-47.97 0.9768
-47.88 0.9778
-47.82 0.9784
-47.78 0.9789
-47.36 0.9829
-47.31 0.9833
-47.24 0.9839
-47.21 0.9842
-47.12 0.9848
-47.06 0.9853
-46.96 0.9859
-46.92 0.9863
-46.85 0.9868
-46.74 0.9875
-46.64 0.9881
-46.52 0.9889
-46.32 0.99
-46.19 0.9905
-45.51 0.9937
-45.39 0.9943
-44.55 0.9975
-44.4 0.998
-44.37 0.9981
-44.14 0.9987
-43.86 0.9995
-43.04 1
-42.86 1
};
%\addlegendentry{outside color2 - color2}
\addplot [thick, forget plot, color2]
table {%
-49.66 0
-48.59 0.0005822
-48.55 0.0009316
-47.98 0.001747
-47.97 0.00198
-47.89 0.002679
-47.7 0.00361
-47.64 0.004076
-47.59 0.004542
-47.58 0.004775
-47.46 0.005357
-47.4 0.006056
-47.38 0.006405
-47.25 0.007686
-47.25 0.007802
-47.14 0.008385
-47.12 0.008734
-47.06 0.0092
-47.02 0.009666
-47.02 0.009782
-46.99 0.01036
-46.9 0.01165
-46.9 0.01176
-46.87 0.01234
-46.85 0.01269
-46.81 0.01351
-46.8 0.01374
-46.76 0.01444
-46.75 0.01467
-46.7 0.01572
-46.69 0.01665
-46.66 0.01712
-46.6 0.01782
-46.52 0.0198
-46.52 0.02003
-46.44 0.02166
-46.39 0.02236
-46.39 0.02259
-46.33 0.02352
-46.33 0.02376
-46.31 0.02446
-46.29 0.0248
-46.26 0.02539
-46.2 0.02713
-46.2 0.02725
-46.19 0.02772
-46.18 0.02807
-46.17 0.02888
-46.17 0.02911
-46.14 0.02993
-46.14 0.03005
-46.06 0.03191
-46.03 0.03237
-46.02 0.03307
-45.99 0.03366
-45.96 0.0347
-45.92 0.03529
-45.92 0.03552
-45.9 0.03622
-45.88 0.03692
-45.87 0.03738
-45.82 0.03901
-45.78 0.03983
-45.78 0.03994
-45.76 0.04053
-45.76 0.04099
-45.76 0.04146
-45.72 0.04216
-45.72 0.04239
-45.69 0.04332
-45.68 0.04344
-45.67 0.0439
-45.61 0.04647
-45.61 0.04693
-45.6 0.04751
-45.5 0.05077
-45.5 0.05101
-45.46 0.05287
-45.44 0.05438
-45.43 0.05462
-45.4 0.05532
-45.4 0.05543
-45.38 0.05613
-45.36 0.05776
-45.34 0.05834
-45.33 0.05904
-45.31 0.06044
-45.3 0.06102
-45.29 0.06137
-45.28 0.06207
-45.27 0.06289
-45.27 0.06335
-45.24 0.06405
-45.24 0.06417
-45.23 0.06475
-45.22 0.06568
-45.21 0.06661
-45.18 0.06778
-45.18 0.06801
-45.16 0.06859
-45.15 0.06941
-45.14 0.06999
-45.11 0.07209
-45.11 0.0722
-45.09 0.07325
-45.09 0.07348
-45.06 0.0743
-45.05 0.075
-45.03 0.07639
-45.02 0.07698
-45.02 0.07709
-45.01 0.07756
-45.01 0.07791
-44.99 0.07884
-44.98 0.07966
-44.97 0.08035
-44.96 0.0807
-44.92 0.08327
-44.91 0.08385
-44.9 0.08466
-44.9 0.0849
-44.89 0.08559
-44.88 0.08664
-44.87 0.08711
-44.87 0.08792
-44.87 0.08839
-44.86 0.0892
-44.85 0.09025
-44.84 0.09118
-44.83 0.09188
-44.83 0.0927
-44.83 0.09375
-44.83 0.09479
-44.82 0.09794
-44.82 0.09992
-44.82 0.1006
-44.82 0.1013
-44.81 0.1019
-44.81 0.1028
-44.81 0.1052
-44.81 0.1064
-44.81 0.1074
-44.81 0.1082
-44.81 0.1087
-44.81 0.1091
-44.8 0.1133
-44.8 0.1155
-44.8 0.1166
-44.8 0.1174
-44.8 0.1187
-44.8 0.1194
-44.8 0.1201
-44.8 0.1208
-44.8 0.1217
-44.79 0.1223
-44.79 0.1234
-44.79 0.1259
-44.79 0.1266
-44.79 0.1279
-44.79 0.1315
-44.79 0.1321
-44.78 0.1333
-44.78 0.1338
-44.78 0.1367
-44.78 0.1404
-44.78 0.141
-44.77 0.1431
-44.77 0.1453
-44.77 0.1464
-44.77 0.148
-44.77 0.1493
-44.77 0.15
-44.76 0.1535
-44.76 0.1598
-44.76 0.1607
-44.76 0.1622
-44.76 0.1627
-44.76 0.1633
-44.74 0.1757
-44.74 0.1768
-44.74 0.1782
-44.74 0.1798
-44.74 0.1812
-44.74 0.1831
-44.73 0.1871
-44.73 0.1919
-44.73 0.1953
-44.72 0.1984
-44.72 0.2009
-44.72 0.205
-44.72 0.2055
-44.72 0.2061
-44.71 0.21
-44.71 0.2111
-44.71 0.2122
-44.71 0.214
-44.71 0.2166
-44.71 0.2186
-44.7 0.2193
-44.7 0.2201
-44.7 0.2243
-44.7 0.2249
-44.7 0.2253
-44.7 0.2263
-44.69 0.2287
-44.69 0.2296
-44.69 0.2349
-44.69 0.2355
-44.69 0.2376
-44.68 0.2383
-44.68 0.2414
-44.68 0.2437
-44.68 0.246
-44.68 0.2465
-44.68 0.2474
-44.68 0.2486
-44.67 0.2506
-44.67 0.2514
-44.67 0.2519
-44.67 0.2546
-44.67 0.2556
-44.67 0.2566
-44.66 0.2655
-44.66 0.2674
-44.66 0.2702
-44.65 0.271
-44.65 0.2723
-44.65 0.2739
-44.65 0.2746
-44.65 0.2756
-44.65 0.2783
-44.65 0.2797
-44.64 0.2888
-44.64 0.2938
-44.63 0.3009
-44.62 0.3059
-44.62 0.3083
-44.62 0.3087
-44.62 0.3104
-44.62 0.3116
-44.62 0.3121
-44.62 0.3137
-44.62 0.3155
-44.61 0.3176
-44.61 0.3192
-44.61 0.3204
-44.61 0.3222
-44.61 0.3236
-44.61 0.326
-44.61 0.3276
-44.6 0.3282
-44.6 0.3293
-44.6 0.3305
-44.6 0.3313
-44.6 0.3321
-44.6 0.3343
-44.6 0.335
-44.59 0.3389
-44.59 0.3438
-44.59 0.3444
-44.59 0.3453
-44.59 0.348
-44.58 0.349
-44.58 0.3516
-44.58 0.3529
-44.57 0.3603
-44.57 0.3611
-44.57 0.3657
-44.57 0.3666
-44.57 0.3673
-44.57 0.3679
-44.57 0.3715
-44.57 0.3729
-44.56 0.3737
-44.56 0.3749
-44.56 0.3791
-44.56 0.38
-44.56 0.3856
-44.55 0.3873
-44.55 0.3879
-44.55 0.3892
-44.55 0.3898
-44.55 0.3919
-44.55 0.3928
-44.55 0.3993
-44.54 0.4003
-44.54 0.4035
-44.54 0.404
-44.54 0.4053
-44.53 0.4154
-44.53 0.4188
-44.53 0.4199
-44.53 0.4223
-44.53 0.4232
-44.52 0.4238
-44.52 0.4274
-44.52 0.4301
-44.52 0.4308
-44.51 0.4402
-44.51 0.4418
-44.5 0.4522
-44.5 0.4527
-44.5 0.4546
-44.5 0.456
-44.49 0.4584
-44.49 0.4588
-44.49 0.4599
-44.49 0.4603
-44.49 0.462
-44.49 0.4658
-44.49 0.4663
-44.49 0.4672
-44.49 0.4678
-44.49 0.469
-44.48 0.4709
-44.48 0.4719
-44.48 0.4726
-44.48 0.473
-44.48 0.4755
-44.47 0.4868
-44.47 0.4872
-44.47 0.4898
-44.47 0.4914
-44.46 0.4939
-44.46 0.4977
-44.46 0.4995
-44.46 0.5003
-44.46 0.5016
-44.46 0.5066
-44.45 0.5075
-44.45 0.5082
-44.45 0.5087
-44.45 0.5123
-44.45 0.5133
-44.45 0.5139
-44.44 0.5194
-44.44 0.52
-44.44 0.5207
-44.44 0.5215
-44.44 0.5224
-44.44 0.5239
-44.44 0.5246
-44.44 0.5259
-44.43 0.5306
-44.43 0.5312
-44.43 0.5319
-44.43 0.5356
-44.43 0.5384
-44.42 0.5406
-44.42 0.5416
-44.42 0.5434
-44.42 0.5491
-44.42 0.5498
-44.42 0.5528
-44.41 0.5542
-44.41 0.5553
-44.41 0.5567
-44.41 0.5574
-44.41 0.5585
-44.41 0.5606
-44.41 0.5618
-44.41 0.5633
-44.4 0.565
-44.4 0.5661
-44.4 0.5678
-44.4 0.5689
-44.4 0.5719
-44.4 0.573
-44.39 0.5767
-44.39 0.5784
-44.39 0.5795
-44.39 0.5804
-44.39 0.581
-44.39 0.5867
-44.39 0.5877
-44.38 0.5883
-44.38 0.5947
-44.38 0.5974
-44.38 0.5981
-44.38 0.5993
-44.37 0.6003
-44.37 0.6009
-44.37 0.6052
-44.37 0.606
-44.37 0.607
-44.37 0.6084
-44.36 0.6117
-44.36 0.614
-44.36 0.615
-44.36 0.6156
-44.36 0.617
-44.36 0.6177
-44.36 0.6194
-44.35 0.6228
-44.35 0.6244
-44.35 0.6258
-44.35 0.6265
-44.35 0.6304
-44.35 0.6308
-44.34 0.632
-44.34 0.6333
-44.34 0.636
-44.34 0.6368
-44.34 0.6397
-44.34 0.6403
-44.34 0.6412
-44.33 0.6436
-44.33 0.6445
-44.33 0.6452
-44.33 0.646
-44.33 0.6466
-44.33 0.647
-44.33 0.6484
-44.33 0.6492
-44.33 0.6506
-44.33 0.6531
-44.32 0.6542
-44.32 0.6556
-44.32 0.657
-44.32 0.6579
-44.32 0.6606
-44.32 0.6619
-44.32 0.6637
-44.31 0.666
-44.31 0.6672
-44.31 0.6705
-44.31 0.6737
-44.3 0.6764
-44.3 0.677
-44.3 0.6779
-44.3 0.6809
-44.3 0.6814
-44.3 0.6821
-44.29 0.6853
-44.29 0.687
-44.29 0.6874
-44.28 0.6941
-44.28 0.6951
-44.28 0.6962
-44.28 0.6968
-44.28 0.6983
-44.28 0.699
-44.27 0.7022
-44.27 0.7033
-44.27 0.7046
-44.27 0.7056
-44.27 0.7063
-44.27 0.7076
-44.26 0.708
-44.26 0.7113
-44.26 0.7145
-44.25 0.7171
-44.25 0.7177
-44.25 0.7227
-44.25 0.7235
-44.25 0.7242
-44.25 0.7249
-44.25 0.7254
-44.24 0.7264
-44.24 0.7277
-44.24 0.7283
-44.24 0.7301
-44.23 0.7334
-44.23 0.7339
-44.23 0.7348
-44.23 0.736
-44.23 0.7374
-44.23 0.7388
-44.23 0.7398
-44.22 0.741
-44.22 0.7436
-44.22 0.744
-44.22 0.7455
-44.22 0.7474
-44.21 0.7481
-44.21 0.7495
-44.21 0.7521
-44.21 0.7547
-44.2 0.7578
-44.2 0.7582
-44.2 0.7593
-44.2 0.7598
-44.2 0.7603
-44.2 0.7612
-44.19 0.7642
-44.19 0.7648
-44.19 0.7656
-44.19 0.7665
-44.19 0.7671
-44.19 0.7679
-44.19 0.7691
-44.18 0.7712
-44.18 0.7716
-44.18 0.7734
-44.18 0.774
-44.18 0.7744
-44.17 0.7757
-44.17 0.7762
-44.17 0.7768
-44.17 0.7798
-44.17 0.7804
-44.16 0.7827
-44.16 0.7833
-44.16 0.7846
-44.16 0.7851
-44.15 0.7864
-44.15 0.7872
-44.15 0.7884
-44.15 0.789
-44.15 0.7899
-44.15 0.7905
-44.14 0.7925
-44.14 0.7943
-44.14 0.795
-44.14 0.7956
-44.13 0.7995
-44.13 0.8018
-44.13 0.8034
-44.13 0.8039
-44.12 0.8051
-44.12 0.8065
-44.12 0.807
-44.12 0.8084
-44.12 0.8115
-44.11 0.8132
-44.11 0.8145
-44.11 0.8152
-44.11 0.8161
-44.1 0.817
-44.1 0.8179
-44.1 0.8183
-44.1 0.8189
-44.1 0.8208
-44.09 0.8217
-44.09 0.8222
-44.09 0.823
-44.09 0.8235
-44.09 0.826
-44.08 0.8267
-44.07 0.8323
-44.07 0.8329
-44.07 0.8335
-44.07 0.8344
-44.07 0.8351
-44.07 0.8356
-44.07 0.8364
-44.06 0.8373
-44.06 0.838
-44.06 0.8388
-44.06 0.8401
-44.05 0.8416
-44.05 0.8436
-44.05 0.8444
-44.05 0.8451
-44.04 0.8462
-44.04 0.847
-44.04 0.8478
-44.04 0.8485
-44.04 0.8494
-44.04 0.8499
-44.04 0.8505
-44.03 0.8514
-44.03 0.852
-44.03 0.8527
-44.03 0.8533
-44.03 0.8538
-44.03 0.8548
-44.03 0.8564
-44.02 0.8596
-44.02 0.8608
-44.01 0.8623
-44.01 0.8637
-44.01 0.8644
-44.01 0.8651
-44.01 0.8655
-44.01 0.866
-44 0.8685
-44 0.8692
-44 0.8702
-44 0.8711
-44 0.8722
-43.99 0.8728
-43.99 0.8735
-43.98 0.8759
-43.97 0.8763
-43.97 0.8767
-43.96 0.8771
-43.96 0.8776
-43.94 0.8783
-43.94 0.8785
-43.92 0.8796
-43.92 0.8799
-43.91 0.8805
-43.86 0.8848
-43.85 0.8858
-43.84 0.8862
-43.79 0.8908
-43.79 0.8915
-43.79 0.8919
-43.78 0.8926
-43.78 0.8929
-43.77 0.8936
-43.76 0.895
-43.76 0.8954
-43.75 0.8959
-43.75 0.8964
-43.74 0.8967
-43.73 0.8975
-43.73 0.8979
-43.72 0.8983
-43.71 0.8988
-43.71 0.8998
-43.7 0.9004
-43.65 0.9047
-43.65 0.9051
-43.64 0.906
-43.64 0.907
-43.64 0.9074
-43.63 0.9081
-43.63 0.9087
-43.62 0.9102
-43.61 0.9109
-43.59 0.9123
-43.59 0.9128
-43.59 0.913
-43.58 0.9135
-43.57 0.915
-43.56 0.9153
-43.55 0.9163
-43.55 0.9166
-43.55 0.9171
-43.54 0.9187
-43.53 0.9194
-43.52 0.9205
-43.51 0.9213
-43.49 0.9224
-43.48 0.9237
-43.48 0.9242
-43.45 0.9263
-43.44 0.9267
-43.44 0.9269
-43.42 0.928
-43.38 0.9308
-43.38 0.9314
-43.37 0.9319
-43.37 0.9327
-43.36 0.9333
-43.36 0.9336
-43.34 0.9351
-43.32 0.937
-43.32 0.9375
-43.31 0.9384
-43.3 0.9389
-43.29 0.9396
-43.29 0.9401
-43.28 0.9408
-43.27 0.9415
-43.27 0.9421
-43.26 0.9427
-43.26 0.9431
-43.24 0.9438
-43.24 0.9443
-43.22 0.946
-43.22 0.9463
-43.21 0.947
-43.21 0.9479
-43.13 0.9519
-43.12 0.9527
-43.11 0.9535
-43.1 0.9537
-43.09 0.9545
-43.09 0.9549
-43.08 0.9556
-43.08 0.956
-43.04 0.958
-43.03 0.9584
-43.02 0.9589
-43.02 0.9594
-43.02 0.9598
-43 0.9609
-42.98 0.962
-42.96 0.963
-42.96 0.9633
-42.95 0.9639
-42.92 0.9653
-42.91 0.9659
-42.88 0.9682
-42.87 0.9687
-42.86 0.9693
-42.86 0.9696
-42.83 0.9702
-42.83 0.9707
-42.82 0.9711
-42.82 0.9715
-42.81 0.9721
-42.8 0.973
-42.8 0.9733
-42.79 0.9739
-42.77 0.9748
-42.75 0.9755
-42.72 0.9769
-42.71 0.9776
-42.69 0.9782
-42.68 0.9785
-42.67 0.9793
-42.66 0.9799
-42.65 0.9804
-42.65 0.9807
-42.63 0.9813
-42.61 0.9819
-42.6 0.9829
-42.6 0.9833
-42.59 0.9838
-42.59 0.9839
-42.57 0.9845
-42.56 0.9846
-42.53 0.9857
-42.52 0.9865
-42.5 0.9871
-42.45 0.9882
-42.44 0.9886
-42.41 0.9896
-42.37 0.9908
-42.34 0.9913
-42.34 0.9914
-42.31 0.9918
-42.19 0.9946
-42.19 0.9948
-42.14 0.9955
-42 0.9959
-41.99 0.996
-41.93 0.9965
-41.92 0.9966
-41.37 0.9999
};
%\addlegendentry{inside color2 - color2}
\addplot [thick, forget plot, color3, dashed]
table {%
-68.43 0
-68.38 8.571e-05
-67.17 0.0006855
-66.22 0.002656
-66.11 0.003256
-65.81 0.003941
-65.44 0.006255
-65.38 0.007026
-65.37 0.007112
-65.02 0.01028
-64.93 0.0108
-64.9 0.01105
-64.79 0.01242
-64.79 0.0126
-64.66 0.01345
-64.57 0.01499
-64.56 0.01525
-64.5 0.01585
-64.5 0.01594
-64.45 0.01688
-64.37 0.01808
-64.34 0.01911
-64.29 0.01979
-64.28 0.02005
-64.18 0.02176
-64.18 0.02202
-64.15 0.02279
-64.12 0.02348
-64.07 0.02399
-64.02 0.02545
-64 0.02579
-63.96 0.0269
-63.93 0.02768
-63.87 0.02836
-63.87 0.02862
-63.65 0.03384
-63.65 0.03402
-63.52 0.03736
-63.52 0.03761
-63.45 0.03984
-63.41 0.04036
-63.41 0.04053
-63.38 0.04147
-63.37 0.0419
-63.35 0.04276
-63.34 0.04318
-63.29 0.04396
-63.28 0.04473
-63.28 0.04524
-63.27 0.04558
-63.25 0.04627
-63.14 0.05124
-63.12 0.05175
-63.11 0.05184
-63.1 0.05235
-63.07 0.05312
-63.05 0.05381
-63.05 0.05389
-63.01 0.05484
-63 0.05509
-62.99 0.05569
-62.98 0.05655
-62.96 0.05749
-62.9 0.05998
-62.87 0.06092
-62.81 0.06332
-62.8 0.06452
-62.77 0.06606
-62.71 0.06992
-62.69 0.07052
-62.68 0.07154
-62.62 0.07454
-62.61 0.07506
-62.6 0.07583
-62.6 0.07626
-62.6 0.07651
-62.56 0.07908
-62.54 0.07968
-62.53 0.0808
-62.52 0.08114
-62.49 0.0826
-62.49 0.08285
-62.48 0.08405
-62.44 0.08534
-62.44 0.0856
-62.43 0.0862
-62.41 0.08842
-62.38 0.09057
-62.37 0.09117
-62.36 0.09202
-62.36 0.09219
-62.35 0.09279
-62.34 0.09382
-62.32 0.09468
-62.32 0.09511
-62.3 0.09622
-62.3 0.09674
-62.27 0.09896
-62.26 0.09956
-62.25 0.09982
-62.24 0.1003
-62.23 0.1013
-62.22 0.102
-62.21 0.1026
-62.2 0.1032
-62.17 0.1049
-62.16 0.1055
-62.13 0.1073
-62.12 0.1076
-62.11 0.1082
-62.1 0.1088
-62.1 0.1094
-62.08 0.1103
-62.07 0.111
-62.06 0.1116
-62.06 0.1118
-62.05 0.1124
-62.05 0.1125
-62.03 0.1132
-62.02 0.1141
-62.01 0.1147
-61.95 0.1189
-61.95 0.1193
-61.93 0.1201
-61.92 0.1214
-61.88 0.1236
-61.88 0.1242
-61.86 0.1259
-61.86 0.1261
-61.85 0.1266
-61.85 0.1268
-61.85 0.1273
-61.84 0.1275
-61.83 0.1286
-61.83 0.129
-61.78 0.1329
-61.78 0.1338
-61.77 0.1347
-61.75 0.1358
-61.73 0.1379
-61.73 0.1383
-61.71 0.139
-61.71 0.1395
-61.7 0.1399
-61.7 0.1402
-61.69 0.1407
-61.68 0.1415
-61.67 0.1421
-61.67 0.1427
-61.65 0.1433
-61.55 0.1511
-61.53 0.1535
-61.53 0.1539
-61.51 0.156
-61.51 0.1562
-61.5 0.1572
-61.49 0.1583
-61.45 0.1615
-61.37 0.1683
-61.32 0.1728
-61.32 0.1733
-61.31 0.174
-61.3 0.1756
-61.29 0.1762
-61.28 0.1768
-61.28 0.1774
-61.26 0.1787
-61.25 0.1797
-61.25 0.1804
-61.22 0.1829
-61.2 0.1862
-61.17 0.1888
-61.16 0.1893
-61.15 0.1898
-61.14 0.1906
-61.13 0.1926
-61.11 0.1936
-61.1 0.1959
-61.08 0.197
-61.08 0.1973
-61.06 0.1986
-61.06 0.199
-61.05 0.1996
-60.99 0.2061
-60.97 0.2067
-60.97 0.2075
-60.94 0.2104
-60.93 0.2111
-60.93 0.2117
-60.92 0.2124
-60.92 0.2129
-60.91 0.2138
-60.9 0.2146
-60.9 0.2151
-60.9 0.2157
-60.9 0.2158
-60.89 0.2169
-60.87 0.2182
-60.87 0.2187
-60.86 0.2198
-60.85 0.2205
-60.84 0.2226
-60.83 0.2231
-60.83 0.2238
-60.82 0.2247
-60.82 0.2253
-60.81 0.226
-60.8 0.2271
-60.79 0.2279
-60.78 0.2291
-60.78 0.2294
-60.77 0.23
-60.77 0.2302
-60.76 0.231
-60.76 0.2319
-60.75 0.2325
-60.74 0.2335
-60.74 0.2341
-60.73 0.2351
-60.72 0.2356
-60.72 0.2367
-60.71 0.2371
-60.69 0.2391
-60.69 0.2397
-60.69 0.2403
-60.68 0.2413
-60.67 0.2418
-60.67 0.242
-60.66 0.2427
-60.64 0.2465
-60.63 0.2475
-60.62 0.2492
-60.6 0.251
-60.6 0.2514
-60.59 0.2522
-60.59 0.2525
-60.59 0.253
-60.58 0.254
-60.58 0.255
-60.57 0.2556
-60.57 0.2564
-60.56 0.2576
-60.55 0.2581
-60.55 0.2583
-60.54 0.26
-60.54 0.2602
-60.53 0.2616
-60.52 0.2618
-60.52 0.2624
-60.52 0.2628
-60.51 0.2634
-60.51 0.2636
-60.51 0.2642
-60.5 0.2649
-60.5 0.2656
-60.49 0.2666
-60.47 0.2682
-60.47 0.2687
-60.44 0.2724
-60.43 0.273
-60.43 0.2738
-60.43 0.2744
-60.43 0.275
-60.42 0.2758
-60.41 0.2763
-60.4 0.2778
-60.4 0.2786
-60.39 0.2793
-60.39 0.28
-60.38 0.2806
-60.37 0.2825
-60.37 0.283
-60.34 0.2852
-60.33 0.2864
-60.33 0.2876
-60.32 0.2887
-60.32 0.2893
-60.28 0.2932
-60.27 0.2948
-60.25 0.2982
-60.24 0.2988
-60.24 0.2989
-60.24 0.2994
-60.23 0.2998
-60.23 0.3001
-60.23 0.3007
-60.23 0.3009
-60.21 0.3015
-60.21 0.3018
-60.2 0.3028
-60.2 0.3038
-60.2 0.3043
-60.19 0.3055
-60.18 0.3062
-60.18 0.3068
-60.18 0.3074
-60.17 0.308
-60.17 0.3088
-60.16 0.3096
-60.14 0.3119
-60.14 0.3126
-60.13 0.3139
-60.13 0.3145
-60.12 0.3152
-60.12 0.3168
-60.11 0.3173
-60.09 0.319
-60.09 0.3195
-60.09 0.3198
-60.08 0.3211
-60.08 0.3213
-60.07 0.3218
-60.07 0.3223
-60.06 0.323
-60.03 0.327
-60.03 0.3277
-60.01 0.3294
-60 0.3315
-60 0.3319
-59.99 0.3326
-59.99 0.3328
-59.98 0.3333
-59.97 0.3357
-59.97 0.3363
-59.97 0.3368
-59.96 0.3375
-59.96 0.339
-59.94 0.3403
-59.94 0.3411
-59.94 0.3417
-59.93 0.3422
-59.93 0.3429
-59.93 0.3435
-59.92 0.3439
-59.92 0.3445
-59.91 0.3457
-59.91 0.3461
-59.91 0.3465
-59.91 0.3473
-59.9 0.3479
-59.9 0.348
-59.89 0.3486
-59.88 0.3509
-59.88 0.3512
-59.88 0.3519
-59.87 0.353
-59.87 0.3532
-59.86 0.3537
-59.83 0.3586
-59.83 0.3588
-59.83 0.3594
-59.82 0.3606
-59.82 0.3609
-59.81 0.3614
-59.81 0.3621
-59.81 0.3624
-59.8 0.363
-59.79 0.366
-59.78 0.3665
-59.76 0.3708
-59.76 0.3711
-59.75 0.3719
-59.75 0.373
-59.74 0.3735
-59.74 0.3741
-59.73 0.3746
-59.73 0.3749
-59.73 0.3756
-59.73 0.3765
-59.72 0.377
-59.71 0.3784
-59.71 0.379
-59.69 0.3821
-59.68 0.3828
-59.68 0.3839
-59.67 0.3845
-59.66 0.3854
-59.65 0.3878
-59.64 0.3887
-59.64 0.3897
-59.62 0.3935
-59.61 0.3943
-59.61 0.395
-59.59 0.398
-59.59 0.3984
-59.57 0.4006
-59.57 0.4008
-59.57 0.4013
-59.56 0.4028
-59.55 0.4036
-59.55 0.4042
-59.55 0.4048
-59.54 0.405
-59.53 0.4062
-59.53 0.4068
-59.52 0.4076
-59.52 0.4085
-59.51 0.4096
-59.5 0.4103
-59.48 0.4138
-59.47 0.4147
-59.47 0.4152
-59.46 0.4171
-59.45 0.4177
-59.44 0.4192
-59.44 0.4204
-59.43 0.4208
-59.43 0.4217
-59.43 0.4222
-59.42 0.4228
-59.42 0.4238
-59.4 0.4271
-59.4 0.4275
-59.39 0.4284
-59.39 0.4288
-59.37 0.4305
-59.37 0.4312
-59.37 0.4317
-59.36 0.4324
-59.36 0.4332
-59.35 0.4338
-59.35 0.4349
-59.34 0.4357
-59.32 0.4388
-59.32 0.4392
-59.32 0.4396
-59.31 0.4401
-59.31 0.4413
-59.28 0.4456
-59.27 0.4468
-59.27 0.4473
-59.27 0.4482
-59.27 0.4486
-59.26 0.4492
-59.26 0.4501
-59.26 0.4505
-59.24 0.453
-59.23 0.454
-59.23 0.4548
-59.23 0.4554
-59.22 0.4566
-59.22 0.4571
-59.22 0.4577
-59.2 0.46
-59.2 0.4609
-59.2 0.4613
-59.19 0.4622
-59.16 0.4677
-59.16 0.4681
-59.15 0.4685
-59.15 0.4691
-59.15 0.4695
-59.14 0.4702
-59.14 0.4706
-59.13 0.4713
-59.13 0.472
-59.13 0.4728
-59.13 0.473
-59.12 0.4734
-59.11 0.4752
-59.1 0.4775
-59.09 0.478
-59.09 0.4785
-59.08 0.4798
-59.08 0.4803
-59.07 0.4809
-59.07 0.4819
-59.07 0.4826
-59.06 0.4837
-59.03 0.4879
-59.03 0.4882
-59.03 0.4888
-59.01 0.4917
-59 0.4923
-58.99 0.4933
-58.99 0.4934
-58.99 0.494
-58.98 0.4948
-58.98 0.4954
-58.97 0.4963
-58.96 0.4987
-58.96 0.4994
-58.95 0.5002
-58.94 0.5008
-58.94 0.5017
-58.93 0.5023
-58.93 0.503
-58.93 0.5035
-58.92 0.5046
-58.92 0.505
-58.91 0.5064
-58.91 0.5067
-58.9 0.5089
-58.89 0.5102
-58.88 0.5114
-58.87 0.5122
-58.87 0.5128
-58.86 0.5146
-58.86 0.5149
-58.85 0.5154
-58.85 0.5161
-58.85 0.5162
-58.84 0.5168
-58.84 0.5174
-58.84 0.5179
-58.83 0.5184
-58.81 0.5208
-58.8 0.5216
-58.8 0.5224
-58.79 0.5229
-58.79 0.5242
-58.78 0.5259
-58.77 0.5267
-58.77 0.5274
-58.76 0.5293
-58.75 0.5314
-58.74 0.5321
-58.73 0.5341
-58.73 0.5347
-58.72 0.5355
-58.71 0.5369
-58.71 0.5382
-58.71 0.539
-58.7 0.5399
-58.7 0.5405
-58.69 0.541
-58.69 0.5417
-58.69 0.542
-58.68 0.5426
-58.64 0.5494
-58.64 0.5497
-58.62 0.5522
-58.62 0.5527
-58.61 0.5533
-58.59 0.5557
-58.58 0.559
-58.57 0.5597
-58.57 0.5604
-58.57 0.5614
-58.56 0.5618
-58.55 0.5634
-58.55 0.5637
-58.55 0.5643
-58.55 0.5645
-58.55 0.5649
-58.54 0.5654
-58.54 0.566
-58.52 0.5693
-58.51 0.5698
-58.51 0.5704
-58.51 0.5707
-58.5 0.5713
-58.46 0.5793
-58.45 0.5798
-58.45 0.5802
-58.45 0.5812
-58.44 0.5818
-58.43 0.5835
-58.43 0.5838
-58.42 0.5844
-58.4 0.5869
-58.4 0.588
-58.39 0.5885
-58.38 0.5904
-58.38 0.5904
-58.38 0.5909
-58.37 0.5912
-58.37 0.5922
-58.37 0.5927
-58.36 0.5933
-58.34 0.5974
-58.34 0.5979
-58.33 0.5994
-58.33 0.5998
-58.33 0.6003
-58.33 0.6007
-58.32 0.6013
-58.32 0.6017
-58.32 0.6023
-58.3 0.6057
-58.28 0.6085
-58.28 0.6091
-58.28 0.6099
-58.27 0.6107
-58.26 0.6133
-58.25 0.6138
-58.24 0.616
-58.23 0.6164
-58.23 0.6169
-58.23 0.6175
-58.22 0.6181
-58.22 0.6186
-58.21 0.6195
-58.21 0.6201
-58.21 0.6206
-58.19 0.6227
-58.19 0.6233
-58.19 0.6234
-58.18 0.6239
-58.17 0.6271
-58.16 0.6277
-58.16 0.6286
-58.16 0.629
-58.15 0.6297
-58.14 0.6316
-58.11 0.6368
-58.11 0.6373
-58.1 0.6381
-58.09 0.6393
-58.08 0.6406
-58.08 0.6418
-58.07 0.643
-58.06 0.6449
-58.06 0.6455
-58.05 0.6463
-58.04 0.6471
-58.04 0.6476
-58.03 0.6482
-58.03 0.6488
-58.03 0.6491
-58.03 0.6496
-58.02 0.6503
-58.02 0.6509
-58.02 0.6514
-58 0.6539
-57.99 0.6551
-57.98 0.6558
-57.96 0.6599
-57.96 0.6603
-57.96 0.6609
-57.95 0.6617
-57.95 0.6622
-57.94 0.6636
-57.93 0.6646
-57.93 0.6655
-57.92 0.6662
-57.91 0.6685
-57.9 0.6694
-57.9 0.6699
-57.89 0.6717
-57.89 0.6719
-57.89 0.6724
-57.88 0.673
-57.88 0.6744
-57.87 0.6746
-57.87 0.6751
-57.87 0.6756
-57.86 0.6765
-57.84 0.6799
-57.84 0.6803
-57.83 0.6812
-57.83 0.6817
-57.82 0.6841
-57.82 0.6843
-57.81 0.6849
-57.81 0.685
-57.8 0.6855
-57.79 0.6879
-57.79 0.6886
-57.78 0.6891
-57.78 0.6896
-57.77 0.6903
-57.74 0.6945
-57.73 0.6951
-57.73 0.6957
-57.73 0.6966
-57.72 0.6978
-57.71 0.6986
-57.71 0.6996
-57.7 0.7014
-57.69 0.7026
-57.68 0.7031
-57.68 0.7037
-57.68 0.7043
-57.66 0.705
-57.66 0.7058
-57.66 0.7061
-57.65 0.7067
-57.64 0.7087
-57.63 0.7095
-57.62 0.7106
-57.62 0.7113
-57.61 0.7122
-57.6 0.7139
-57.6 0.7147
-57.59 0.7152
-57.58 0.7165
-57.58 0.7171
-57.57 0.7184
-57.56 0.7191
-57.56 0.7202
-57.55 0.7206
-57.55 0.7212
-57.54 0.7229
-57.54 0.7238
-57.53 0.7243
-57.53 0.7252
-57.51 0.7278
-57.51 0.728
-57.51 0.7288
-57.5 0.73
-57.5 0.7305
-57.49 0.7316
-57.46 0.7359
-57.45 0.7372
-57.45 0.7379
-57.44 0.7385
-57.44 0.7392
-57.44 0.7396
-57.43 0.7407
-57.4 0.745
-57.39 0.7456
-57.38 0.7468
-57.38 0.7474
-57.38 0.7479
-57.37 0.749
-57.33 0.7534
-57.32 0.7542
-57.32 0.7544
-57.31 0.7552
-57.31 0.7561
-57.3 0.7565
-57.3 0.7568
-57.29 0.7576
-57.29 0.7581
-57.28 0.7593
-57.27 0.7602
-57.27 0.7605
-57.26 0.7613
-57.25 0.7631
-57.25 0.7639
-57.23 0.7661
-57.23 0.767
-57.22 0.7675
-57.22 0.7682
-57.22 0.7687
-57.21 0.7694
-57.21 0.7698
-57.19 0.7711
-57.18 0.7729
-57.17 0.7739
-57.16 0.7745
-57.16 0.7747
-57.16 0.7753
-57.15 0.7764
-57.14 0.7771
-57.13 0.7783
-57.13 0.7788
-57.12 0.7799
-57.11 0.7805
-57.1 0.7811
-57.1 0.7818
-57.09 0.7824
-57.08 0.784
-57.07 0.7847
-57.07 0.7851
-57.07 0.7857
-57.06 0.7861
-57.06 0.7867
-57.05 0.7878
-57.05 0.7885
-57.04 0.7893
-57.02 0.7908
-57.01 0.7914
-57 0.7929
-57 0.7933
-57 0.7937
-56.99 0.7947
-56.98 0.7958
-56.97 0.7965
-56.97 0.7969
-56.96 0.7976
-56.96 0.798
-56.95 0.7986
-56.93 0.8007
-56.92 0.8013
-56.92 0.8018
-56.92 0.8021
-56.9 0.8037
-56.89 0.8045
-56.88 0.805
-56.87 0.8059
-56.86 0.8077
-56.85 0.8082
-56.84 0.809
-56.84 0.8094
-56.84 0.81
-56.83 0.8115
-56.82 0.812
-56.81 0.8139
-56.8 0.8147
-56.79 0.8153
-56.79 0.8159
-56.78 0.8164
-56.78 0.8166
-56.76 0.8185
-56.76 0.8189
-56.75 0.8198
-56.74 0.8208
-56.74 0.8216
-56.73 0.823
-56.72 0.8239
-56.71 0.8259
-56.7 0.8264
-56.68 0.8284
-56.68 0.8287
-56.67 0.8298
-56.65 0.8322
-56.65 0.833
-56.64 0.8336
-56.63 0.8347
-56.61 0.8358
-56.61 0.8365
-56.59 0.8381
-56.58 0.8393
-56.57 0.8398
-56.57 0.8405
-56.55 0.8413
-56.55 0.8415
-56.54 0.842
-56.54 0.8422
-56.53 0.843
-56.52 0.8444
-56.5 0.8457
-56.5 0.8459
-56.49 0.8471
-56.48 0.8492
-56.47 0.8497
-56.46 0.8505
-56.43 0.8544
-56.42 0.855
-56.41 0.8567
-56.4 0.8573
-56.39 0.8592
-56.37 0.8601
-56.37 0.8604
-56.36 0.861
-56.36 0.8616
-56.35 0.8621
-56.35 0.8628
-56.34 0.8639
-56.32 0.8645
-56.31 0.8656
-56.3 0.8667
-56.29 0.8673
-56.29 0.8675
-56.27 0.8692
-56.26 0.8701
-56.25 0.8708
-56.24 0.8712
-56.22 0.8726
-56.22 0.873
-56.22 0.8736
-56.21 0.8739
-56.21 0.8742
-56.2 0.8748
-56.16 0.8796
-56.15 0.8802
-56.14 0.8817
-56.14 0.8819
-56.12 0.8828
-56.12 0.8833
-56.11 0.8839
-56.11 0.8842
-56.09 0.8853
-56.06 0.8874
-56.03 0.8903
-56.02 0.8909
-56.01 0.8924
-55.98 0.8951
-55.97 0.8956
-55.97 0.896
-55.96 0.8967
-55.96 0.8972
-55.95 0.898
-55.95 0.8982
-55.92 0.8997
-55.9 0.9009
-55.9 0.9012
-55.89 0.9019
-55.88 0.9025
-55.87 0.903
-55.87 0.9032
-55.86 0.9037
-55.86 0.9041
-55.83 0.9055
-55.83 0.9059
-55.82 0.9067
-55.76 0.9122
-55.75 0.9128
-55.74 0.9141
-55.72 0.9148
-55.72 0.9152
-55.69 0.9165
-55.69 0.9169
-55.68 0.9176
-55.65 0.9191
-55.63 0.9213
-55.63 0.9215
-55.57 0.9239
-55.56 0.9245
-55.55 0.9253
-55.55 0.9257
-55.54 0.9262
-55.54 0.9265
-55.53 0.927
-55.5 0.9285
-55.49 0.929
-55.49 0.9291
-55.47 0.9299
-55.42 0.9329
-55.41 0.9333
-55.39 0.9343
-55.38 0.9348
-55.37 0.9352
-55.36 0.9358
-55.35 0.9363
-55.35 0.9368
-55.35 0.9369
-55.34 0.9375
-55.33 0.9383
-55.32 0.939
-55.3 0.9394
-55.29 0.9395
-55.28 0.9402
-55.27 0.9405
-55.26 0.9411
-55.26 0.9414
-55.26 0.942
-55.25 0.9425
-55.25 0.9426
-55.23 0.9431
-55.23 0.9434
-55.22 0.9439
-55.22 0.9441
-55.2 0.945
-55.2 0.9456
-55.09 0.9503
-55.08 0.9515
-55.07 0.9518
-55.06 0.9525
-54.99 0.9561
-54.98 0.9566
-54.97 0.9572
-54.97 0.9575
-54.94 0.9584
-54.94 0.9589
-54.93 0.9594
-54.93 0.9596
-54.91 0.9602
-54.91 0.9605
-54.88 0.9612
-54.87 0.9614
-54.81 0.9632
-54.78 0.9641
-54.74 0.9647
-54.74 0.965
-54.72 0.9655
-54.72 0.9656
-54.68 0.9663
-54.67 0.9668
-54.65 0.9677
-54.65 0.9678
-54.63 0.9683
-54.63 0.9686
-54.61 0.9691
-54.61 0.9696
-54.56 0.9708
-54.56 0.971
-54.54 0.9715
-54.54 0.9716
-54.49 0.9728
-54.48 0.9736
-54.41 0.9752
-54.41 0.9752
-54.38 0.9758
-54.38 0.9762
-54.36 0.9767
-54.33 0.9772
-54.12 0.9813
-54.11 0.9818
-54.07 0.9825
-54.05 0.9829
-54.02 0.9836
-53.98 0.9845
-53.93 0.9856
-53.93 0.9858
-53.9 0.9863
-53.69 0.9894
-53.69 0.9895
-53.69 0.9895
-53.62 0.9901
-53.58 0.9907
-53.58 0.991
-53.18 0.9948
-53.04 0.9954
-53.03 0.9955
-52.87 0.996
-52.85 0.9961
-52.69 0.9969
-52.67 0.9972
-52.33 0.9988
-52.01 0.9994
-51.98 0.9996
-51.73 0.9998
-51.56 0.9999
};
%\addlegendentry{outside color2 - color}
\addplot [thick, forget plot, color3]
table {%
-68.94 0
-65.73 0.003788
-65.31 0.007576
-64.52 0.01136
-64.46 0.01515
-63.32 0.01894
-63.13 0.02273
-63.06 0.02652
-63.05 0.0303
-62.81 0.03409
-62.78 0.03788
-62.76 0.04167
-62.73 0.04545
-62.51 0.04924
-62.51 0.05303
-62.43 0.05682
-62.38 0.06061
-62.37 0.06439
-62.37 0.06818
-62.23 0.07197
-62.2 0.07576
-62.18 0.07955
-62.1 0.08333
-62.09 0.08712
-61.91 0.09091
-61.77 0.0947
-61.76 0.09848
-61.76 0.1023
-61.76 0.1061
-61.75 0.1098
-61.7 0.1136
-61.56 0.1174
-61.4 0.125
-61.4 0.1288
-61.38 0.1326
-61.34 0.1364
-61.34 0.1402
-61.34 0.1439
-61.22 0.1477
-61.21 0.1515
-61.07 0.1553
-61.06 0.1591
-61.06 0.1629
-61.05 0.1667
-60.97 0.1705
-60.95 0.1742
-60.84 0.1818
-60.76 0.1856
-60.67 0.1932
-60.66 0.197
-60.62 0.2008
-60.6 0.2045
-60.57 0.2083
-60.5 0.2121
-60.44 0.2159
-60.43 0.2197
-60.36 0.2273
-60.36 0.2311
-60.27 0.2348
-60.21 0.2386
-60.14 0.2424
-60.14 0.2462
-60.1 0.25
-60.09 0.2538
-60.06 0.2576
-60.04 0.2614
-60 0.2652
-59.96 0.2689
-59.95 0.2727
-59.92 0.2765
-59.9 0.2803
-59.88 0.2841
-59.87 0.2879
-59.83 0.2917
-59.82 0.2955
-59.77 0.2992
-59.74 0.303
-59.71 0.3068
-59.65 0.3106
-59.65 0.3144
-59.6 0.3182
-59.57 0.322
-59.54 0.3258
-59.54 0.3295
-59.5 0.3333
-59.45 0.3371
-59.42 0.3447
-59.33 0.3485
-59.29 0.3523
-59.23 0.3598
-59.2 0.3636
-59.16 0.3674
-59.16 0.375
-59.15 0.3788
-59.13 0.3826
-59.11 0.3864
-59.03 0.3902
-59.01 0.3939
-59 0.3977
-58.93 0.4053
-58.93 0.4091
-58.9 0.4129
-58.89 0.4167
-58.86 0.4242
-58.8 0.428
-58.78 0.4356
-58.78 0.4394
-58.74 0.4432
-58.74 0.447
-58.73 0.4508
-58.7 0.4545
-58.66 0.4583
-58.56 0.4621
-58.53 0.4659
-58.52 0.4697
-58.5 0.4735
-58.45 0.4773
-58.45 0.4811
-58.45 0.4848
-58.37 0.4886
-58.37 0.4924
-58.36 0.4962
-58.36 0.5
-58.3 0.5038
-58.26 0.5076
-58.25 0.5114
-58.13 0.5152
-58.12 0.5189
-58.11 0.5227
-58.09 0.5265
-58.07 0.5303
-58.03 0.5341
-57.99 0.5379
-57.99 0.5417
-57.98 0.5455
-57.98 0.5492
-57.96 0.553
-57.96 0.5568
-57.94 0.5606
-57.9 0.5644
-57.86 0.5682
-57.85 0.572
-57.84 0.5758
-57.82 0.5795
-57.81 0.5833
-57.76 0.5871
-57.73 0.5909
-57.73 0.5947
-57.71 0.5985
-57.71 0.6023
-57.69 0.6061
-57.69 0.6098
-57.69 0.6136
-57.68 0.6174
-57.67 0.6212
-57.64 0.625
-57.63 0.6288
-57.52 0.6326
-57.43 0.6402
-57.38 0.6439
-57.37 0.6477
-57.3 0.6515
-57.24 0.6591
-57.18 0.6629
-57.17 0.6667
-57.16 0.6705
-57.14 0.6742
-57.14 0.678
-57.13 0.6818
-57.12 0.6856
-57.12 0.6894
-57.11 0.6932
-57.07 0.697
-57.02 0.7008
-57.02 0.7045
-57 0.7121
-56.99 0.7159
-56.92 0.7197
-56.9 0.7235
-56.83 0.7273
-56.83 0.7311
-56.8 0.7348
-56.79 0.7386
-56.79 0.7424
-56.77 0.7462
-56.77 0.75
-56.77 0.7538
-56.75 0.7576
-56.72 0.7614
-56.51 0.7652
-56.51 0.7689
-56.49 0.7727
-56.47 0.7765
-56.37 0.7803
-56.36 0.7841
-56.35 0.7879
-56.32 0.7917
-56.3 0.7955
-56.29 0.7992
-56.24 0.803
-56.24 0.8068
-56.23 0.8106
-56.2 0.8144
-56.17 0.822
-56.17 0.8258
-56.14 0.8295
-56.13 0.8371
-56.11 0.8409
-56.11 0.8447
-56.1 0.8485
-56.07 0.8523
-56.06 0.8561
-55.99 0.8598
-55.98 0.8636
-55.96 0.8674
-55.89 0.8712
-55.78 0.875
-55.75 0.8788
-55.74 0.8826
-55.68 0.8864
-55.56 0.8902
-55.5 0.8939
-55.48 0.8977
-55.34 0.9015
-55.32 0.9053
-55.29 0.9091
-55.28 0.9129
-55.17 0.9167
-55.15 0.9205
-55.06 0.9242
-55 0.928
-54.98 0.9318
-54.94 0.9356
-54.93 0.9394
-54.84 0.9432
-54.38 0.947
-54.35 0.9508
-54.34 0.9545
-54.13 0.9583
-54.1 0.9621
-53.99 0.9659
-53.8 0.9735
-53.62 0.9773
-53.51 0.9811
-53.5 0.9848
-53.01 0.9886
-52.61 0.9924
-51.94 0.9962

};
%\addlegendentry{inside color2 - color}
\addplot [thick, forget plot, color4, dashed]
table {%

-74.01 0
-74.01 0.0002328
-74 0.0004656
-72.88 0.00163
-72.87 0.001863
-72.83 0.002095
-72.71 0.002561
-72.6 0.003492
-72.27 0.004424
-72.26 0.004657
-71.85 0.006985
-71.83 0.007451
-71.51 0.00908
-71.43 0.01001
-71.34 0.01071
-71.34 0.01094
-71.3 0.01187
-71.3 0.01211
-71.23 0.01281
-71.21 0.0135
-71.07 0.0142
-70.96 0.0149
-70.95 0.01537
-70.81 0.01607
-70.77 0.01723
-70.68 0.01793
-70.68 0.01816
-70.58 0.01909
-70.58 0.01932
-70.46 0.02095
-70.39 0.02212
-70.35 0.02258
-70.35 0.02282
-70.34 0.02305
-70.29 0.02375
-70.28 0.02445
-70.2 0.02747
-70.11 0.02934
-70.03 0.03073
-70.01 0.03143
-70.01 0.03166
-69.96 0.03283
-69.9 0.03329
-69.9 0.03353
-69.86 0.03446
-69.85 0.03492
-69.82 0.03586
-69.8 0.03632
-69.74 0.03702
-69.73 0.03772
-69.72 0.03795
-69.65 0.03888
-69.64 0.03958
-69.59 0.04051
-69.58 0.04098
-69.57 0.04144
-69.54 0.04191
-69.53 0.04261
-69.45 0.04563
-69.45 0.04587
-69.43 0.04633
-69.43 0.04703
-69.42 0.04773
-69.42 0.04796
-69.41 0.04843
-69.38 0.04889
-69.36 0.04983
-69.34 0.05029
-69.32 0.05099
-69.32 0.05169
-69.32 0.05192
-69.24 0.05308
-69.23 0.05355
-69.23 0.05402
-69.2 0.05541
-69.2 0.05565
-69.19 0.05634
-69.19 0.05658
-69.14 0.05751
-69.03 0.06123
-69.01 0.0631
-68.99 0.0638
-68.97 0.06589
-68.97 0.06612
-68.93 0.06775
-68.91 0.06892
-68.88 0.07008
-68.88 0.07078
-68.87 0.07148
-68.86 0.07218
-68.84 0.07288
-68.84 0.07311
-68.81 0.07381
-68.81 0.07404
-68.79 0.07497
-68.79 0.0759
-68.75 0.0766
-68.75 0.07683
-68.74 0.0773
-68.72 0.078
-68.71 0.0787
-68.7 0.07939
-68.7 0.07986
-68.69 0.08102
-68.69 0.08196
-68.67 0.08359
-68.67 0.08382
-68.67 0.08405
-68.65 0.08475
-68.65 0.08591
-68.64 0.08638
-68.63 0.08708
-68.62 0.08731
-68.61 0.08801
-68.6 0.08941
-68.6 0.08987
-68.58 0.09104
-68.57 0.0922
-68.57 0.09313
-68.56 0.09383
-68.56 0.09406
-68.55 0.09476
-68.55 0.09546
-68.55 0.09616
-68.54 0.09802
-68.54 0.09849
-68.53 0.09965
-68.53 0.1001
-68.53 0.1006
-68.53 0.101
-68.53 0.1015
-68.52 0.1024
-68.52 0.1031
-68.52 0.1034
-68.52 0.1038
-68.52 0.1048
-68.52 0.1052
-68.52 0.1057
-68.51 0.1071
-68.51 0.1097
-68.5 0.1115
-68.5 0.1122
-68.5 0.1139
-68.49 0.1148
-68.49 0.1166
-68.48 0.1178
-68.48 0.118
-68.47 0.1183
-68.46 0.119
-68.45 0.1204
-68.45 0.1206
-68.44 0.1213
-68.43 0.1222
-68.42 0.1229
-68.42 0.1239
-68.38 0.1255
-68.36 0.1264
-68.36 0.1267
-68.35 0.1271
-68.33 0.1278
-68.32 0.1288
-68.32 0.129
-68.31 0.1295
-68.31 0.1299
-68.31 0.1311
-68.29 0.1329
-68.29 0.1334
-68.28 0.1341
-68.27 0.1353
-68.27 0.1357
-68.27 0.1362
-68.23 0.1383
-68.23 0.1397
-68.22 0.1404
-68.21 0.1413
-68.2 0.142
-68.19 0.1427
-68.19 0.143
-68.17 0.1448
-68.16 0.1455
-68.15 0.146
-68.15 0.1464
-68.15 0.1471
-68.14 0.1476
-68.14 0.1483
-68.14 0.1488
-68.13 0.1495
-68.13 0.1497
-68.12 0.1506
-68.11 0.1509
-68.1 0.1523
-68.1 0.1527
-68.1 0.153
-68.09 0.1539
-68.08 0.1544
-68.07 0.1551
-68.07 0.1555
-68.05 0.1569
-68.05 0.1574
-68.04 0.1581
-68.04 0.159
-68.03 0.1597
-68.03 0.1602
-68.01 0.1614
-68.01 0.162
-68 0.1625
-67.99 0.1632
-67.98 0.1651
-67.98 0.1655
-67.97 0.1662
-67.97 0.1672
-67.96 0.1679
-67.95 0.1683
-67.95 0.1686
-67.93 0.169
-67.92 0.1704
-67.92 0.1709
-67.91 0.1718
-67.91 0.1725
-67.9 0.173
-67.9 0.1732
-67.88 0.1751
-67.87 0.176
-67.86 0.1765
-67.86 0.1774
-67.86 0.1776
-67.85 0.1783
-67.84 0.1793
-67.84 0.1795
-67.83 0.1802
-67.83 0.1804
-67.82 0.1809
-67.81 0.1818
-67.81 0.1825
-67.79 0.1837
-67.78 0.1853
-67.78 0.186
-67.78 0.1863
-67.74 0.1902
-67.73 0.1912
-67.72 0.1923
-67.72 0.193
-67.72 0.1935
-67.72 0.1939
-67.71 0.1942
-67.71 0.1949
-67.7 0.1956
-67.7 0.1965
-67.69 0.1972
-67.69 0.1974
-67.68 0.1988
-67.66 0.1995
-67.65 0.2014
-67.64 0.2019
-67.63 0.2026
-67.63 0.2035
-67.61 0.2054
-67.58 0.2084
-67.58 0.2088
-67.57 0.2098
-67.56 0.2105
-67.56 0.2109
-67.56 0.2116
-67.55 0.2121
-67.55 0.213
-67.55 0.2135
-67.55 0.214
-67.54 0.2147
-67.54 0.2151
-67.5 0.2207
-67.5 0.2212
-67.49 0.2219
-67.49 0.2226
-67.48 0.2235
-67.48 0.224
-67.47 0.2249
-67.46 0.2272
-67.45 0.2277
-67.43 0.231
-67.43 0.2314
-67.42 0.2331
-67.41 0.2338
-67.41 0.234
-67.4 0.2347
-67.4 0.2354
-67.4 0.2359
-67.39 0.2375
-67.38 0.238
-67.37 0.2405
-67.36 0.2412
-67.35 0.2421
-67.35 0.2428
-67.34 0.2435
-67.34 0.2449
-67.34 0.2456
-67.34 0.2463
-67.33 0.247
-67.33 0.248
-67.33 0.2482
-67.33 0.2487
-67.33 0.2494
-67.33 0.2496
-67.32 0.2503
-67.31 0.2512
-67.31 0.2522
-67.29 0.254
-67.28 0.2554
-67.28 0.2561
-67.27 0.2582
-67.25 0.2589
-67.25 0.2598
-67.25 0.2603
-67.23 0.2622
-67.22 0.264
-67.21 0.2654
-67.21 0.2661
-67.21 0.2664
-67.2 0.2673
-67.19 0.2682
-67.19 0.2685
-67.19 0.2689
-67.18 0.2694
-67.18 0.2696
-67.17 0.2701
-67.17 0.2703
-67.16 0.271
-67.15 0.2722
-67.14 0.2743
-67.13 0.275
-67.13 0.2761
-67.12 0.2771
-67.12 0.278
-67.12 0.2785
-67.12 0.2787
-67.11 0.2799
-67.11 0.2806
-67.09 0.2817
-67.09 0.2824
-67.09 0.2831
-67.08 0.2841
-67.07 0.2857
-67.07 0.2866
-67.06 0.2882
-67.06 0.2887
-67.06 0.2892
-67.05 0.2896
-67.05 0.2901
-67.05 0.2908
-67.04 0.2913
-67.04 0.2922
-67.03 0.2934
-67.03 0.2945
-67.01 0.2985
-67 0.2992
-67 0.3003
-67 0.301
-66.99 0.3027
-66.98 0.3034
-66.98 0.3038
-66.98 0.3043
-66.97 0.3055
-66.97 0.3066
-66.97 0.3071
-66.96 0.3085
-66.95 0.3094
-66.94 0.3101
-66.94 0.3104
-66.93 0.3111
-66.93 0.312
-66.93 0.3129
-66.92 0.3143
-66.91 0.3155
-66.91 0.3159
-66.91 0.3164
-66.91 0.3173
-66.89 0.3194
-66.89 0.3211
-66.88 0.3218
-66.88 0.3227
-66.88 0.3234
-66.86 0.3253
-66.86 0.3255
-66.85 0.3264
-66.85 0.3269
-66.84 0.3283
-66.82 0.3295
-66.82 0.3302
-66.82 0.3306
-66.82 0.3313
-66.81 0.3318
-66.8 0.3336
-66.8 0.3341
-66.79 0.3353
-66.79 0.3362
-66.78 0.3367
-66.78 0.3371
-66.77 0.3392
-66.76 0.3395
-66.74 0.3444
-66.74 0.3446
-66.73 0.3455
-66.73 0.346
-66.72 0.3469
-66.72 0.3471
-66.71 0.3481
-66.7 0.3485
-66.7 0.3492
-66.69 0.3497
-66.69 0.3504
-66.68 0.3511
-66.68 0.352
-66.67 0.3525
-66.67 0.3527
-66.67 0.3532
-66.66 0.3534
-66.66 0.3548
-66.65 0.3551
-66.65 0.3558
-66.65 0.3565
-66.63 0.3593
-66.63 0.36
-66.62 0.3607
-66.61 0.3627
-66.61 0.3639
-66.61 0.3646
-66.59 0.3662
-66.59 0.3667
-66.59 0.3672
-66.58 0.3676
-66.58 0.3679
-66.57 0.369
-66.57 0.37
-66.57 0.3707
-66.54 0.373
-66.54 0.3737
-66.53 0.3746
-66.53 0.3756
-66.53 0.3767
-66.52 0.3776
-66.52 0.379
-66.5 0.3823
-66.49 0.3835
-66.49 0.3839
-66.49 0.3844
-66.48 0.3877
-66.47 0.3881
-66.47 0.3886
-66.47 0.3893
-66.46 0.3898
-66.46 0.3902
-66.46 0.3912
-66.46 0.3914
-66.43 0.3942
-66.43 0.3949
-66.42 0.3956
-66.42 0.3958
-66.41 0.3967
-66.41 0.3972
-66.4 0.3986
-66.4 0.3993
-66.39 0.4
-66.39 0.4005
-66.39 0.4009
-66.38 0.4014
-66.38 0.4023
-66.37 0.4033
-66.37 0.4042
-66.37 0.4044
-66.36 0.4049
-66.36 0.4051
-66.35 0.4061
-66.35 0.4065
-66.34 0.4072
-66.34 0.4077
-66.34 0.4081
-66.33 0.4086
-66.33 0.4091
-66.33 0.4098
-66.32 0.4109
-66.32 0.4114
-66.31 0.4119
-66.31 0.4123
-66.31 0.4128
-66.3 0.4135
-66.3 0.4144
-66.3 0.4149
-66.29 0.4154
-66.29 0.4163
-66.28 0.4172
-66.28 0.4179
-66.28 0.4186
-66.27 0.4198
-66.27 0.4203
-66.27 0.421
-66.26 0.4219
-66.26 0.4228
-66.26 0.4237
-66.25 0.4244
-66.25 0.4249
-66.25 0.4258
-66.25 0.4272
-66.24 0.4279
-66.24 0.4282
-66.23 0.4291
-66.23 0.4296
-66.21 0.4319
-66.21 0.4324
-66.21 0.4333
-66.2 0.4338
-66.19 0.4349
-66.18 0.4354
-66.18 0.4361
-66.18 0.4366
-66.17 0.4373
-66.16 0.4386
-66.15 0.4396
-66.15 0.44
-66.15 0.4405
-66.15 0.441
-66.14 0.4421
-66.14 0.4435
-66.13 0.444
-66.13 0.4445
-66.13 0.4447
-66.12 0.4454
-66.12 0.4461
-66.12 0.4468
-66.11 0.4477
-66.1 0.4484
-66.1 0.4489
-66.09 0.4517
-66.07 0.4531
-66.05 0.4559
-66.05 0.4566
-66.05 0.4575
-66.05 0.458
-66.03 0.4594
-66.03 0.4603
-66.03 0.4608
-66.03 0.461
-66.03 0.4612
-66.02 0.4617
-66.02 0.4626
-66.01 0.4633
-66.01 0.4643
-66 0.4652
-65.98 0.4685
-65.98 0.4687
-65.97 0.4694
-65.97 0.4701
-65.96 0.4708
-65.96 0.4712
-65.96 0.4722
-65.96 0.4729
-65.96 0.4736
-65.95 0.474
-65.95 0.475
-65.95 0.4761
-65.95 0.4764
-65.94 0.4773
-65.94 0.4778
-65.93 0.4789
-65.92 0.482
-65.92 0.4827
-65.91 0.4841
-65.91 0.4843
-65.9 0.4852
-65.9 0.4864
-65.9 0.4866
-65.89 0.4873
-65.89 0.4885
-65.88 0.4899
-65.88 0.4903
-65.88 0.4908
-65.88 0.491
-65.87 0.492
-65.87 0.4927
-65.86 0.4934
-65.85 0.4964
-65.85 0.4969
-65.84 0.4973
-65.84 0.4983
-65.83 0.499
-65.83 0.4994
-65.82 0.5003
-65.82 0.5022
-65.8 0.5029
-65.79 0.505
-65.79 0.5059
-65.78 0.5071
-65.78 0.5076
-65.78 0.508
-65.78 0.5085
-65.77 0.5092
-65.76 0.5106
-65.76 0.5111
-65.76 0.5125
-65.75 0.5134
-65.74 0.5155
-65.74 0.5169
-65.73 0.5187
-65.72 0.5199
-65.72 0.5208
-65.71 0.5222
-65.71 0.5229
-65.71 0.5234
-65.7 0.5243
-65.69 0.5274
-65.68 0.5278
-65.68 0.5281
-65.68 0.5285
-65.67 0.5299
-65.67 0.5304
-65.67 0.5311
-65.66 0.5334
-65.66 0.5336
-65.66 0.5341
-65.65 0.536
-65.64 0.5367
-65.63 0.5385
-65.63 0.5395
-65.62 0.542
-65.62 0.5425
-65.61 0.5434
-65.61 0.5446
-65.6 0.5453
-65.6 0.546
-65.6 0.5464
-65.59 0.5471
-65.59 0.5476
-65.58 0.5481
-65.58 0.5485
-65.57 0.5504
-65.57 0.5516
-65.56 0.5525
-65.55 0.5532
-65.55 0.5534
-65.54 0.5546
-65.54 0.5548
-65.54 0.5553
-65.54 0.5555
-65.54 0.5562
-65.54 0.5567
-65.53 0.5579
-65.52 0.5586
-65.52 0.5593
-65.52 0.5602
-65.52 0.5607
-65.52 0.5611
-65.51 0.5616
-65.51 0.5623
-65.51 0.5627
-65.51 0.563
-65.5 0.5634
-65.5 0.5646
-65.5 0.5655
-65.49 0.5667
-65.48 0.5674
-65.48 0.569
-65.47 0.5695
-65.47 0.5702
-65.46 0.5707
-65.46 0.5711
-65.45 0.5716
-65.45 0.5725
-65.44 0.5751
-65.43 0.576
-65.43 0.5767
-65.42 0.5774
-65.42 0.5776
-65.42 0.5783
-65.41 0.5793
-65.41 0.5797
-65.41 0.5809
-65.4 0.5816
-65.4 0.5823
-65.39 0.583
-65.39 0.5837
-65.38 0.5849
-65.38 0.586
-65.36 0.5881
-65.36 0.5886
-65.35 0.5891
-65.35 0.59
-65.35 0.5907
-65.34 0.5912
-65.34 0.5916
-65.34 0.5919
-65.32 0.5935
-65.31 0.596
-65.3 0.597
-65.3 0.5974
-65.29 0.5981
-65.29 0.5984
-65.28 0.5995
-65.27 0.6014
-65.27 0.6021
-65.26 0.603
-65.25 0.6037
-65.25 0.6044
-65.24 0.6054
-65.23 0.6061
-65.22 0.6065
-65.21 0.6079
-65.2 0.6084
-65.19 0.6095
-65.19 0.6102
-65.18 0.6107
-65.18 0.6109
-65.18 0.6114
-65.18 0.6116
-65.17 0.6123
-65.17 0.6128
-65.16 0.6137
-65.15 0.617
-65.15 0.6172
-65.14 0.6182
-65.14 0.6189
-65.13 0.6198
-65.12 0.6205
-65.12 0.621
-65.12 0.6217
-65.12 0.6224
-65.1 0.6249
-65.1 0.6256
-65.1 0.6261
-65.09 0.627
-65.09 0.6286
-65.08 0.6291
-65.08 0.6298
-65.08 0.6305
-65.07 0.631
-65.07 0.6314
-65.06 0.6321
-65.05 0.634
-65.04 0.6345
-65.04 0.6349
-65.03 0.6356
-65.02 0.638
-65.01 0.6386
-65.01 0.6393
-65.01 0.6396
-65 0.6403
-65 0.6405
-64.99 0.6424
-64.99 0.6428
-64.97 0.6447
-64.96 0.6461
-64.96 0.6466
-64.96 0.648
-64.95 0.6484
-64.95 0.6489
-64.95 0.6494
-64.94 0.6503
-64.93 0.6522
-64.93 0.6529
-64.93 0.6533
-64.93 0.6545
-64.92 0.6549
-64.91 0.6568
-64.9 0.6575
-64.9 0.6582
-64.89 0.6587
-64.89 0.6594
-64.88 0.6596
-64.87 0.6605
-64.86 0.6619
-64.86 0.6624
-64.86 0.6633
-64.84 0.6657
-64.82 0.6678
-64.82 0.6687
-64.82 0.6698
-64.81 0.6703
-64.81 0.6708
-64.81 0.6717
-64.81 0.6719
-64.81 0.6726
-64.8 0.6733
-64.8 0.674
-64.8 0.6745
-64.8 0.6752
-64.79 0.6759
-64.78 0.6761
-64.78 0.6768
-64.77 0.6782
-64.77 0.6792
-64.77 0.6799
-64.77 0.6806
-64.73 0.6829
-64.72 0.6838
-64.72 0.6845
-64.72 0.6847
-64.72 0.6854
-64.71 0.6873
-64.7 0.6882
-64.7 0.6889
-64.69 0.692
-64.68 0.6927
-64.68 0.6934
-64.68 0.6943
-64.66 0.6957
-64.66 0.6962
-64.66 0.6966
-64.65 0.6976
-64.65 0.6983
-64.65 0.699
-64.64 0.7001
-64.64 0.7008
-64.63 0.7015
-64.62 0.7027
-64.62 0.7029
-64.61 0.7034
-64.61 0.7045
-64.6 0.7052
-64.6 0.7064
-64.6 0.7069
-64.6 0.7076
-64.59 0.7083
-64.59 0.709
-64.58 0.7094
-64.58 0.7097
-64.56 0.712
-64.56 0.7127
-64.55 0.7139
-64.55 0.7146
-64.55 0.715
-64.55 0.7155
-64.54 0.7164
-64.53 0.7197
-64.52 0.7201
-64.51 0.7215
-64.51 0.7225
-64.51 0.7227
-64.5 0.7234
-64.5 0.7236
-64.5 0.7241
-64.5 0.7246
-64.49 0.7253
-64.49 0.7267
-64.48 0.7274
-64.47 0.7281
-64.47 0.7288
-64.47 0.7295
-64.47 0.7297
-64.47 0.7304
-64.45 0.7327
-64.45 0.7332
-64.45 0.7341
-64.44 0.7355
-64.44 0.736
-64.43 0.7369
-64.43 0.7385
-64.42 0.7397
-64.41 0.742
-64.41 0.7427
-64.41 0.743
-64.39 0.7437
-64.39 0.7441
-64.37 0.7453
-64.35 0.749
-64.35 0.7499
-64.33 0.7511
-64.32 0.752
-64.32 0.7523
-64.31 0.7527
-64.31 0.7537
-64.29 0.7548
-64.29 0.7555
-64.29 0.756
-64.29 0.7567
-64.28 0.7574
-64.28 0.7576
-64.27 0.7583
-64.27 0.7588
-64.26 0.7597
-64.26 0.76
-64.25 0.762
-64.24 0.7625
-64.24 0.763
-64.23 0.7648
-64.23 0.7653
-64.22 0.7658
-64.22 0.7662
-64.21 0.7669
-64.2 0.7686
-64.19 0.769
-64.19 0.7697
-64.19 0.7704
-64.18 0.7709
-64.18 0.7716
-64.18 0.7723
-64.17 0.7739
-64.16 0.7746
-64.15 0.7753
-64.15 0.7756
-64.15 0.7763
-64.15 0.7765
-64.12 0.7809
-64.12 0.7818
-64.11 0.7823
-64.11 0.783
-64.11 0.7835
-64.11 0.7837
-64.1 0.7846
-64.07 0.7879
-64.07 0.7886
-64.06 0.7895
-64.04 0.7914
-64.03 0.7919
-64.03 0.7925
-64.02 0.7932
-64.02 0.7935
-64 0.7942
-64 0.7951
-64 0.7953
-63.98 0.7963
-63.96 0.7979
-63.94 0.7991
-63.94 0.7998
-63.94 0.8007
-63.94 0.8009
-63.92 0.8019
-63.91 0.8037
-63.91 0.8042
-63.9 0.8049
-63.9 0.8054
-63.9 0.8058
-63.89 0.8061
-63.88 0.8068
-63.87 0.8077
-63.87 0.8081
-63.87 0.8084
-63.86 0.8093
-63.86 0.8095
-63.85 0.8107
-63.84 0.8116
-63.84 0.8123
-63.82 0.8128
-63.82 0.813
-63.82 0.8135
-63.82 0.8137
-63.81 0.8142
-63.8 0.8172
-63.8 0.8179
-63.8 0.8184
-63.78 0.8198
-63.77 0.8217
-63.76 0.8226
-63.76 0.8237
-63.76 0.8242
-63.75 0.8247
-63.72 0.8275
-63.72 0.8282
-63.71 0.8291
-63.7 0.8296
-63.7 0.8303
-63.69 0.8312
-63.69 0.8319
-63.69 0.8324
-63.68 0.8338
-63.68 0.834
-63.64 0.8363
-63.64 0.8366
-63.63 0.8375
-63.63 0.838
-63.61 0.8398
-63.61 0.8403
-63.6 0.841
-63.6 0.8414
-63.6 0.8419
-63.59 0.8426
-63.59 0.8428
-63.59 0.8435
-63.59 0.8438
-63.57 0.8442
-63.56 0.8454
-63.54 0.8487
-63.52 0.8494
-63.52 0.8503
-63.51 0.8512
-63.51 0.8517
-63.5 0.8526
-63.5 0.8529
-63.5 0.8531
-63.49 0.8538
-63.49 0.854
-63.48 0.8547
-63.48 0.8549
-63.47 0.8563
-63.47 0.8568
-63.46 0.8575
-63.46 0.8584
-63.43 0.8617
-63.42 0.8622
-63.42 0.8624
-63.4 0.8643
-63.4 0.8647
-63.4 0.8652
-63.39 0.8654
-63.39 0.8659
-63.39 0.8661
-63.37 0.8668
-63.37 0.8678
-63.36 0.8694
-63.35 0.8698
-63.35 0.8703
-63.32 0.8722
-63.32 0.8726
-63.31 0.8733
-63.31 0.8738
-63.3 0.8745
-63.3 0.8747
-63.29 0.8754
-63.29 0.8759
-63.28 0.8766
-63.26 0.8785
-63.26 0.8796
-63.25 0.8801
-63.25 0.8803
-63.2 0.8831
-63.19 0.8838
-63.16 0.8861
-63.15 0.8866
-63.14 0.8875
-63.13 0.888
-63.13 0.8882
-63.12 0.8896
-63.12 0.8901
-63.1 0.8924
-63.08 0.8943
-63.08 0.895
-63.07 0.8957
-63.07 0.8962
-63.07 0.8969
-63.05 0.898
-63.05 0.8987
-63.04 0.8994
-63.04 0.9006
-63.03 0.9013
-63.03 0.902
-63.03 0.9024
-63.03 0.9029
-63.02 0.9036
-63.02 0.9041
-63 0.9048
-62.99 0.9052
-62.98 0.9066
-62.98 0.9071
-62.97 0.9078
-62.96 0.9083
-62.96 0.9087
-62.95 0.9094
-62.95 0.9099
-62.94 0.9106
-62.94 0.9108
-62.92 0.912
-62.91 0.9127
-62.87 0.915
-62.87 0.9155
-62.84 0.9192
-62.84 0.9194
-62.83 0.9201
-62.82 0.9213
-62.82 0.9218
-62.82 0.9229
-62.81 0.9232
-62.79 0.9239
-62.79 0.9241
-62.77 0.926
-62.77 0.9262
-62.77 0.9267
-62.76 0.9278
-62.76 0.9283
-62.75 0.9297
-62.74 0.9306
-62.73 0.9313
-62.72 0.932
-62.72 0.9325
-62.69 0.9341
-62.68 0.9348
-62.68 0.9355
-62.63 0.9376
-62.61 0.9383
-62.59 0.9404
-62.59 0.9409
-62.58 0.9411
-62.58 0.9416
-62.57 0.942
-62.56 0.9427
-62.54 0.9441
-62.54 0.9448
-62.52 0.9462
-62.51 0.9467
-62.5 0.9474
-62.47 0.9488
-62.47 0.9492
-62.45 0.9502
-62.45 0.9511
-62.42 0.953
-62.41 0.9548
-62.4 0.9555
-62.4 0.9558
-62.39 0.9562
-62.38 0.9569
-62.37 0.9576
-62.37 0.9588
-62.37 0.9593
-62.32 0.9609
-62.28 0.9627
-62.28 0.963
-62.25 0.9641
-62.24 0.9646
-62.23 0.9651
-62.23 0.9655
-62.22 0.9662
-62.22 0.9665
-62.21 0.9669
-62.2 0.9683
-62.19 0.969
-62.18 0.9695
-62.17 0.97
-62.17 0.9704
-62.13 0.9723
-62.13 0.9725
-62.12 0.9732
-62.1 0.9739
-62.1 0.9742
-62.08 0.9746
-62.08 0.9751
-62 0.9795
-61.91 0.9825
-61.85 0.9844
-61.83 0.9851
-61.82 0.9856
-61.82 0.9858
-61.82 0.986
-61.79 0.9867
-61.79 0.987
-61.77 0.9879
-61.66 0.9912
-61.64 0.9916
-61.64 0.9919
-61.58 0.9925
-61.56 0.993
-61.51 0.9953
-61.51 0.9956
-61.48 0.9963
-61.39 0.9977
-61.36 0.9988
-61.14 0.9995
-61.14 0.9998

};
%\addlegendentry{outside color2 - color4}
\addplot [thick, forget plot, color4]
table {%

-66.96 0
-66.83 0.0119
-66.59 0.02381
-66.49 0.03571
-66.49 0.04762
-66.47 0.05952
-66.39 0.07143
-66.36 0.08333
-66.35 0.09524
-66.35 0.1071
-66.32 0.119
-66.26 0.131
-66.21 0.1429
-66.09 0.1548
-66.09 0.1667
-66.09 0.1786
-66.08 0.1905
-66.07 0.2024
-66.07 0.2143
-66.06 0.2262
-66.05 0.2381
-65.93 0.25
-65.91 0.2619
-65.88 0.2857
-65.86 0.2976
-65.85 0.3095
-65.75 0.3214
-65.69 0.3333
-65.64 0.3452
-65.5 0.3571
-65.48 0.369
-65.48 0.381
-65.44 0.3929
-65.43 0.4048
-65.3 0.4167
-65.28 0.4286
-65.21 0.4405
-65.12 0.4524
-64.95 0.4643
-64.92 0.4762
-64.8 0.4881
-64.79 0.5
-64.76 0.5119
-64.74 0.5238
-64.71 0.5357
-64.58 0.5476
-64.53 0.5595
-64.46 0.5714
-64.43 0.5833
-64.42 0.5952
-64.39 0.6071
-64.39 0.619
-64.37 0.631
-64.36 0.6429
-64.36 0.6548
-64.35 0.6667
-64.35 0.6786
-64.34 0.6905
-64.34 0.7024
-64.33 0.7143
-64.3 0.7262
-64.25 0.7381
-64.24 0.75
-64.24 0.7619
-64.23 0.7738
-64.21 0.7857
-64.19 0.7976
-64.06 0.8095
-64.01 0.8214
-64 0.8333
-63.86 0.8452
-63.71 0.8571
-63.69 0.869
-63.59 0.881
-63.33 0.8929
-63.31 0.9048
-63.28 0.9167
-63.27 0.9286
-63.26 0.9405
-63.24 0.9524
-63.24 0.9643
-63.23 0.9762
-63.23 0.9881
};
%\addlegendentry{inside color2 - color4}
\addplot [thick, forget plot, color1, dashed]
table {%

-86.97 0
-86.24 0.0002272
-82.66 0.0009089
-82.04 0.001136
-79.59 0.001818
-79.15 0.002272
-77.95 0.002954
-75.63 0.005226
-75.62 0.005453
-75.57 0.005908
-75.56 0.006135
-75.48 0.006362
-74.79 0.007044
-74.55 0.007953
-74.51 0.00818
-74.19 0.009089
-73.7 0.01022
-73.3 0.01068
-72.26 0.01386
-72.01 0.01454
-72 0.01477
-71.44 0.01613
-71.43 0.01636
-71.41 0.01659
-71.26 0.01727
-71.24 0.0175
-71.08 0.0184
-70.98 0.01909
-70.78 0.01977
-70.69 0.02045
-70.5 0.02113
-70.42 0.02181
-70.37 0.02249
-70.03 0.02409
-69.86 0.02477
-69.85 0.02499
-69.59 0.02613
-69.29 0.02795
-69.17 0.02954
-69.1 0.03045
-68.98 0.03181
-68.44 0.03272
-68.44 0.03295
-67.92 0.03476
-67.76 0.03567
-67.75 0.0359
-67.57 0.03636
-67.57 0.03658
-67.5 0.03726
-67.48 0.03772
-67.45 0.0384
-67.36 0.03908
-67.36 0.03931
-67.24 0.04135
-67.16 0.04204
-66.66 0.04703
-66.59 0.0484
-66.43 0.04931
-66.42 0.05022
-66.18 0.05135
-66.17 0.05181
-66.06 0.05226
-66.03 0.05294
-66.02 0.0534
-65.89 0.05385
-65.68 0.05749
-65.68 0.05771
-65.66 0.0584
-65.64 0.05885
-65.61 0.0593
-65.6 0.05976
-65.47 0.06135
-65.46 0.06226
-65.44 0.06294
-65.44 0.06362
-65.42 0.06408
-65.42 0.0643
-65.39 0.06521
-65.27 0.06748
-65.27 0.06771
-65.16 0.06908
-65.15 0.06953
-64.97 0.07294
-64.97 0.07317
-64.95 0.07385
-64.88 0.07521
-64.85 0.07612
-64.83 0.07726
-64.83 0.07748
-64.78 0.07839
-64.78 0.07862
-64.77 0.07907
-64.73 0.07975
-64.72 0.08021
-64.66 0.08112
-64.64 0.0818
-64.58 0.08294
-64.53 0.08362
-64.53 0.08384
-64.41 0.08589
-64.38 0.08634
-64.35 0.08725
-64.32 0.08793
-64.31 0.08816
-64.26 0.08884
-64.22 0.0893
-64.22 0.08953
-64.1 0.09043
-64.04 0.09112
-64.03 0.09157
-64 0.09225
-63.98 0.09271
-63.95 0.09339
-63.95 0.09407
-63.81 0.09589
-63.77 0.09657
-63.61 0.09861
-63.53 0.1007
-63.52 0.1013
-63.49 0.1022
-63.46 0.1032
-63.44 0.1036
-63.44 0.1038
-63.42 0.1043
-63.42 0.1045
-63.26 0.1063
-63.22 0.1077
-63.22 0.1079
-63.16 0.1088
-63.16 0.1091
-63.15 0.1093
-63.1 0.11
-63.1 0.1102
-63.04 0.1113
-63.03 0.1122
-62.99 0.1129
-62.99 0.1132
-62.97 0.1136
-62.97 0.1138
-62.96 0.1141
-62.88 0.1157
-62.87 0.1161
-62.83 0.1168
-62.8 0.1175
-62.76 0.1182
-62.75 0.1184
-62.72 0.1191
-62.72 0.1193
-62.7 0.1197
-62.7 0.1202
-62.67 0.1211
-62.63 0.1218
-62.58 0.1232
-62.57 0.1236
-62.56 0.1243
-62.55 0.1247
-62.42 0.1266
-62.42 0.1268
-62.28 0.1291
-62.21 0.1297
-62.19 0.1307
-62.18 0.1313
-62.18 0.1316
-62.12 0.1334
-62.09 0.1341
-62.09 0.1343
-62.04 0.1347
-62.03 0.1357
-62.03 0.1359
-62 0.1366
-62 0.1368
-61.94 0.1377
-61.94 0.1379
-61.92 0.1386
-61.9 0.1391
-61.9 0.1397
-61.85 0.1413
-61.83 0.142
-61.78 0.1431
-61.76 0.1436
-61.76 0.1438
-61.7 0.1459
-61.67 0.1475
-61.6 0.1481
-61.6 0.1484
-61.6 0.1491
-61.56 0.1495
-61.56 0.1497
-61.52 0.1513
-61.52 0.1516
-61.48 0.1522
-61.47 0.1527
-61.45 0.1534
-61.45 0.1536
-61.4 0.155
-61.32 0.1575
-61.28 0.1581
-61.27 0.1591
-61.18 0.1609
-61.18 0.1611
-61.15 0.1618
-61.11 0.1629
-61.08 0.1638
-61.07 0.1645
-61.07 0.165
-61.06 0.1654
-61.04 0.1659
-61.03 0.1663
-61 0.1672
-60.96 0.1691
-60.93 0.1697
-60.82 0.1736
-60.79 0.1745
-60.79 0.1747
-60.76 0.1759
-60.73 0.177
-60.68 0.1779
-60.68 0.1781
-60.61 0.1802
-60.6 0.1811
-60.59 0.1818
-60.57 0.1827
-60.56 0.184
-60.56 0.1843
-60.53 0.1852
-60.53 0.1854
-60.52 0.1859
-60.52 0.1861
-60.5 0.1865
-60.49 0.1872
-60.47 0.1879
-60.46 0.1884
-60.45 0.189
-60.45 0.1893
-60.43 0.1906
-60.39 0.1913
-60.39 0.1918
-60.37 0.1925
-60.37 0.1927
-60.35 0.1936
-60.28 0.1968
-60.19 0.1981
-60.14 0.2
-60.11 0.2009
-60.11 0.2011
-60.04 0.2025
-60.04 0.2027
-60.02 0.2034
-60.01 0.204
-59.95 0.2056
-59.94 0.2061
-59.92 0.2068
-59.92 0.207
-59.88 0.2075
-59.88 0.2077
-59.86 0.2084
-59.85 0.209
-59.81 0.2097
-59.81 0.21
-59.79 0.2109
-59.79 0.2125
-59.77 0.2131
-59.76 0.2134
-59.76 0.214
-59.76 0.2145
-59.74 0.2156
-59.72 0.2165
-59.71 0.217
-59.69 0.2177
-59.69 0.2179
-59.68 0.2186
-59.65 0.2199
-59.65 0.2202
-59.59 0.2211
-59.53 0.2245
-59.52 0.2252
-59.51 0.2256
-59.48 0.2272
-59.48 0.2274
-59.46 0.2281
-59.44 0.2297
-59.43 0.2304
-59.43 0.2306
-59.42 0.2311
-59.42 0.2318
-59.41 0.2322
-59.41 0.2327
-59.39 0.2336
-59.38 0.2343
-59.31 0.237
-59.31 0.2372
-59.28 0.2381
-59.25 0.2399
-59.24 0.2409
-59.24 0.2415
-59.24 0.2418
-59.22 0.2427
-59.21 0.2434
-59.17 0.247
-59.13 0.2486
-59.1 0.2502
-59.1 0.2504
-59.04 0.2524
-59.04 0.2529
-59.04 0.2531
-59.02 0.2538
-59.01 0.254
-59 0.2545
-59 0.2549
-59 0.2554
-58.99 0.2561
-58.97 0.2565
-58.97 0.2568
-58.95 0.2574
-58.95 0.2577
-58.94 0.259
-58.94 0.2595
-58.9 0.2613
-58.9 0.262
-58.86 0.2638
-58.83 0.2652
-58.8 0.2663
-58.79 0.2668
-58.78 0.2674
-58.78 0.2679
-58.76 0.2693
-58.73 0.2711
-58.71 0.2718
-58.71 0.2724
-58.69 0.2729
-58.69 0.2733
-58.64 0.2763
-58.64 0.2774
-58.61 0.2781
-58.61 0.2783
-58.58 0.279
-58.58 0.2793
-58.56 0.2804
-58.56 0.2811
-58.54 0.2822
-58.54 0.2824
-58.52 0.2829
-58.52 0.2836
-58.51 0.284
-58.51 0.2843
-58.5 0.2847
-58.49 0.2852
-58.48 0.2858
-58.48 0.2861
-58.48 0.2865
-58.47 0.2872
-58.44 0.2883
-58.44 0.2893
-58.42 0.2899
-58.42 0.2908
-58.4 0.2913
-58.4 0.292
-58.4 0.2924
-58.39 0.2931
-58.38 0.2936
-58.34 0.2947
-58.34 0.2949
-58.33 0.2954
-58.3 0.2965
-58.29 0.2972
-58.26 0.299
-58.24 0.3004
-58.22 0.3024
-58.21 0.3031
-58.21 0.3038
-58.21 0.3042
-58.18 0.3056
-58.18 0.3061
-58.16 0.3072
-58.16 0.3074
-58.14 0.3081
-58.11 0.3099
-58.05 0.3127
-58.04 0.3138
-58 0.3161
-58 0.3163
-58 0.3167
-57.97 0.3174
-57.96 0.3186
-57.95 0.3195
-57.95 0.3197
-57.95 0.3199
-57.92 0.3208
-57.92 0.3211
-57.92 0.3213
-57.91 0.322
-57.91 0.3222
-57.9 0.3229
-57.9 0.3231
-57.88 0.3249
-57.84 0.3265
-57.84 0.3267
-57.83 0.3274
-57.83 0.3279
-57.8 0.3297
-57.78 0.3311
-57.77 0.3317
-57.76 0.3327
-57.71 0.3336
-57.71 0.334
-57.71 0.3342
-57.66 0.3363
-57.65 0.3367
-57.64 0.3372
-57.63 0.3379
-57.63 0.3383
-57.62 0.3397
-57.56 0.3422
-57.55 0.3431
-57.54 0.3438
-57.51 0.3456
-57.49 0.3461
-57.49 0.3467
-57.48 0.3474
-57.46 0.3481
-57.42 0.3504
-57.42 0.3515
-57.38 0.3538
-57.37 0.3542
-57.34 0.3567
-57.33 0.3574
-57.31 0.3586
-57.31 0.3588
-57.28 0.3611
-57.28 0.3615
-57.27 0.3622
-57.25 0.3638
-57.22 0.3676
-57.19 0.3695
-57.18 0.3701
-57.14 0.3736
-57.12 0.3742
-57.09 0.3761
-57.06 0.3767
-57.05 0.3776
-57.04 0.3788
-57 0.3822
-57 0.3824
-57 0.3826
-56.99 0.3833
-56.98 0.3842
-56.97 0.3851
-56.96 0.3858
-56.95 0.3867
-56.95 0.3876
-56.94 0.3883
-56.94 0.3885
-56.93 0.3892
-56.92 0.3897
-56.9 0.3915
-56.89 0.392
-56.89 0.3926
-56.87 0.3938
-56.86 0.3949
-56.84 0.3956
-56.84 0.3958
-56.83 0.3963
-56.82 0.3967
-56.8 0.3983
-56.8 0.3985
-56.79 0.3995
-56.79 0.3997
-56.78 0.401
-56.76 0.4017
-56.72 0.4033
-56.7 0.4038
-56.7 0.4042
-56.67 0.4067
-56.67 0.4072
-56.67 0.4074
-56.65 0.4097
-56.63 0.4104
-56.63 0.4108
-56.63 0.4113
-56.58 0.4129
-56.58 0.4133
-56.56 0.4149
-56.55 0.4158
-56.54 0.4167
-56.53 0.4174
-56.52 0.4179
-56.52 0.4181
-56.51 0.4185
-56.51 0.419
-56.51 0.4195
-56.48 0.4204
-56.48 0.4215
-56.48 0.4217
-56.46 0.4222
-56.45 0.4235
-56.45 0.4238
-56.43 0.4247
-56.43 0.4256
-56.42 0.4272
-56.42 0.4276
-56.41 0.4281
-56.41 0.4285
-56.41 0.4292
-56.39 0.4315
-56.38 0.4322
-56.38 0.4324
-56.37 0.4326
-56.36 0.4331
-56.35 0.4344
-56.34 0.4354
-56.34 0.436
-56.33 0.4367
-56.32 0.4372
-56.31 0.4381
-56.29 0.4399
-56.29 0.4406
-56.27 0.4419
-56.26 0.4429
-56.24 0.4438
-56.23 0.4444
-56.22 0.446
-56.21 0.4469
-56.21 0.4472
-56.2 0.4479
-56.17 0.4501
-56.17 0.4504
-56.15 0.451
-56.15 0.4524
-56.14 0.4531
-56.14 0.4533
-56.14 0.4538
-56.13 0.4544
-56.13 0.4549
-56.12 0.4558
-56.12 0.4565
-56.11 0.4572
-56.11 0.4574
-56.1 0.4579
-56.08 0.4597
-56.06 0.4608
-56.06 0.4615
-56.04 0.4622
-56.04 0.4628
-56.03 0.4635
-56.03 0.4642
-56.02 0.4647
-56.02 0.4651
-56.01 0.466
-55.99 0.4669
-55.99 0.4683
-55.97 0.469
-55.97 0.4699
-55.95 0.4706
-55.95 0.471
-55.94 0.4717
-55.92 0.4731
-55.92 0.4735
-55.91 0.474
-55.9 0.4753
-55.9 0.4756
-55.88 0.4767
-55.88 0.4769
-55.86 0.4774
-55.85 0.4781
-55.84 0.479
-55.84 0.4794
-55.83 0.4799
-55.83 0.4803
-55.81 0.4817
-55.78 0.4831
-55.75 0.4842
-55.75 0.4847
-55.74 0.4853
-55.74 0.4856
-55.73 0.4863
-55.72 0.4867
-55.72 0.4872
-55.72 0.4874
-55.71 0.4885
-55.68 0.4894
-55.68 0.4908
-55.67 0.4915
-55.66 0.4926
-55.65 0.4931
-55.65 0.4933
-55.64 0.4942
-55.64 0.4944
-55.63 0.4949
-55.63 0.4953
-55.61 0.4963
-55.61 0.4967
-55.6 0.4969
-55.59 0.4976
-55.59 0.4978
-55.58 0.4985
-55.58 0.499
-55.57 0.4997
-55.56 0.5001
-55.56 0.5008
-55.55 0.5015
-55.54 0.5022
-55.54 0.5024
-55.54 0.5028
-55.54 0.5033
-55.53 0.5035
-55.53 0.5042
-55.52 0.5051
-55.51 0.5056
-55.48 0.5074
-55.48 0.5078
-55.46 0.5101
-55.45 0.5108
-55.44 0.5122
-55.43 0.5133
-55.42 0.5147
-55.36 0.5174
-55.36 0.5176
-55.35 0.5183
-55.32 0.5215
-55.31 0.5219
-55.31 0.5224
-55.29 0.5231
-55.29 0.5237
-55.28 0.5242
-55.28 0.5244
-55.28 0.5249
-55.28 0.5253
-55.27 0.5258
-55.26 0.5262
-55.25 0.5276
-55.22 0.5283
-55.22 0.529
-55.21 0.5294
-55.19 0.5324
-55.18 0.5331
-55.16 0.5344
-55.16 0.5347
-55.16 0.5353
-55.16 0.5356
-55.14 0.5362
-55.13 0.5378
-55.12 0.539
-55.12 0.5394
-55.11 0.5403
-55.09 0.5412
-55.06 0.5426
-55.04 0.5435
-55.04 0.544
-55.03 0.5446
-55.03 0.5451
-55.02 0.546
-55.01 0.5465
-55.01 0.5471
-55 0.5478
-54.97 0.5496
-54.96 0.5506
-54.96 0.551
-54.95 0.5517
-54.93 0.5531
-54.93 0.5533
-54.91 0.554
-54.91 0.5544
-54.88 0.5551
-54.87 0.5562
-54.87 0.5569
-54.86 0.5574
-54.86 0.5578
-54.84 0.5603
-54.82 0.5615
-54.8 0.5626
-54.79 0.5633
-54.77 0.5642
-54.77 0.5649
-54.76 0.5656
-54.76 0.5658
-54.73 0.5676
-54.73 0.5678
-54.71 0.569
-54.71 0.5694
-54.71 0.5703
-54.7 0.571
-54.69 0.5719
-54.68 0.5726
-54.67 0.5731
-54.66 0.5746
-54.66 0.5756
-54.66 0.5758
-54.64 0.5771
-54.64 0.5776
-54.63 0.5781
-54.63 0.5785
-54.63 0.5796
-54.61 0.5803
-54.61 0.5805
-54.59 0.5812
-54.58 0.5828
-54.58 0.5837
-54.57 0.5844
-54.57 0.5846
-54.56 0.5855
-54.55 0.5862
-54.55 0.5867
-54.54 0.5871
-54.53 0.5883
-54.51 0.5887
-54.51 0.589
-54.49 0.5894
-54.49 0.5905
-54.49 0.591
-54.47 0.5919
-54.47 0.5924
-54.46 0.593
-54.46 0.5935
-54.45 0.594
-54.45 0.5942
-54.35 0.5996
-54.33 0.6003
-54.32 0.6005
-54.32 0.601
-54.32 0.6017
-54.3 0.6035
-54.29 0.6044
-54.28 0.6049
-54.28 0.6053
-54.27 0.6058
-54.27 0.6062
-54.26 0.6069
-54.26 0.6074
-54.25 0.6083
-54.25 0.609
-54.23 0.6103
-54.2 0.6119
-54.18 0.613
-54.17 0.6144
-54.17 0.6149
-54.16 0.6155
-54.16 0.6162
-54.14 0.6171
-54.14 0.6176
-54.14 0.6183
-54.11 0.6199
-54.11 0.6201
-54.11 0.6205
-54.07 0.6224
-54.05 0.6239
-54.05 0.6244
-54.03 0.6253
-54.03 0.626
-54.01 0.6267
-54.01 0.6271
-53.98 0.6299
-53.97 0.6308
-53.96 0.6321
-53.95 0.6326
-53.95 0.633
-53.93 0.6337
-53.92 0.6342
-53.92 0.6349
-53.91 0.636
-53.91 0.6371
-53.91 0.6374
-53.88 0.6394
-53.86 0.6403
-53.86 0.6408
-53.86 0.641
-53.85 0.6417
-53.85 0.6421
-53.84 0.6428
-53.84 0.6437
-53.83 0.646
-53.82 0.6464
-53.82 0.6471
-53.82 0.6483
-53.81 0.6489
-53.8 0.6503
-53.8 0.6508
-53.79 0.6514
-53.78 0.6526
-53.78 0.653
-53.77 0.6539
-53.76 0.6544
-53.76 0.6549
-53.75 0.6555
-53.74 0.6564
-53.74 0.6571
-53.74 0.6574
-53.73 0.6583
-53.73 0.6585
-53.68 0.6617
-53.65 0.6626
-53.65 0.6628
-53.64 0.6639
-53.62 0.6646
-53.62 0.6648
-53.6 0.666
-53.58 0.6667
-53.57 0.6673
-53.56 0.6683
-53.55 0.6696
-53.55 0.6701
-53.54 0.6705
-53.54 0.6708
-53.51 0.6719
-53.5 0.673
-53.49 0.6739
-53.49 0.6748
-53.47 0.6767
-53.45 0.6773
-53.45 0.6776
-53.44 0.678
-53.42 0.6787
-53.41 0.6798
-53.41 0.6808
-53.4 0.6817
-53.39 0.6828
-53.39 0.6835
-53.39 0.6839
-53.39 0.6844
-53.38 0.6848
-53.38 0.6853
-53.37 0.686
-53.37 0.6864
-53.37 0.6869
-53.36 0.6876
-53.36 0.6878
-53.34 0.6898
-53.34 0.6905
-53.34 0.6912
-53.33 0.6921
-53.3 0.6942
-53.28 0.6962
-53.28 0.6964
-53.28 0.6967
-53.27 0.6971
-53.27 0.6978
-53.27 0.6983
-53.26 0.6989
-53.26 0.6992
-53.25 0.7001
-53.23 0.7019
-53.22 0.7026
-53.21 0.7032
-53.21 0.7039
-53.21 0.7046
-53.21 0.7048
-53.15 0.7085
-53.15 0.7087
-53.15 0.7094
-53.14 0.7105
-53.13 0.7114
-53.1 0.7144
-53.09 0.716
-53.08 0.7167
-53.08 0.7173
-53.07 0.7182
-53.07 0.7187
-53.06 0.7194
-53.06 0.7203
-53.06 0.7205
-53.05 0.7212
-53.05 0.7214
-53.04 0.7228
-53.04 0.723
-53.02 0.7239
-53.02 0.7248
-53.01 0.7255
-53.01 0.726
-53 0.7267
-53 0.7269
-52.99 0.728
-52.98 0.7285
-52.98 0.7289
-52.97 0.7301
-52.97 0.7307
-52.96 0.7312
-52.96 0.7319
-52.95 0.733
-52.95 0.7332
-52.95 0.7337
-52.95 0.7339
-52.94 0.7342
-52.92 0.7353
-52.91 0.7364
-52.91 0.7367
-52.9 0.7373
-52.89 0.738
-52.87 0.7387
-52.87 0.7392
-52.86 0.7401
-52.86 0.7405
-52.86 0.741
-52.86 0.7412
-52.84 0.7426
-52.83 0.7437
-52.82 0.7446
-52.81 0.7457
-52.8 0.7464
-52.8 0.7469
-52.8 0.7471
-52.79 0.7485
-52.76 0.7514
-52.76 0.7521
-52.75 0.7526
-52.75 0.7528
-52.72 0.7548
-52.72 0.7557
-52.71 0.7569
-52.69 0.7598
-52.67 0.7605
-52.67 0.7614
-52.66 0.7619
-52.66 0.7626
-52.65 0.7628
-52.64 0.7639
-52.64 0.7641
-52.63 0.7646
-52.63 0.7653
-52.63 0.7662
-52.62 0.7669
-52.62 0.7687
-52.61 0.7701
-52.61 0.7705
-52.61 0.7721
-52.6 0.7737
-52.6 0.7773
-52.59 0.7789
-52.59 0.7794
-52.59 0.7798
-52.59 0.7807
-52.58 0.7812
-52.58 0.7819
-52.58 0.7825
-52.58 0.7837
-52.57 0.7844
-52.56 0.7875
-52.55 0.7882
-52.55 0.7889
-52.52 0.7912
-52.51 0.7919
-52.5 0.7925
-52.49 0.7937
-52.48 0.7944
-52.48 0.7946
-52.47 0.7953
-52.45 0.7973
-52.43 0.7985
-52.42 0.7991
-52.42 0.7996
-52.41 0.8
-52.4 0.8007
-52.4 0.8012
-52.39 0.8019
-52.38 0.803
-52.37 0.8032
-52.37 0.8041
-52.34 0.8048
-52.34 0.805
-52.34 0.8055
-52.34 0.8057
-52.31 0.8071
-52.28 0.8082
-52.26 0.8091
-52.25 0.8098
-52.25 0.81
-52.22 0.811
-52.22 0.8116
-52.21 0.8121
-52.21 0.813
-52.2 0.8137
-52.2 0.8141
-52.19 0.815
-52.18 0.816
-52.18 0.8162
-52.16 0.8166
-52.15 0.8171
-52.13 0.8182
-52.13 0.8185
-52.11 0.8194
-52.11 0.8196
-52.09 0.8203
-52.05 0.8221
-52.04 0.8225
-52.04 0.8234
-52.03 0.8239
-52.02 0.8244
-52.01 0.8246
-52 0.8253
-51.98 0.8273
-51.96 0.8284
-51.96 0.8287
-51.96 0.8291
-51.96 0.8294
-51.93 0.8314
-51.9 0.8325
-51.9 0.8328
-51.83 0.8373
-51.83 0.8378
-51.8 0.8396
-51.8 0.8405
-51.78 0.8409
-51.78 0.8412
-51.75 0.8425
-51.69 0.8459
-51.69 0.8464
-51.68 0.8475
-51.62 0.8505
-51.62 0.8512
-51.62 0.8514
-51.61 0.8521
-51.6 0.8525
-51.59 0.8532
-51.59 0.8539
-51.58 0.8544
-51.58 0.8546
-51.55 0.8553
-51.55 0.8559
-51.54 0.8564
-51.5 0.858
-51.47 0.86
-51.46 0.8605
-51.45 0.8614
-51.45 0.8621
-51.44 0.863
-51.43 0.8637
-51.41 0.8648
-51.39 0.8666
-51.38 0.8673
-51.35 0.8693
-51.35 0.8696
-51.31 0.8714
-51.3 0.8718
-51.3 0.8721
-51.28 0.8725
-51.27 0.8741
-51.25 0.875
-51.24 0.8759
-51.22 0.8766
-51.22 0.8771
-51.19 0.8782
-51.19 0.8784
-51.18 0.8793
-51.15 0.8805
-51.14 0.8807
-51.13 0.8816
-51.1 0.883
-51.08 0.8843
-51.08 0.8846
-51.06 0.8857
-51.06 0.8859
-51.03 0.8868
-51.02 0.8875
-51.02 0.888
-51 0.8887
-50.96 0.8898
-50.96 0.8905
-50.94 0.8914
-50.93 0.8928
-50.88 0.8946
-50.88 0.8948
-50.87 0.8955
-50.87 0.8962
-50.85 0.8971
-50.82 0.8984
-50.8 0.8991
-50.78 0.9002
-50.78 0.9005
-50.71 0.9025
-50.63 0.9052
-50.63 0.9055
-50.62 0.9059
-50.6 0.9068
-50.6 0.9071
-50.58 0.908
-50.57 0.9089
-50.55 0.9098
-50.53 0.9107
-50.47 0.9125
-50.42 0.9132
-50.42 0.9137
-50.4 0.9143
-50.36 0.9148
-50.31 0.9171
-50.31 0.9177
-50.3 0.918
-50.25 0.9187
-50.21 0.9202
-50.2 0.9207
-50.19 0.9216
-50.18 0.9221
-50.14 0.923
-50.09 0.9259
-50.09 0.9262
-50.05 0.9275
-50 0.9282
-49.82 0.9355
-49.82 0.9359
-49.8 0.9368
-49.79 0.9371
-49.77 0.9382
-49.77 0.9384
-49.74 0.9393
-49.73 0.94
-49.73 0.9402
-49.69 0.9416
-49.69 0.9418
-49.61 0.9446
-49.61 0.9448
-49.53 0.9471
-49.52 0.948
-49.49 0.9493
-49.48 0.9498
-49.45 0.9507
-49.45 0.9509
-49.38 0.9523
-49.38 0.9525
-49.31 0.9539
-49.24 0.955
-49.23 0.9555
-49.1 0.958
-49.06 0.9584
-49.06 0.9589
-49 0.96
-49 0.9602
-48.95 0.9609
-48.53 0.9664
-48.52 0.9668
-48.49 0.9675
-48.48 0.9677
-48.43 0.9693
-48.43 0.9696
-48.43 0.9698
-48.25 0.9714
-48.16 0.9721
-48.16 0.9723
-48.14 0.973
-48.13 0.9734
-48.05 0.9746
-48.04 0.9748
-47.99 0.9755
-47.99 0.9757
-47.95 0.9764
-47.94 0.9768
-47.85 0.9775
-47.84 0.9777
-47.78 0.9789
-47.78 0.9793
-47.77 0.9798
-47.74 0.9809
-47.71 0.9818
-47.7 0.9825
-47.62 0.9841
-47.6 0.9848
-47.59 0.9852
-47.56 0.9857
-47.5 0.9864
-47.47 0.9868
-47.47 0.987
-47.32 0.9895
-47.3 0.9902
-47.12 0.9918
-47.11 0.992
-47.07 0.9927
-47.06 0.9934
-47.04 0.9943
-46.96 0.995
-46.82 0.9957
-46.62 0.9968
-46.45 0.9973
-46.45 0.9975
-46.44 0.998
-46.17 0.9998

};
%\addlegendentry{outside color2 - color1 (pi/2)}
\addplot [thick, forget plot, color1]
table {%

-55.97 0
-55.96 0.007752
-55.62 0.0155
-55.62 0.02326
-55.39 0.03101
-55.36 0.03876
-55.36 0.04651
-55.33 0.05426
-55.29 0.06202
-55.23 0.06977
-55.22 0.07752
-55.2 0.08527
-55.18 0.09302
-55.17 0.1008
-55.14 0.1085
-55.13 0.1163
-55.13 0.124
-55.12 0.1318
-55.12 0.1473
-55.11 0.155
-55.11 0.1628
-55.11 0.1705
-55.09 0.1783
-55.08 0.186
-55.08 0.1938
-55.08 0.2016
-55.07 0.2093
-55.07 0.2171
-55.06 0.2248
-55.05 0.2326
-55.04 0.2558
-55.03 0.2713
-55.03 0.2868
-55.02 0.2946
-55.02 0.3023
-55.02 0.3101
-55.01 0.3178
-55.01 0.3256
-55.01 0.3333
-55.01 0.3411
-55.01 0.3488
-55.01 0.3566
-55 0.3643
-55 0.3721
-55 0.3798
-55 0.3876
-54.99 0.3953
-54.99 0.4031
-54.99 0.4109
-54.98 0.4186
-54.97 0.4264
-54.95 0.4341
-54.95 0.4419
-54.89 0.4496
-54.78 0.4574
-54.68 0.4729
-54.58 0.4806
-54.5 0.4961
-54.47 0.5039
-54.45 0.5116
-54.38 0.5194
-54.37 0.5271
-54.35 0.5349
-54.34 0.5426
-54.25 0.5504
-54.24 0.5581
-54.2 0.5659
-54.13 0.5736
-54.12 0.5814
-54.12 0.5891
-54.02 0.5969
-54.01 0.6047
-53.99 0.6124
-53.98 0.6202
-53.98 0.6279
-53.92 0.6357
-53.85 0.6434
-53.85 0.6512
-53.81 0.6589
-53.79 0.6667
-53.68 0.6744
-53.65 0.6822
-53.62 0.6899
-53.61 0.6977
-53.61 0.7054
-53.56 0.7132
-53.54 0.7209
-53.53 0.7287
-53.5 0.7364
-53.49 0.7442
-53.46 0.7519
-53.4 0.7597
-53.39 0.7674
-53.39 0.7752
-53.38 0.7829
-53.32 0.7907
-53.3 0.7984
-53.28 0.8062
-53.28 0.814
-53.22 0.8217
-53.22 0.8295
-53.22 0.8372
-53.21 0.845
-53.2 0.8527
-53.17 0.8605
-53.17 0.8682
-53.03 0.876
-53.03 0.8837
-53.02 0.8915
-53.01 0.8992
-52.91 0.907
-52.89 0.9147
-52.79 0.9225
-52.78 0.9302
-52.76 0.938
-52.66 0.9457
-52.62 0.9535
-52.53 0.9612
-52.51 0.969
-52.36 0.9767
-52.34 0.9845
-52.26 0.9922

};
%\addlegendentry{inside color2 - color1 (pi/2)}


\addlegendimage{no markers, black, line width=1pt}
\addlegendimage{no markers, black, dashed, line width=1pt}

\addlegendimage{empty legend}

\addlegendimage{draw opacity=0, area legend, no markers, fill=color2}
\addlegendimage{draw opacity=0, area legend, no markers, fill=color3}
\addlegendimage{draw opacity=0, area legend, no markers, fill=color4}
\addlegendimage{draw opacity=0, area legend, no markers, fill=color1}


\legend{Inside focal region, Outside focal region, \phantom{placehold}, strategy 1, strategy 2,strategy 3, strategy 4}
\end{axis}

\end{tikzpicture}

\end{figure}

\FloatBarrier%
\section{Using abbreviations}










