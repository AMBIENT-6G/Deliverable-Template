


\chapter{Introduction}\label{ch:introduction}

This deliverable presents the proof-of-concept developments from Task 5.1 and Task 5.2 of \projectname, showcasing the smart connectivity platforms enabled by the RadioWeaves architecture. It validates the functionality of resilient wireless applications in dynamic scenarios and demonstrates their interaction with energy-neutral devices. The deliverable highlights that these improvements in service levels are achievable with practical implementation complexity.

\section*{Relation to Other Deliverables}
Although D5.2 is officially classified as a \emph{Demonstrator}, this document is provided to consolidate and summarize the proof-of-concept achievements from \projectname. D5.1 contained the experimental plans in \projectname. The conducted experiments together with their description, methodologies, and experimental findings are provided in D5.3. These are distilled into demonstration videos in D5.2 and described here. These videos offer a concise and visual representation of the results disseminated in D5.3.

\subsection*{Demonstration Videos}
The deliverable includes a description of a series of demonstration videos, each addressing specific challenges tackled in \projectname. The videos cover:
\begin{enumerate}
    \item Overview of testbeds, simulators, and experiments. (\cref{ch:overview})
    \item Experiment 1: Dynamic resource allocations via federations. (\cref{ch:exp1})
    \item Experiment 2: Robustness versus latency trade-offs. (\cref{ch:exp2})
    \item Experiment 3: Accurate positioning. (\cref{ch:exp3})
    \item Experiment 4:
    \begin{enumerate}[label=\alph*)]
        \item Powering energy-neutral devices. (\cref{ch:exp4a})
        \item Communicating with energy-neutral devices. (\cref{ch:exp4b})
        \item Positioning energy-neutral devices. (\cref{ch:exp4c})
    \end{enumerate}
    \item Experiment 5: Multi-user capabilities. (\cref{ch:exp5})
\end{enumerate}

These videos will be made publicly available on social media, the \projectname website, and through the \projectname~results video showcase at:
\begin{itemize}
    \item[] \url{https://vimeo.com/showcase/reindeer-results-video-showcase}.
\end{itemize}


\subsection*{Deliverable Structure}
The deliverable is organized as follows:
\begin{itemize}
    \item \Cref{ch:overview} summarizes the testbeds, simulators, and experiments, including their contributions to various challenges.
    \item Chapters~\ref{ch:exp1}-\ref{ch:exp5} describe each demonstration video, following a standardized format:
    \begin{itemize}
        \item Aim of the demonstrator.
        \item Key takeaways.
        \item Experimental results and conclusions.
        \item Video description.
        \item References to published or submitted materials from the \consortium.
    \end{itemize}
\end{itemize}

This document not only bridges the insights from previous tasks but also provides a clear and engaging format to disseminate the project's key findings to diverse audiences. 


% This deliverable provides the proof-of-concept realized in both Task 5.1 and Task 5.2 of \projectname. It shows the smart connectivity platforms realized by the RadioWeaves architectures, validate resilient wireless applications in a dynamic scenario, and demonstrate the extended interaction with energy-neutral devices. It demonstrates that the improved service levels can be achieved with feasible implementation complexity.


% \textbf{How D5.2 related to D5.1 and D5.3.} 
% D5.2 is officially a \emph{Demonstrator}. Despite this, the demonstrators and proof-of-concepts realized in \projectname, are summarized in this document. The description, methodology, and conclusions of the conducted experiments are introduced in D5.1 and detailed in D5.3. The results disseminated in D5.3 are distilled into demonstration videos, described in this deliverable.

% The following demonstration videos are created, each focusing on a subset of challenges addressed in \projectname:
% \begin{itemize}
%     \item Overview of testbeds, simulators and experiments
%     \item Experiment 1 - Dynamic resource allocations via federations
%     \item Experiment 2 - Robustness versus latency trade-off
%     \item Experiment 3 - Accurate positioning
%     \item Experiment 4a - Powering energy-neutral devices
%     \item Experiment 4b - Communicating with energy-neutral devices
%     \item Experiment 4c - Positioning of energy-neutral devices
%     \item Experiment 5 - Multi-user capabilities
% \end{itemize}

% All the demonstration videos will be published on our socials and available at \url{https://vimeo.com/showcase/reindeer-results-video-showcase}. 


% \textbf{Deliverable outline.}
% The available testbeds, simulators and experiments are summarized in~\cref{ch:overview}. Also, how the testbeds and simulators contribute to the different experiments is included. The following chapters each describe the demonstration video of the corresponding experiments~(1-5). Each experiment chapter follows the outline:
% \begin{itemize}
%     \item aim of the demonstrator/demonstration video
%     \item key takeaway points of the demonstration
%     \item experimental results and conclusions
%     \item references to submitted or published materials from the \consortium.
% \end{itemize}






