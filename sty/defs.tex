
%\AddToHook{cmd/section/before}{\FloatBarrier\clearpage} % all sections should start on a new page and add a floatbarrier, so no floating between sections.

% ---------------------------------------------------------
% Some extra stuff to handle comments and TODOs
% ---------------------------------------------------------





% Using the TODO package to keep track of document todos. ADD "disable" to parameters to get a "clean" document

\usepackage[backgroundcolor=pcolorlight!30!white,
linecolor=pcolorlight, bordercolor=pcolorlight]{todonotes}

\usepackage[backend=biber,style=ieee, maxnames=99]{biblatex}
\addbibresource{bib.bib}
\AtBeginBibliography{\footnotesize}


\usepackage[export]{adjustbox}
\usepackage[nochapter]{vhistory}



\usepackage{xifthen}
\newcommand{\contents}[2]{\todo[inline]{%
		\textbf{Add content:} #1
		%\ifthenelse{\isempty{#2}}%requires xifthen
		%{}%
		%{
		\newline%
		\textit{Responsible partner:} #2}%
		%}
}
		
\newcommand{\missing}[2]{\todo[inline]{%
		\textbf{Add to document structure:} #1
		\newline%
		\textit{Responsible partner:} #2}}


\newcommand{\highlightChanges}{true} %                  set "false" to make changes appear as normal text
\ifthenelse{\equal{\highlightChanges}{true}}%           assert variable
        {\newcommand{\changes}[1]{{{\color{orange}#1}}}}%  highlight changes if "true"
        {\newcommand{\changes}[1]{{#1}}}%               do not highlight else


\usepackage{tcolorbox}

%When adding notes, nothing is added to the TODO list. Only a RED text in the document. Second argument to identify who wrote the note.
\newcommand{\note}[2]{\textcolor{red}{\textbf{Note:} #1 [#2]}}
%\newcommand{\note}[2]{}

\usepackage{array, booktabs, xltabular} % tables
\usepackage{tabularx}
\usepackage{tcolorbox}
\usepackage{bm}			% for \bm ... bold math 
\usepackage{siunitx}    		% for \SI{}{} units
\DeclareSIUnit{\dBm}{dBm}	% SI unit "dBm"
\DeclareSIUnit{\eq}{eq}	    % unit "eq" (equivalent)
\usepackage{layouts}
\usepackage{overpic}		% for the overpic environment (annotations)
\usepackage{fancyvrb}
\usepackage{xspace}
\usepackage{amsmath}
\usepackage{amssymb}
\usepackage{mathtools}
\usepackage{amsthm}
\usepackage{multirow}
\usepackage{algorithm}
\usepackage{algpseudocode}

\usepackage{float}
\usepackage{threeparttable}




%%%%%%%%%%%%%%%%%%%%%%%%%%%%%%%%%%%%%%%%%%%%%%%%%%%


\newcommand{\colonlist}[1]{% this command treats the last item differently...
  \def\process{\def\process{, }}%
  \def\do##1{\process##1}%
  \dolistloop{#1}}




%%%%%%%%%%%%%%%%%%%%%%%%%%%%%%%%%%%%%%%%%%%%%%%%%%%%%%%%%%%
%%%%%%%%%%%%%%%%%%% PARTNER COMMANDS %%%%%%%%%%%%%%%%%%%%%%
%%%%%%%%%%%%%%%%%%%%%%%%%%%%%%%%%%%%%%%%%%%%%%%%%%%%%%%%%%%
\newcommand{\update}[1]{{\color{blue}[UPDATE: #1]}}


%%%%%%%%%%%%%%%% KUL %%%%%%%%%%%%%%%
\newcommand{\kul}[2][]{{\color{BurntOrange}[KUL #1: #2]}}
\newcommand{\kulColor}[1]{{\color{BurntOrange} #1}}
\newcommand{\gilles}[1]{\kul[gilles]{#1}}
\newcommand{\atkul}{{\color{BurntOrange} @KUL}\xspace}

%%%%%%%%%%%%%%%% TUG %%%%%%%%%%%%%%%
\newcommand{\tug}[2][]{{\color{red}[TUG #1: #2]}}
\newcommand{\tugColor}[1]{{\color{red} #1}}
\newcommand{\tw}[1]{\tug[Thomas]{#1}}
\newcommand{\attug}{{\color{red} @TUG}\xspace}

%%%%%%%%%%%%%%%% LU %%%%%%%%%%%%%%%
\newcommand{\lu}[2][]{{\color{LimeGreen}[LU #1: #2]}}
\newcommand{\luColor}[1]{{\color{LimeGreen} #1}}
\newcommand{\atlu}{{\color{LimeGreen} @LU}\xspace}


%%%%%%%%%%%%%%%% EAB %%%%%%%%%%%%%%%
\newcommand{\eab}[2][]{{\color{BlueGreen}[EAB #1: #2]}}
\newcommand{\eabColor}[1]{{\color{BlueGreen} #1}}
\newcommand{\ateab}{{\color{BlueGreen} @EAB}\xspace}

%%%%%%%%%%%%%%%% LiU %%%%%%%%%%%%%%%
\newcommand{\liu}[2][]{{\color{Maroon}[LiU #1: #2]}}
\newcommand{\liuColor}[1]{{\color{Maroon} #1}}
\newcommand{\atliu}{{\color{Maroon} @LiU}\xspace}

%%%%%%%%%%%%%%%% NXP %%%%%%%%%%%%%%%
\newcommand{\nxp}[2][]{{\color{NavyBlue}[NXP #1: #2]}}
\newcommand{\nxpColor}[1]{{\color{NavyBlue} #1}}
\newcommand{\atnxp}{{\color{NavyBlue} @NXP}\xspace}

%%%%%%%%%%%%%%%% BlooLoc %%%%%%%%%%%%%%%
\newcommand{\blooloc}[2][]{{\color{Fuchsia}[BlooLoc #1: #2]}}
\newcommand{\bloolocColor}[1]{{\color{Fuchsia} #1}}
\newcommand{\atblooloc}{{\color{Fuchsia} @Blooloc}\xspace}

%%%%%%%%%%%%%%%%%%%%%%%%%%%%%%%%%%%%%%%%%%%%%%%%%%%%%%%%%%%
%%%%%%%%%%%%%%%%%%%%%%%%%%%%%%%%%%%%%%%%%%%%%%%%%%%%%%%%%%%







\usepackage{pdflscape} % for landscape pages with auto rotation
\usepackage[capitalise,noabbrev]{cleveref}
%\renewcommand{\cref}[1]{\Cref{#1}} % ensure always write in full




\newcommand{\ac}[1]{\gls{#1}} % legacy way of using abbreviations


\usepackage{minted}

\newcommand{\reviewcomment}[1]{\textcolor{ambientdarkgreen}{\textbf{Review Comment}: #1}\newline}


\usepackage{enumitem}
\usepackage{svg}

%------------------------------------------------
% naming defintions 
\newcommand{\projectname}{AMBIENT-6G\xspace}
\newcommand{\consortium}{\projectname consortium\xspace}
\newcommand{\rpi}{\gls{rpi}~4\xspace}
\newcommand{\ie}{\textit{i.e.},\xspace} 
\newcommand{\eg}{\textit{e.g.},\xspace} 
%------------------------------------------------



%----------------------------------------------------------------------------------------
%	PROJECT INFORMATION
%----------------------------------------------------------------------------------------
\def\pAcronym{AMBIENT-6G}
%------------------------------------------------